%%%%%%%%%%%%%%%%%%%%%%%%%%%%%%%%%%%%%%%%%%%%%%%%%%%%%%%%%%%%%%%%%%%%%%%
\subsection{多孔質体モデル}
図-\ref{fig:fig3}に,ランダムウォーク・シミュレーションに用いる4種類の
多孔質体構造(モデルタイプ1$\sim$4)を示す.これらは固相粒子配置の規則性,粒径分布,
粒子形状がマクロ拡散挙動に与える影響を調べることを意図したものである.
図-\ref{fig:fig3}でグレーの部分は固相を,黄色の部分は間隙を表し,
間隙水が存在しない$(S_r=0)$状態での多孔質体モデルを示している. 
モデルタイプ1は,直径$\frac{W}{10}$の円形粒子を,ユニットセル内で10$\times$10の正方格子状に配置したもので,
飽和度が拡散係数に与える影響や,拡散係数の大小を議論する際の参照モデルの役割を果たす.
モデルタイプ2は,直径$\frac{W}{10}$の円形粒子100個をユニットセル内にランダムに配置したものである.
粒子直径と粒子数$N_p$はモデルタイプ1と同じであることから,間隙率$n$も互いに等しく,
その値は
\begin{equation}
	n=1-\frac{\pi}{4}\simeq 0.215
	\label{eqn:n_val}
\end{equation}
である.一方,モデルタイプ3と4は,粒径分布の影響をみるためのもので,
タイプ3は円形粒子を,タイプ4はアスペクト比$a$が$0.4\sim 1.0$の楕円粒子を
充填したものである.なお,アスペクト比$a$は一様分布で,粒子直径はガンマ分布に従って
ランダムに与える.ただし,いずれのモデルも間隙率$n$は式(\ref{eqn:n_val})程度と
なるよう粒子数を設定した.また,粒子直径は最小値を$\frac{W}{100}$最大値を$\frac{W}{5}$とし,
平均粒子直径が約$0.085W$,粒子直径の標準偏差が約$0.05$,粒子数$N_p$が100前後と
なるようにした.さらに,モデルタイプ2$\sim$4は不規則な充填構造であるため,
固相粒子の粒径や数,配置に応じてマクロ拡散係数値も変動すると予想される.
そこで,これら3種類のモデルタイプについては,粒子位置や粒径が異なる20個のモデルを作成し,
各々のモデルについて複数の飽和度でランダムウォークによる拡散シミュレーションを行った.
以上,タイプ1から4のモデルに関する
特徴を表\ref{tbl:types}にまとめて示す.
\begin{table}[htb]
  \caption{多孔質構造(タイプ1$\sim$4)の特徴}
%{\small
  \begin{tabular}{c|c|c|c|c}
\hline
    タイプ & 1 & 2 & 3 & 4 \\
	\hline\hline
	  充填構造& 単純立方& \multicolumn{3}{c}{ランダム} 
%	  & ランダム &ランダム 
	  \\
    \hline
粒子形状 & 円 &円 & 円 & 楕円\\
	\hline
   平均粒径 & 0.1 & 0.1 & 0.089 & 0.082\\ 
 (標準偏差) & (0) & (0) & (0.047) & (0.050) \\
\hline
粒子数 & 100 & 100 & 94$\sim$119 & 81$\sim$145 \\
(平均) & (100) & (100) & (107) & (111) \\
\hline
モデル数& 1 &20 &20 & 20 \\
\hline
  \end{tabular}
\label{tbl:types}
%}
\end{table}
\begin{figure}[t]
\begin{center}
%\includegraphics[clip,scale=0.4]{Figs/fig2.eps}
\caption{
	数値シミュレーションに用いた4種類の多孔質構造モデル. 
}
\label{fig:fig3}
\end{center}
\end{figure}
%%%%%%%%%%%%%%%%%%%%%%%%%%%%%%%%%%%%%%%%%%%%%%%%%%%
\subsection{間隙水の配置}
モデルタイプ1$\sim$4の多孔質構造モデルそれぞれについて,飽和度$Sr$を0.2から1.0まで0.1刻みで
変化させた不飽和多孔質体モデルを作成する.なお,タイプ1については,固相粒子の配置は一通りに
決まっているため,間隙水の分布だけが異なる20通りのモデルを各飽和度で作成した.
すなわち,4種類のモデルタイプ,9段階の飽和度で各20個の多孔質体モデルを用い,
合計720ケースのランダムウォーク・シミュレーションを行う.
各多孔質体モデルは,水分配置を十分な空間解像度で表現するため,ユニットセルの一辺を1,024分割し,
セルサイズを$h=\frac{W}{1,204}$とした.
液相配置決定のためのモンテカルロ・シミュレーションで必要となる界面自由エネルギー$\gamma_{\alpha\beta}$の値は,
$\alpha=1$で気相を,$\alpha=2,3$でそれぞれ液相と固相を表すとき,
\begin{equation}
	\gamma_{12}=70.0,\,  \gamma_{23}=370.0, \, \gamma_{31}=420.0, \ \ 
	[{\rm mJ/m}]
	\label{eqn:gamma_val} 
\end{equation}
とした\cite{Berkowitz}.これは,気相として空気を,液相として水を,固相として石英を想定したもので,
固相表面の濡れ角\cite{Gennes}が$45$度となる親水的な表面に相当する.
なお,界面自由エネルギーの最小化では,$\gamma_{\alpha\beta}$の値は互いの比だけが影響し,
近傍セルの接触面積$\Delta s$や$\gamma_{\alpha\beta}$の絶対値には依存しない.
以上の計算条件に対して得られた不飽和多孔質体モデルの一例を図\ref{fig:fig4}に示す.
これは,モデルタイプ2で,飽和度を$Sr=0.2,0.4,0.6$および$0.8$としたときの結果を示したもので,
飽和度が低い場合には固相粒子間の狭隘部に間隙水が集中し,飽和度が比較的高い場合には
固相粒子直径程度の気泡を残しつつ,次第に広い間隙部分を水分が埋めていく様子が現れている.
\begin{figure}[t]
\begin{center}
%\includegraphics[clip,scale=0.4]{Figs/fig3.eps}
\caption{
	4つの異なる飽和度における間隙水分布の様子.
	モデルタイプ2,飽和度$Sr$が(a) 0.2,(b) 0.4,(c) 0.6,(d) 0.8の場合を示す. 
}
\label{fig:fig4}
\end{center}
\end{figure}
%%%%%%%%%%%%%%%%%%%%%%%%%%%%%%%%%%%%%%%%%%%%%%%%%%%
\subsection{ランダムウォークとマクロ拡散係数の評価}
%本研究では,ランダムウォークを液相セルの中心をノードとする格子上で行うため,
%空間ステップ長は液相セルのサイズ$h=\frac{W}{1,024}$に等しい. 
ユニットセル内各点からの影響が反映された拡散特性を調べるためには,
液相内にある全てのランダムウォークノードを,いずれかの拡散粒子が
繰り返し訪れることができるように拡散粒子数$N_{wk}$や時間ステップ$N_t$を設定する
必要がある.本研究で用いる多孔質体モデルは,ユニットセル中には,およそ100個の固相粒子がある.
そこで,各固相粒子の周辺におよそ100個の拡散粒子が常時存在する状況が生じるように, 
ここでは1万個のランダムウォーカーをユニットセル中に一様に分布させた状態から
ランダムウォーク・シミュレーションを行った.
また,適切な時間ステップ数$N_t$を設定するために,
式(\ref{eqn:abnormal})を無次元化し,以下に示す無次元化時間を元に$N_t$を与えた.\\
式(\ref{eqn:abnormal})には時間と長さの次元を持つ量が含まれる.
そこで,ユニットセルサイズ$W$を長さの基準にとり,変位成分$u_i,v_i$を
\begin{equation}
	U_i=\frac{u_i}{W},\ \  V_i=\frac{v_i}{W}
\end{equation}
と無次元化する.時間に関しては,
\begin{equation}
	\tau=\frac{D_0}{W^2}t
	\label{eqn:tau}
\end{equation}
により,液相の拡散係数$D_0$のスケールにあった無次元化時間を導入する.
このとき式(\ref{eqn:abnormal})を
\begin{equation}
	\left<U_i\right>=\left<V_i\right>=2\bar K \tau ^{\alpha}
	\label{eqn:Kb}
\end{equation}
と書けば,一般化拡散係数$\bar{K}$は
\begin{equation}
	\bar{K}=\frac{K}{D_0W^{2(1-\alpha)}}
	\label{eqn:Kb_def}
\end{equation}
で与えられる無次元量となる,
ここで,単位時間$\tau=1$は$t=W^2/D_0$に相当し,この時刻を式(\ref{eqn:Einstein})に代入すれば
\begin{equation}
	\left<u_i^2\right> =\frac{2D}{D_0}W^2
\end{equation}
となる.従って,$D$と$D_0$のオーダーが大きく異ならない限り,$\tau=1$の時点では,
ユニットセル内で拡散が十分進行した状態にあると言える.
またこのことは,ユニットセルにおける拡散挙動は,$\tau<1$程度の時間スケールで調べれば
十分であることを意味する.そこで,以下のシミュレーションでは$\tau=1/4$となる
時間ステップまでランダムウォークを行うこととする.
$\tau=1/4$に達するための時間ステップ数は,式(\ref{eqn:dt})と式(\ref{eqn:tau})より,
\begin{equation}
	N_t=\left. \frac{t}{\Delta t} \right|_{\tau=1/4}=\frac{W^2}{h^2}=1,048,576
\end{equation}
と与えられる.\\
\hspace{\parindent}
以上の設定で,無限に広い液相領域における拡散解析をランダムウォークで行った
結果を図-\ref{fig:rwk0}に示す.
この図において,(a)は無次元化した平均2乗変位$\left<U^2\right>$と時間$\tau$の関係を,
(b)は$\tau=0.23$における拡散粒子の変位分布を表している.
なお,図\ref{fig:rwk0}-(b)に示されるように,拡散はほぼ等方的であることから,
$X$方向変位$U_i$,$Y$方向変位$V_i$を区別せずに平均した,
\begin{equation}
	\left< U^2 \right>=\frac{
			\left<U_i^2 \right>
		+
		\left<V_i^2 \right>
	}{2}
	=\sum_{i=1}^{N_p}
	\frac{U_i^2+V_i^2}{2N_p}
	\label{eqn:Ubar}
\end{equation}
を,図\ref{fig:rwk0}-(a)では示している.この結果から一般化拡散係数$\bar K$と
べき指数$\alpha$を求めると,それぞれ
\begin{equation}
	\bar{K}=0.9996, \ \ \alpha=1.0000
\end{equation}
となる.この問題では拡散の妨げとなる固相や気相領域が存在しないことから,
マクロ拡散係数$D$とミクロ拡散係数$D_0$は同じである.
従って,$\bar K=1, \alpha=1$が正解であり,ランダムウォークでの拡散解析によって
非常に近い値が得られていることが分かる.
また,図\ref{fig:rwk0}-(b)の
変位分布について,共分散行列を
\begin{equation}
	\fat{S}( \{U_i\},\{V_i\})
	=\left(
	\begin{array}{cc}
		S_{UU} & S_{UV}	 \\
		S_{VU} & S_{VV} 	
	\end{array}
	\right)
\end{equation}
として,各分散値を計算すれば,
\begin{equation}
	\bar{S}= \frac{S_{UU}+S_{VV}}{2}=0.4604
\end{equation}
\begin{equation}
	\fat{S}( \{U_i\},\{V_i\})
	=
	\bar{S}
	\left(
	\begin{array}{cc}
		 0.9897 & -0.0008 \\
		-0.0008 &  1.0103
	\end{array}
	\right)
\end{equation}
であった.このように,対角項は1$\%$程度の差で一致し,非対角項はそれよりも3桁程度は小さく
ほぼ等方的な拡散が再現されていることが分かる.以上のように,拡散場は等方的で平均2乗変位の
グラフが滑らかに推移し,マクロ拡散係数も正しく得られていることから,拡散粒子数も十分で
あったと考えられる.
\begin{figure}[t]
\begin{center}
%\includegraphics[clip,scale=0.32]{Figs/fig10.eps}
\caption{
	ランダムウォークによる無限液相領域における拡散解析の結果.
	(a) 平均2乗変位の時間変化. (b)$\tau=$0.23における拡散粒子の変位分布. 
}
\label{fig:rwk0}
\end{center}
\end{figure}

