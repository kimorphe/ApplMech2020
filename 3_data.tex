%%%%%%%%%%%%%%%%%%%%%%%%%%%%%%%%%%%%%%%%%%%%%%%%%%%%%%%%%%%%%%%%%%%%%%%
\subsection{波動伝播状況の可視化}
実験で得られた時間波形データの全体を,
\begin{equation}
	{\cal D}=\left\{
		a(x,y,t)\left| (x,y,t)\in {\cal G}\times [0,T_d] \right.
	\right\}
	\label{eqn:dataset}
\end{equation}
と表す.ただし,$T_d$(=200$\mu$s)は計測時間範囲を意味する.
図-\ref{fig:fig5}は,計測データ$\cal D$から同一時刻の振幅を取り出して作成した鉛直振動場のスナップショットで,観測時間は$t=21\mu$秒と$t=23\mu$秒である.
これらの時間$t$は,圧電素子に電圧を印加して駆動した時間を$t=0$として測ったもので,探触子のウェッジ内部を超音波が伝わる時間も含まれている.
図-\ref{fig:fig5}の上段は花崗岩供試体を,下段はアルミニウム供試体を用いて計測した結果を示し,
いずれもオシロスコープで計測された波形振幅値[mV]をカラ−表示したものである.
均質なアルミニウム供試体の場合,若干のゆらぎはあるものの,概ね波形を保ったまま右($x>0$)方向に超音波が伝播し,鉛直方向に伸びる直線的な波面がはっきりと観察できる.
一方,強い不均質性を持つ花崗岩供試体では,アルミニウムと同程度の速度で波動場が右方向へ進むものの,振幅のゆらぎは大きい.初動の到達位置もあまり明確ではなく,大きな振幅を持つ振動が初動到達後も継続する様子が見られる.ここで,$a(x,y,t)$の時間$t$に関するフーリエ変換を
\begin{equation}
	A(x,y,\omega)=\int a(x,y,t)e^{i\omega t} dt
	\label{eqn:Fourier_t}
\end{equation}
とし,フーリエ変換$A(x,y,\omega)$の位相
\begin{equation}
	\Phi(x,y,\omega)={\rm Arg}\left\{ A(x,y,\omega) \right\}
	\label{eqn:def_Phi}
\end{equation}
を求める.ただし,$\omega$は角周波数を,Arg($\cdot$)は複素数の偏角の主値をとる
操作を表す.$\Phi$の範囲は
\begin{equation}
	-\pi < \Phi \leq \pi, 
	\label{eqn:dom_phi}
\end{equation}
$A$と$\Phi$の関係は
\begin{equation}
	A=\left| A \right|e^{i\Phi}
	\label{eqn:phi2A}
\end{equation}
である.
図-\ref{fig:fig6}に,$\cal D$からFFTによって求めた位相$\Phi(x,y,\omega)$の空間分布を示す.
これらは,周波数0.7MHzと1.0MHzにおける結果で,位相が一定となる点を結んだ曲線は波面を表すと解釈できる.
アルミニウム供試体の場合,上下($y$方向)へほぼ直線的に伸びる波面群が形成されていることが分かる.
周波数0.7MHzの場合,$x=0$付近で若干波面の屈曲が見られるが,これは,信号成分の周波数帯域下限に近くノイズの影響を受けやすいことと,送信領域$\cal S$の端部で発生する回折波の影響が低周波側で発生し易いことによると考えられる.
一方,花崗岩供試体では,いずれの周波数でも場所によらず波面は著しく屈曲している.
波面の伸展方向は大局的には$y$軸に並行しており配向性が認められるが,個々の波面がどのような曲線を描いているかを視認することは困難である.なお,図-\ref{fig:fig5}から明らかなように,超音波は全体として$x>0$方向に進行している.そのため,観測領域全体では,位相は$x>0$方向へ増加の傾向を示す.
ただし,ここでは位相の範囲を式(\ref{eqn:dom_phi})のようにしているため,位相の増加は$-\pi$から$\pi$の間にとどまり,$\pi$を超えたところで負の側に折り返される.その結果,鋸刃状の変動を繰返すパターンが,赤と青の縞模様となって図-\ref{fig:fig6}に現れている.
\begin{figure}
\begin{center}
	\includegraphics[clip,scale=0.5]{Figs/snapshot.eps}
\caption{
	観測領域${\cal R}$における振動速度分布のスナップショット.
}
\label{fig:fig5}
\end{center}
	\vspace{-5mm}
\end{figure}
\begin{figure}
\begin{center}
	\includegraphics[clip,scale=0.5]{Figs/phase_xy.eps}
	\caption{位相$\Phi(x,y,\omega)$の空間分布.周波数0.7および1.0MHzに対する結果.}
	\label{fig:fig6}
\end{center}
	\vspace{-10mm}
\end{figure}
\subsection{波数-周波数スペクトル}
伝播距離と時間から伝播速度を求める場合,波形の立ち上がり位置を読み取る必要がある.
しかしながら,とりわけ不均質材では,波形の立ち上がり位置を合理的かつ客観的に決定することは困難なことが多い.そこで,波数-周波数スペクトルのピークから位相速度を読み取り,後述する本研究での提案手法よる結果と比較を行う.

波数-周波数スペクトルは,$x$および$y$方向のフーリエ変換により
\begin{equation}
	\hat{\hat {A}}(\xi_x,\xi_y,\omega) =
	\iint A(x,y,\omega)e^{-i(\xi_x x +\xi_y y)}dxdy
	\label{eqn:Fkk_spctr}
\end{equation}
で与えられる.ここでは,主たる波動伝播方向が$x$軸の方向であるため,
\begin{equation}
	\hat{\hat {A}}(\xi_x,0,\omega) =
	\iint A(x,y,\omega)e^{-i\xi_x x}dxdy
	\label{eqn:Fk_spctr}
\end{equation}
を見ることで$x$方向への伝播速度について検討する.以下では,$\xi_x=2\pi k_x$と置き,式(\ref{eqn:Fk_spctr})の波数−周波数スペクトルをあらためて
\begin{equation}
	\bar{A}(k_x,\omega)=\hat{\hat {A}}(\xi_x,0,\omega)
	\label{eqn:def_Ak}
\end{equation}
と書く.図-\ref{fig:fig7}と図-\ref{fig:fig8}に,観測データ$\cal D$から計算した
波数−周波数スペクトル$\bar{A}$を示す.
図-\ref{fig:fig7}はアルミブロックの,図\ref{fig:fig8}は花崗岩供試体に対する結果を
示す.各々,横軸が周波数$f(=\omega/2\pi)$を,縦軸が波数$k_x$を表し,
スペクトル振幅$\left| \bar{A}\right|$を最大値で無次元化してカラ−表示している.
アルミニウム供試体に対する結果では,波数と周波数の間にはっきりとした直線的な関係があり,
特に1.0から2.5MHz程度の周波数成分が観測波形に含まれていることが分かる.
一方,花崗岩供試体では,大きな振幅を持つ成分は0.7から1.3MHz程度の狭い帯域に集中し,
波数と周波数の直線関係も,アルミニウム供試体程明確ではない.
ここで,$x$方向の位相速度を$c$とすれば,位相速度は波数と角周波数を用いて
\begin{equation}
	c=\frac{\omega}{\xi_x}
	\label{eqn:def_c}
\end{equation}
と表される.そこで,波数$\xi_x$において$|\bar A|$のピークを与える角周波数
$\omega_{peak}(k)$を特定し,これを式(\ref{eqn:def_c})へ代入して
位相速度を求める.その結果を波数$\xi_x$について平均することで,最終的な位相速度とすれば,
\begin{eqnarray}
	c&=&2.964[{\rm km/s}](アルミニウム) \label{eqn:c_al}\\
	c&=&2.894[{\rm km/s}](花崗岩ブロック) \label{eqn:c_blk}
\end{eqnarray}
となる.ただし,平均を取る波数の範囲は,アルミニウム供試体では,$0.1\leq \xi_x \leq 1.0$[mm]$^{-1}$, 花崗岩供試体では$0.1\leq \xi_x \leq 0.5$[mm]$^{-1}$とした.
これらの位相速度は,観測領域領域全体における平均的な伝播速度を表すと考えられる.
\begin{figure}
\begin{center}
	\includegraphics[clip,scale=0.5]{Figs/fkplot_Al.eps}
	\caption{波数−周波数スペクトル(アルミニウム供試体).}
	\label{fig:fig7}
\end{center}
	\vspace{-5mm}
\end{figure}
\begin{figure}
\begin{center}
	\includegraphics[clip,scale=0.5]{Figs/fkplot_bar2.eps}
	\caption{波数−周波数スペクトル(花崗岩供試体).}
	\label{fig:fig8}
\end{center}
	\vspace{-5mm}
\end{figure}

