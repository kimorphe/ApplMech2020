%%%%%%%%%%%%%%%%%%%%%%%%%%%%%%%%%%%%%%%%%%%%%%%%%%%%%%%%%%%%%%%%%%%%%%%
実験で得られた波形データの全体を,
\begin{equation}
	{\cal D}=\left\{
		a(x,y,t)\left| (x,y,t)\in {\cal R}\times [0,T_d] \right.
	\right\}
	\label{eqn:dataset}
\end{equation}
と表す.ここに,$T_d$は計測時間範囲を意味する.
図-\ref{fig:fig4}は,データセット$\cal D$から同一の時刻$t$における振幅を取り出して作成した
鉛直振動場のスナップショットを,$t=21\mu$秒,$t=23\mu$秒について示したものである.
上段は花崗岩供試体,下段は均質材であるアルミニウムブロックを用いて計測した結果を示し,
いずれもオシロスコープで計測された振幅値[mV]でをカラ−表示したものである.
均質なアルミニウム供試体の場合,若干のゆらぎはあるものの,概ね波形を保ったまま
右($x>0$)方向に超音波が伝播し,鉛直方向に伸びる直線的な波面がはっきりと観察されている.
一方,強い不均質性を持つ花崗岩供試体では,アルミニウムと同程度の速度で
波動場が右方向へ進展していることは分かるものの,振幅のゆらぎが非常に大きく,
初動の到達位置は明確でなく,大きな振幅を持つ波動の通過後も,振動が継続する様子が
見られる.ここで,$a(x,y,t)$の時間$t$に関するフーリエ変換を
\begin{equation}
	A(x,y,\omega)=\int a(x,y,t)e^{i\omega t} dt
	\label{eqn:Fourier_t}
\end{equation}
とし,フーリエ変換$A(x,y,\omega)$の位相を
\begin{equation}
	\Phi(x,y,\omega)={\rm Arg}\left\{ A(x,y,\omega) \right\}
	\label{eqn:def_Phi}
\end{equation}
と表す.ここに,$\omega$は角周波数を表し,位相$\Phi$は$A$と
\begin{equation}
	A=\left| A \right|e^{i\Phi}
	\label{eqn:phi2A}
\end{equation}
の関係にあり,$\Phi$は
\begin{equation}
	-\pi < \Phi \leq \pi
	\label{eqn:dom_phi}
\end{equation}
となるように取る.図\ref{fig:fig5}は,$\cal D$からFFTによって求めた位相$\Phi(x,y,\omega)$
の受信領域$\cal R$における空間分布を,周波数0.7MHzと1.0MHzについて示したものである.
同一の位相となる点を結ぶ曲線は波面を表し,アルミニウム供試体の場合,上下にほぼ直線的
に伸びる波面群が現れていることが分かる.0.7MHzの場合,$x=0$付近で若干波面が屈曲
している用に見える。これは,送信周波数帯域の下限に近い周波数のため,
ノイズの影響が1.0MHzの場合よりも強く現れるためと考えられる.
これに対して花崗岩供試体では,いずれの周波数においても,場所によらず波面は著しく屈曲
している.また,波面は$y$方向へ伸びる傾向は観察でき,配向性を示すことは明らかな
ものの,特定の個々の波面がどのような曲線を描いているかを視認することは困難である。
なお,図-\ref{fig:fig4}に示されるように,超音波は$x>0$の方向に進行している.
そのため,観測領域全体でみたとき,位相は$x>0$方向に増加傾向を示す。
ただし,ここでは位相の範囲を式(\ref{eqn:dom_phi})のようにしていることに注意が必要である。
その結果,位相は$-\pi$から$\pi$の間で増加し,$\pi$を超えたところで負の側に折り返される.
これにより,アルミニウム供試体に対する結果では$x$方向に鋸刃状の変動を繰返すパターンが
赤と青の縞模様となって示されている.
\begin{figure}
\begin{center}
\includegraphics[clip,scale=0.5]{Figs/snapshot.eps}
\caption{
	振動速度分布のスナップショット.
}
\label{fig:fig4}
\end{center}
\end{figure}
\begin{figure}
\begin{center}
	\includegraphics[clip,scale=0.5]{Figs/phase_xy.eps}
	\caption{位相$\Phi(x,y,\omega)$の空間分布.周波数0.7および1.0MHz
	の結果.}
	\label{fig:fig5}
\end{center}
\end{figure}
\begin{equation}
	\fat{k}= (k_x,k_y)= \frac{1}{2\pi} \nabla \phi (x,y,\omega)
	\label{eqn:}
\end{equation}
\begin{equation}
	{\rm Prob} [k](\omega), \ \ 
	{\rm Prob} [\theta](\omega)
	\label{eqn:}
\end{equation}
\subsection{分散関係}
波動場の分散性の有無を調べるために波数-周波数スペクトルを求める.
波数-周波数スペクトルは,$x$および$y$方向のフーリエ変換により
\begin{equation}
	\hat{\hat {A}}(\xi_x,\xi_y,\omega) =
	\iint A(x,y,\omega)e^{-i(\xi_x x +\xi_y y)}dxdy
	\label{eqn:Fkk_spctr}
\end{equation}
で与えられる.ここでは、主たる波動の伝播方向が$x$方向であるため,
\begin{equation}
	\hat{\hat {A}}(\xi_x,0,\omega) =
	\iint A(x,y,\omega)e^{-i\xi_x x}dxdy
	\label{eqn:Fk_spctr}
\end{equation}
を見ることで$x$方向に関する分散関係について検討する。
以下では,$\xi_x=2\pi k_x$と置き,式(\ref{eqn:Fk_spctr})の波数−周波数スペクトルを
\begin{equation}
	\bar{A}(k_x,\omega)=\hat{\hat {A}}(\xi_x,0,\omega)
	\label{eqn:def_Ak}
\end{equation}
と書く.
図-\ref{fig:fig6}と図-\ref{fig:fig7}に,データ$\cal D$からFFTによって計算した
波数−周波数スペクトル$\bar{A}$を示す.
図-\ref{fig:fig6}はアルミブロックの,図\ref{fig:fig7}は花崗岩供試体に対する結果を
表す.
各々,(a)は横軸を周波数$f(=\omega/2\pi)$,縦軸を波数$k_x$として
$\left| \bar{A}\right|$をカラ−表示したものを表し,(b)はカーブフィッティングによって
推定した位相速度と群速度の周波数依存性を表している.アルミニウム供試体に
対する結果では,波数と周波数の間にははっきりとした直線的な関係があり,
特に1.0から2.5MHz程度の周波数成分を持つ波が観測波形に含まれていることが分かる。
一方,花崗岩供試体では,大きな振幅を持つ成分は0.7から1.2MHz程度と狭い帯域に集中し,
波数と周波数の関係は概ね直線的と推定されるが,アルミニウム供試体の結果程明確な傾向は
読み取れない.$x$方向の位相速度を$c$,群速度を$c_g$とすれば,
\begin{equation}
	c=\frac{\omega}{k_x}
	\label{eqn:def_c}
\end{equation}
\begin{equation}
	c_g=\frac{d\omega}{dk_x}
	\label{eqn:def_cg}
\end{equation}
で与えられる.そこで,各周波数において$|\bar A|$のピークを
与える波数を特定して$c$を求めると,図-\ref{fig:fig7}および図\ref{fig:fig8}の
(b)に示したような結果が得られる.さらに,波数スペクトルのピークを与える
$(\omega,k_x)$のデータを1次式,2次式で最小二乗法によりフィッテイングして,
式(\ref{eqn:def_cg})より群速度を求めると,(b)のグラフに青と赤の直線で
示したような結果が得られる.
(a)の図に示した実線と破線はこれらの近似曲線を示したものである。
アルミニウム供試体の場合,回帰式の次数による差はほとんど無く,
波数と周波数は直線関係にあると結論できる.
一方,花崗岩供試体の場合,フィッティング結果から得られる波数と周波数の関係は
波数−周波数平面上では大差ないものの、その勾配を計算して群速度として見たときには,
最大$\pm$0.3km/sと無視できない差がある.つまり、この場合$k_x-\omega$の関係は1次式,
2次式いずれとも言い難く,平均的な位相速度の見積もりには有用なものの,
$k_x$と$\omega$の関係を一意に定めることには無理があると言える.
言い換えれば,花崗岩を透過する波の挙動を調べるにあたり,簡単な周波数分散の関係で表現出来ない
ことを前提として議論をする必要があることを示唆している.
\begin{figure}
\begin{center}
	\includegraphics[clip,scale=0.32]{Figs/fkplot_Al.eps}
	\caption{(a)波数−周波数スペクトル.(b)伝播速度と周波数の関係. (アルミニウム供試体)}
	\label{fig:fig6}
\end{center}
\end{figure}
\begin{figure}
\begin{center}
	\includegraphics[clip,scale=0.32]{Figs/fkplot_bar2.eps}
	\caption{(a)波数−周波数スペクトル.(b)伝播速度と周波数の関係. (花崗岩供試体)}
	\label{fig:fig7}
\end{center}
\end{figure}
% メモ
%Fig.4に周波数0.4,0.6,0.9および1.2MHzの波形成分に対する位相の空間分布を示す.
%相対的に低周波の0.4と0.6MHzでは,等位相線(波面)がy軸方向に伸びる1次元的な構造を示している.
%ただし,波面は屈曲して直線的ではない.一方,0.9および1.2MHzでは,y方向への位相の揺らぎが大きく,
%平面波的な構造が見られない.Fig.5に,位相分布の勾配を中央差分で近似して求めた,
%波数ベクトルkの確率密度分布を示す.Fig.5-(a)は,波数ベクトルの大きさkに関する,
%(b)はx軸方向から測ったkの方向に関する確率密度を示している.(a)の図にあるように,k
%の確率密度は非対称かつ有限な幅をもち,ガウス分布的でもない.また,周波数が大きく
%なるにつれ分散が増加している.これは,波数と周波数の関係が確定的に定められないこと,
%高周波になる程波動場の分散性が強まるが,波数の大きさは一定範囲に留まることを示している.
%また,波数ベクトルは入射方向に配向するが,相対的に高周波の1.0MHzと1.2MHzでは配向性が低下し,
%伝搬経路の屈曲が強まることを示している.これら波数ベクトルの確率密度と周波数の関係を記述する
%法則を見出すことは今後の課題だが,花崗岩のランダム媒体としてのモデル化においては,
%このような波数ベクトルの特徴を反映する必要がある.
