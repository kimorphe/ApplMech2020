本節では,参照モデルとしての役割を果たすモデルタイプ1に対するシミュレーション
結果をはじめに,次に,不規則な充填構造をもつモデルタイプ2,3および4に対する
結果を,モデルタイプ1と比較する形で示す.
それらの結果を踏まえ,固相粒子形状や充填構造,粒径分布が
マクロ拡散係数や異常拡散の程度に与える影響を調べる.
%
\subsection{モデルタイプ1に対する結果}
図-\ref{fig:fig5}に,ランダムウォーク・シミュレーションで得られた
平均2乗変位の時間変化を示す.このグラフは,横軸を無次元化時間$\tau$とし,
縦軸を$W$で無次元化した拡散粒子の平均2乗変位としたもので,
$S_r$=0.2から1.0まで,異なる9段階の飽和度に対する結果を示している.
先に述べたように,各飽和度での計算は20通りの異なる水分配置に対して行っており,
図-\ref{fig:fig5}はそれら全ての結果を合わせて算出した平均2乗変位を表している.
%
%なお,平均2乗変位の算出に先立ち,拡散粒子変位の$XY$平面内における分布を調べたところ,
%目立った配向性はなく等方的であった.そこで,$X$方向変位$U_i$,
%$Y$方向変位$V_i$を区別せずに平均した,
%\begin{equation}
%	\left< U^2 \right>=\frac{
%			\left<U_i^2 \right>
%		+
%		\left<V_i^2 \right>
%	}{2}
%	=\sum_{i=1}^{N_p}
%	\frac{U_i^2+V_i^2}{2N_p}
%	\label{eqn:Ubar}
%\end{equation}
%を平均2乗変位として図\ref{fig:fig5}に示した.
このグラフから明らかなように,いずれの飽和度でも$\left<U^2\right>$は時間に対して増加するが,
飽和度が$S_r=0.2$から0.5程度の場合,$\left<U^2\right>$の値は途中で頭打ちとなる.
これは,互いに分離した液相領域が多数存在するために,有限な液相領域のサイズを超えて変位が
増加できないためである.一方,$S_r>0.5$のときには,$\left<U^2\right>$は$\tau$に対して
単調に増加を続ける.このことは,飽和度$S_r$の増加に伴い孤立していた液相領域が連結して,
ユニットセルをパーコレートする拡散パスが次第に形成されることを表している.\\
\hspace{\parindent}
図-\ref{fig:fig5}の結果に式(\ref{eqn:Kb})を最小2乗法でフィッティングし,
拡散係数$\bar{K}$とべき指数$\alpha$を求めた結果を,それぞれ図-\ref{fig:fig6}と
図-\ref{fig:fig7}に示す.$\bar{K}$は$S_r<0.4$ではほぼゼロで,
概ね$S_r=0.4$をしきい値として増加をはじめ,最終的に間隙が完全に飽和した$S_r=1.0$では
マクロ拡散係数の値が0.4程度となっている.
つまり,飽和状態でマクロ拡散係数は,液相の拡散係数$D_0$の4割程度となり,
間隙率が約21.5\%であることを考慮すれば,ユニットセルに占める拡散相の比率よりも
大きな値であると言える.このことは,多孔質体のマクロ拡散係数を評価する際,
拡散相と非拡散相(ここでは気相と固相)の割合だけでなく,
間隙形状の効果を考慮することが必要であることを意味している.
一方,図-\ref{fig:fig6}に示したべき指数$\alpha$の挙動を見ると,
$\alpha$は$S_r=0.3$から顕著に増加し,$S_r=1.0$でのみ$\alpha=1$となっている.
すなわち,飽和度に応じた程度の差はあるものの,不飽和状態では常に遅い異常拡散が起こること,
通常拡散となるのは飽和状態の場合に限られることが示されている.
\begin{figure}[t]
\begin{center}
%\includegraphics[clip,scale=0.50]{Figs/fig4.eps}
\caption{
	平均2乗変位$\left<U^2\right>$の時間変化.
	モデルタイプ1,飽和度$S_r=0.2\sim 1.0$に対する結果.
}
\label{fig:fig5}
\end{center}
\end{figure}
%---------------------------------------------------
\begin{figure}
\begin{center}
%\includegraphics[clip,scale=0.50]{Figs/fig5.eps}
\caption{
	マクロ拡散係数$\bar K$と飽和度の関係(モデルタイプ1対する結果).
	}
\label{fig:fig6}
\end{center}
\end{figure}
%---------------------------------------------------
\begin{figure}
\begin{center}
%\includegraphics[clip,scale=0.5]{Figs/fig6.eps}
\caption{
	べき指数$\alpha$と飽和度の関係(モデルタイプ1に対する結果).
	}
\label{fig:fig7}
\end{center}
\end{figure}
\vspace{-5mm}
%%%%%%%%%%%%%%%%%%%%%%%%%%%
%
\begin{figure}[t]
\begin{center}
%\includegraphics[clip,scale=0.50]{Figs/fig11.eps}
\caption{
	平均2乗変位$\left<U^2\right>$の時間変化.
	モデルタイプ2,飽和度$S_r=0.2\sim 1.0$に対する結果.
}
\label{fig:fig11}
\end{center}
\end{figure}
\begin{figure}[t]
\begin{center}
%\includegraphics[clip,scale=0.50]{Figs/fig12.eps}
\caption{
	平均2乗変位$\left<U^2\right>$の時間変化.
	モデルタイプ3,飽和度$S_r=0.2\sim 1.0$に対する結果.
}
\label{fig:fig12}
\end{center}
\end{figure}
\begin{figure}[t]
\begin{center}
%\includegraphics[clip,scale=0.50]{Figs/fig13.eps}
\caption{
	平均2乗変位$\left<U^2\right>$の時間変化.
	モデルタイプ4,飽和度$S_r=0.2\sim 1.0$に対する結果.
}
\label{fig:fig13}
\end{center}
\end{figure}
\begin{figure}[h]
\begin{center}
%\includegraphics[clip,scale=0.50]{Figs/fig7.eps}
\caption{
	マクロ拡散係数$\bar K$と飽和度の関係(モデルタイプ1$\sim$4に対する結果).
	}
\label{fig:fig8}
\end{center}
\end{figure}
%
\begin{figure}[h]
\begin{center}
%\includegraphics[clip,scale=0.5]{Figs/fig8.eps}
\caption{
	べき指数$\alpha$と飽和度の関係(モデルタイプ1$\sim$4に対する結果).
	}
\label{fig:fig9}
\end{center}
\end{figure}
\subsection{モデルタイプによる拡散挙動の違い}
図-\ref{fig:fig11}$\sim$図-\ref{fig:fig13}に,タイプ2$\sim$4のモデルに対する
平均2乗変位の時間変化を示す.それぞれのグラフは,図-\ref{fig:fig5}と同様,
横軸が無次元化時間$\tau$,縦軸は平均2乗変位$\left<U^2\right>$としたもので,
計算を行った全ての飽和度$S_r$に対する結果が示されている.
これらタイプ2$\sim$4に対する結果は互いによく似た形状の曲線群となっている.
一方,タイプ1との比較でみると,時間に対して直線的に変化するケースがタイプ2$\sim$4
では見当たらず,程度の差はあれ,いずれも異常拡散となっていることが示唆されている.\\
\hspace{\parindent}
平均2乗変位の時間推移から決定した,タイプ1$\sim$4のモデルに対する
マクロ拡散係数$\bar{K}$とべき指数$\alpha$を,それぞれ,図-\ref{fig:fig8}と
図-\ref{fig:fig9}に示す.
図-\ref{fig:fig8}にあるように,拡散係数$\bar{K}$はモデルタイプによらず
飽和度$S_r$に対して類似した関数形で単調増加する.
ただし,不規則な充填構造を持つモデルタイプ2と3および4は,モデルタイプ1に比べて
より拡散係数の値が小さい.これは,モデルタイプ1では固相粒子が同一直線上
に並んでいるため,間隙に十分な水分がある場合,拡散粒子が直線的な経路でユニットセル
を横断あるいは縦断できるためと考えられる.
これに対し,不規則な充填構造を持つモデルでは,拡散粒子物が不規則に並んだ
固相粒子を迂回しながら移動し,拡散経路が常に屈曲することからマクロ拡散係
数値は小さくなり,べき指数も1に達することなく常に遅い拡散となっている.
% --- 追加
ただし,タイプ1のべき指数$\alpha$は,$Sr<0.4$の低飽和度の側では
他のモデルと比べて若干小さな値となっている.
これは,タイプ2$\sim$4では大小様々な粒径の固相粒子を迂回して拡散粒子が
移動するのに対し,タイプ1で水分が少ない場合は,拡散粒子が常に同一粒径
の固相粒子表面近傍を這うように迂回しながら移動する必要があることに起因する.
%
次に,モデルタイプ3と4の結果を比較すると,$\bar{K},\alpha$とも大きな
違いはなく,平均粒径や間隙率,飽和度が同程度であれば,粒子形状がマクロな
拡散に与える影響は小さいことが分かる.
一方,モデルタイプ2では,べき指数$\alpha$の挙動はその他不規則充填構造モデル
(タイプ3,4)と大差ないものの,
マクロ拡散係数は$S_r>0.8$程度の飽和度において他と比べ明らかに小さい.
これは以下の理由によると考えられる.
%
粒子配置がランダムな場合,水分量が多いときでも間隙のネットワークが屈曲し,
拡散経路が長くなる. さらに,均一粒径の固相粒子を充填した場合,
広い粒径分布を持つ粒子を充填したときに比べ,互いに接触した粒子が長いネットワークを作り易い.
つまり,拡散のボトルネックとなるような狭隘部を迂回する経路が見つかりにくくなる.
タイプ2のモデルでは,これら2つの効果が相まって,拡散係数が他のモデルよりも
小さくなると考えられる.
%
%均一粒径の粒子を容器内に密に充填することが,広い粒径分布をもつ粒子を
%同じ容器に充填することに比べてより困難なことは,日常経験からも明らかである.
%このことは,互いに接触して密に配列した粒子のネットワークが,容器を横断あるいは縦断するように発達し易いことを意味する.
%従って,均一粒径の粒子が不規則に充填されているとき,拡散物質は
%固相粒子が密に並んだ箇所を簡単には迂回できず,変位の増加が抑制される.
%この結果,マクロ拡散係数の値がモデル2では小さくなった原因と考えられる.
%%単一粒径の粒子を不規則な配置で充填した場合,粒子の一部は平均より密に,
%%残る部分は疎に充填されざるを得ない.固相粒子の疎に配置された箇所では,
%%拡散粒子は移動しやすく,密に固相粒子が配置された領域では移動が抑制される.
%%従って,固相粒子が密に配置された領域を迂回する拡散経路が無い限り,
%%局所的な拡散係数の増加は平均2乗変位の増加,すなわちマクロ拡散係数の
%%増加に貢献しない.
%%以上の挙動を整理すると,モデルタイプ1では,拡散経路が液相領域に制限される
%%ことでマクロ拡散係数は,液相の拡散係の1/2程度まで低減される.
%%モデルタイプ2から4では,
%%拡散経路の屈曲による効果が加わることでより拡散係数が低下する.
%%さらに,単分散粒子系を不規則に充填したタイプ2では,局所的な間隙率の変動に起因した
%%拡散物質の移動抑制効果が現れる.さらに,べき指数がモデルタイプ2から4では1
%%に達しないことから判断して,拡散経路の屈曲が遅い異常拡散の原因となることがわかる.
\subsection{ランダムウォーカー変位の確率密度分布}
最後に,ランダムウォーカー変位の確率密度分布について調べた結果を
図-\ref{fig:fig10}に示す.これは,時刻$\tau=0.25$における
%変位成分$U_i$と$V_i$を,ヒストグラムを正規化して示したものであり,
水平変位$U_i$と鉛直変位$V_i$を併せてヒストグラム化したもので,
縦軸は正規化されており確率密度とみなすことができる.
これら4つのプロットのうち(a)と(b)は,モデルタイプ1に対する結果を
(c)と(d)はモデルタイプ4に対する結果を表す.
それぞれ飽和度は$S_r=$0.7と1.0で,$S_r=0.7$のときの拡散係数$\bar K$は,
$S_r=1.0$の場合に比べていずれのモデルも概ね1/4程度となっている.
なお,青の実線は,変位成分$\left\{U_i,V_i\right\}$と同じ分散をもつ正規分布を示している.
タイプ1に対する結果を表示するにあたり,ヒストグラムの階級幅を,モデルタイプ4のプロット
に比べて意図的に粗く設定している.タイプ1のモデルでは,固相粒子が規則的に配置されているために,
階級幅を固相粒子間距離である$0.1W$よりも小さくすると,確率密度分布に
周期$0.1W$の凹凸が現れ,全体の分布形状がわかりにくくなる.
これを避けるために,図-\ref{fig:fig10}の(a)と(b)では階級幅を
0.67$W$とし,(c)と(d)ではその半分の値としている.\\
\hspace{\parindent}
ここで,図-\ref{fig:fig10}-(b)をみると,
モデルタイプ1では,飽和度$S_r=1.0$のとき確率密度分布がほぼ完全な正規分布となって
いることが分かる.実際,この場合のべき指数は$\alpha=1$であり,
通常の拡散方程式に従う結果であることが確認される.
一方,モデルタイプ4では,(d)のプロットにあるように$S_r=1.0$で概ね正規分布に近い形
となるものの,(b)との比較でみると若干正規分布よりも裾の広い形になっていることに気づく.
このケースでは,べき指数も$\alpha=0.7$と1より小さく,やや遅いタイプの拡散で,
その影響が確率密度分布にも現れていることが分かる.
次に,飽和度が$0.7$の場合について見れば,正規分布からのずれがより
明確となっている.特に(c)のケースでは,確率密度分布が中央で尖った形となっており,
裾野も正規分布に比べて明らかに広い.
このとき$\alpha$は0.5程度の遅い異常拡散となっており,
変位の確率密度も正規分布で表現できないことは明らかである.
以上の結果は,不飽和多孔質媒体のマクロ拡散問題を解析する際,
モデルタイプ1で$S_r=1.0$のような特別な状況を除き,
通常の拡散方程式を用いることことはできないことを意味する.
従って,アップスケーリングされた拡散問題を考える際には,
例えば非整数階微分の拡散方程式を用いることや,
図\ref{fig:fig10}に示されるような変位の確率密度に従うランダムウォーク
を設計する等の対応が必要と考えられる.
%間隙スケールの情報を取り込んだ上で,マクロな拡散解析をランダムウォークで行うことができる.
%その場合,平均2乗変位の時間変化に対する関数形を仮定することなくアップスケーリングを行うことができ,
%同様な手順を繰り返すことで,多段階のアップスケーリングにつなげることが可能となる.
%いずれのアプローチが有効であるかは今後検討すべき課題と.
%
\begin{figure}[h]
\begin{center}
%\includegraphics[clip,scale=0.35]{Figs/fig9.eps}
\caption{
	ランダムウォーカー変位の確率密度分布.青の実線は,分散を一致させたときの正規分布を表す.
	}
\label{fig:fig10}
\end{center}
\end{figure}

