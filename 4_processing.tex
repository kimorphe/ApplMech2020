本節では,局所的な波数を定義し,その統計的な挙動から花崗岩中の
波動伝播を特徴付け,伝播速度や到達時間のゆらぎを定量的に評価する
ことを試みる.
\subsection{局所波数ベクトル}
フーリエ変換によって定義される波数$(\xi_x,\xi_y)$は,
観測領域全体の変動を反映した大域的な量である.
花崗岩中を伝播する波動は,各所で伝播方向や波長が変化するため,
その空間変動を評価するには,局所的な波数を用いることがよりふさわしい.
そこで,位相の勾配で与えらる局所的な波数ベクトル$\fat{k}=(k_x,k_y)$の
分布を調べることにする.
ただし,式(\ref{eqn:def_Phi})で与えられる位相$\Phi$は$\pm \pi$
を挟んで跳躍がありそのままでは微分出来ない.
そこで,微分可能な形にアンラップされた位相
\begin{equation}
	\phi=\arg A
	\label{eqn:}
\end{equation}
を考え,この勾配として波数ベクトルを定義する。
\begin{equation}
	2\pi \fat{k}= 
	2\pi (k_x,k_y)=
	\left(
		\frac{\partial \phi}{\partial x}
		, \, 
		\frac{\partial \phi}{\partial y} 
	\right)
	\label{eqn:}
\end{equation}
\subsection{数値計算方法}
\begin{equation}
	x_i=x_0+i \Delta x,\, y_j=y_0+j\Delta y
	\label{eqn:}
\end{equation}
\begin{equation}
	f_{i,j}=f(x_i,y_j)
	\label{eqn:}
\end{equation}
$i=0,1,\dots N_x-1$, $j=0,\dots N_y-1$に対して.
ここでは
\begin{equation}
	\Delta x=\Delta y=0.5{\rm mm}
	\label{eqn:}
\end{equation}
\begin{equation}
	N_x=41, N_y=61
\end{equation}
\begin{equation}
	(x_0,y_0)=(0,-15){\rm mm}
	\label{eqn:}
\end{equation}
\begin{equation}
	2\pi (k_x)_{i+\frac{1}{2},j}
	=
	\left( \frac{\partial \phi }{\partial x}\right) _{i+\frac{1}{2},j}
	\simeq 
	\frac{\phi_{i+1,j}-\phi_{i,j}}{\Delta x}
	\label{eqn:}
\end{equation}
\begin{equation}
	\phi_{i+1,j}-\phi_{i,j}
	=
	\arg \left(\frac{ A_{i+1,j}}{A_{i,j}}\right)
	=
	{\rm Arg} \left(\frac{ A_{i+1,j}}{A_{i,j}} \right)
	\label{eqn:}
\end{equation}
\begin{equation}
	\left( k_x \right)_{i+\frac{1}{2},j} 
	\simeq 
	\frac{1}{2\pi \Delta x}
	{\rm Arg} \left( \frac{ A_{i+1,j}}{A_{i,j}} \right)
	\label{eqn:}
\end{equation}
\begin{equation}
	\left( k_y \right)_{i,j+\frac{1}{2}} 
	\simeq 
	\frac{1}{2\pi \Delta y}
	{\rm Arg} \left( \frac{ A_{i,j+1}}{A_{i,j}} \right)
	\label{eqn:}
\end{equation}
\begin{equation}
	\phi(x,y)=2\pi \int_{\Gamma} \fat{k}\cdot d\fat{s}
	\label{eqn:}
\end{equation}
$\fat{x}_0=(0,y)$に対して,
\begin{equation}
	\phi(\fat{x})
	=\min_{\left(\Gamma, \fat{x}_0\right)}
	\int_\Gamma 2\pi \fat{k} \cdot d\fat{s}
	\label{eqn:}
\end{equation}
ただし
\begin{equation}
	\fat{k}\cdot d\fat{s}\geq 0 \, {\rm on} \, \Gamma 
	\label{eqn:}
\end{equation}
\begin{equation}
	\fat{k}\cdot d\fat{s}
	=
	\left\{
	\begin{array}{c}
		\pm k_x \Delta x \\
		\pm k_y \Delta y
	\end{array}
	\right.
	\label{eqn:}
\end{equation}

\begin{figure}
\begin{center}
	\includegraphics[clip,scale=0.3]{Figs/pdf_kw_Al.eps}
	\caption{波数ベクトルの(a)大きさと(b)方向の分布(アルミニウム供試体).}
	\label{fig:fig8}
\end{center}
\end{figure}
\begin{figure}
\begin{center}
	\includegraphics[clip,scale=0.3]{Figs/pdf_kw_bar.eps}
	\caption{波数ベクトルの(a)大きさと(b)方向の分布(花崗岩供試体).}
	\label{fig:fig9}
\end{center}
\end{figure}
\begin{figure}
\begin{center}
	\includegraphics[clip,scale=0.8]{Figs/unwrap.eps}
	\caption{波数ベクトルの積分経路$\Gamma$とその折れ線近似$\tilde \Gamma$のイメージ.}
	\label{fig:fig12}
\end{center}
\end{figure}
\begin{figure}
\begin{center}
	\includegraphics[clip,scale=0.4]{Figs/phi_R.eps}
	\caption{アンラップした位相から計算した到達時間の空間分布. (a)アルミニウム供試体,(b)花崗岩供試体.いずれも周波数1.0MHzの結果.}
	\label{fig:fig10}
\end{center}
\end{figure}
\begin{figure}
\begin{center}
	\includegraphics[clip,scale=0.3]{Figs/tof_hist.eps}
	\caption{到達時間の頻度分布. (a)アルミニウム供試体,(b)花崗岩供試体.}
	\label{fig:fig11}
\end{center}
\end{figure}
\begin{figure}
\begin{center}
	\includegraphics[clip,scale=0.5]{Figs/sigma_t.eps}
	\caption{到達時間の標準偏差と伝播距離の関係. (a)アルミニウム供試体,(b)花崗岩供試体.}
	\label{fig:fig11}
\end{center}
\end{figure}
\begin{figure}
\begin{center}
	\includegraphics[clip,scale=0.5]{Figs/sigma_y.eps}
	\caption{到達距離の標準偏差と伝播時間の関係. (a)アルミニウム供試体,(b)花崗岩供試体.}
	\label{fig:fig11}
\end{center}
\end{figure}
