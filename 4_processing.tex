本節では,局所的な量としての波数ベクトルをあらためて定義し,その統計分布に
基づいて花崗岩中の波動伝播挙動を調べることで,伝播速度,到達時間とそのゆらぎを
定量的に評価する.
\subsection{局所波数ベクトル}
フーリエ変換によって定義される波数$(\xi_x,\xi_y)$は,観測領域全体の情報を反映した
大域的な量と言える.花崗岩中を伝播する波動は,各所で伝播方向や波長が変化するため,
そのような空間変動を評価するには局所的な量として定義した波数を用いることがより適切である.
そこで,位相の勾配で与えらる局所的な波数ベクトル$\fat{k}=(k_x,k_y)$を新たに定義して,
そ統計的な分布を調べる.ただし,式(\ref{eqn:def_Phi})で与えられる位相$\Phi$は$\pm \pi$
を挟んで跳躍があり,そのままでは微分出来ない点が現れる.
そこで,着目点$\fat{x}=(x,y)$において微分可能となるように,アンラップされた位相$\phi$を
考え,これを式(\ref{eqn:def_Phi})と区別して
\begin{equation}
	\phi(\fat{x},\omega)=\arg\left\{  A(\fat{x},\omega) \right\}
	\label{eqn:def_phi}
\end{equation}
と表す.$\phi$と$\Phi$の間には
\begin{equation}
	\phi(x,\omega)=\Phi(\fat{x},\omega) + 2\pi n(\fat{x},\omega )
	\label{eqn:unwrap}
\end{equation}
関係があり,$n$は$\fat{x}$において位相$\Phi$の跳躍を打ち消すように
選ばれた整数を表す.位相$\phi$を用いれば,
局所量としての波数ベクトル$\fat{k}=(k_x,k_y)$を
\begin{equation}
	2\pi \fat{k}= 
	2\pi (k_x,k_y)=
	\left(
		\frac{\partial \phi}{\partial x}
		, \, 
		\frac{\partial \phi}{\partial y} 
	\right)
	\label{eqn:}
\end{equation}
によって得ることができる.このようにして与えた波数ベクトルは,
位置$\fat{x}=(x,y)$と角周波数$\omega$の関数となる.
\subsection{波数ベクトルの計算方法}
波数ベクトルを観測データ$\cal D$から数値的に評価するには,
以下のようにして差分近似を行う.

$\cal R$上の関数$f(x,y)$の,格子点$(x_i,\, y_j)$における値を
\begin{equation}
	f_{i,j}=f(x_i,y_j)
	\label{eqn:}
\end{equation}
と表す.この表記を用い,$\phi$の$x$方向への偏微分を次のように中央差分で近似する.
\begin{equation}
	2\pi (k_x)_{i+\frac{1}{2},j}
	=
	\left( \frac{\partial \phi }{\partial x}\right) _{i+\frac{1}{2},j}
	\simeq 
	\frac{\phi_{i+1,j}-\phi_{i,j}}{\Delta x}
	\label{eqn:fd_x}
\end{equation}
ここで,式(\ref{eqn:fd_x})最右辺の分母の項は,
\begin{equation}
	\phi_{i+1,j}-\phi_{i,j}
	=
	\arg \left(\frac{ A_{i+1,j}}{A_{i,j}}\right)
\end{equation}
と書くことができる.格子間隔$\Delta x$が十分に小さければ,この項は
\begin{equation}
	\left| \phi_{i+1,j}-\phi_{i,j}\right| < \pi
	\label{eqn:phi_bound}
\end{equation}
としてよく,その場合
\begin{equation}
	\arg \left(\frac{ A_{i+1,j}}{A_{i,j}}\right)
	=
	{\rm Arg}\left(\frac{ A_{i+1,j}}{A_{i,j}}\right)
\end{equation}
とできる.以上より,
\begin{equation}
	\left( k_x \right)_{i+\frac{1}{2},j} 
	\simeq 
	\frac{1}{2\pi \Delta x}
	{\rm Arg} \left( \frac{ A_{i+1,j}}{A_{i,j}} \right)
	\label{eqn:kx_FD}
\end{equation}
となり,直接式(\ref{eqn:unwrap})の$n$を$(\fat{x},\omega)$毎に求めることなく
波数$k_x$を得ることができる.
$y$方向への微分についても同様に考えれば,$k_y$は
\begin{equation}
	\left( k_y \right)_{i,j+\frac{1}{2}} 
	\simeq 
	\frac{1}{2\pi \Delta y}
	{\rm Arg} \left( \frac{ A_{i,j+1}}{A_{i,j}} \right)
	\label{eqn:ky_FD}
\end{equation}
となることが分かる.$(x_i,y_j)$における波数ベクトル$\fat{k}_{i,j}$が
必要な場合には,
\begin{eqnarray}
	(k_x)_{i,j} &=&
	\frac{1}{2}\left\{ (k_x)_{i+\frac{1}{2},j}+ (k_x)_{i-\frac{1}{2},j} \right\}
	\label{eqn:} \\
	(k_y)_{i,j} &=&
	\frac{1}{2}\left\{ (k_y)_{i,j+\frac{1}{2}}+ (k_y)_{i,j-\frac{1}{2}} \right\}
	\label{eqn:}
\end{eqnarray}
で代用する.
\subsection{波数ベクトルの確率分布}
\subsection{頻度分布}
波数ベクトル$\fat{k}$のて統計的な特徴を見るために,$\fat{k}$の頻度分布を求める.
そこで,格子$\cal G$上で得られた観測量を$g(\fat{x})$とし,その全体
$\left\{ g(\fat{x}), \fat{x}\in {\cal G}) \right\}$について,階級幅$\Delta g$で
カウントしたときの頻度分布を$P_{\cal G} [g;\Delta g]$と書くことにする.
またこれを,"${\cal G}$で観測したデータ$g$の,階級幅$\Delta g$で評価した頻度分布"と読むことにする.
なお,頻度分布は次のように正規化されているものとする.
\begin{equation}
	\sum_{g} P_{\cal G}[g;\Delta g]=1
	\label{eqn:normalize}
\end{equation}
なお,頻度分布と確率密度関数は異なるものであるが,本研究では頻度分布の階級幅を
一定とするため,分布形状や平均値等の値は,確率密度関数を使って全ての議論を行った
場合と変わらない.
\subsection{波数ベクトルの分布特性}
波数ベクトルの大きさと方向を
\begin{equation}
	k=\left| \fat{k} \right|=\sqrt{k_x^2+k_y^2},  \ \ 
	\alpha = \tan^{-1} \left( \frac{k_y}{k_x} \right)
	\label{eqn:kpol}
\end{equation}
と表す.これら正規化した頻度分布:
\[
	P_{\cal G}[k;\Delta k], \ \ 
	P_{\cal G}[\alpha;\Delta \alpha]
\]
を,図\ref{fig:fig8}と図\ref{fig:fig9}に示す.
これらの図は,それぞれアルミニウム供試体,花崗岩供試体に対する結果で,
(a)は$k$の,(b)は$\alpha$の頻度分布を周波数との関係で示している.
階級幅は$\Delta k=0.024$mm$^{-1}$, $\Delta \alpha=$7.2度で,
図中,白の実線は平均値を,破線は平均値$\pm$標準偏差を示したものである.
アルミニウム供試体の結果を見ると,波数と周波数は0.2MHzから2.5MHz程度の
帯域で明確な比例関係にあることがわかる.また,波数ベクトルの方向は,
同じ周波数帯域において0度付近に集中し,$x$軸方向へ強く配向していることが分かる.
ただし,2MHz程度から$\alpha$の標準偏差が次第に大きくなり,2.5MHzを超えるあたりで
配向性が消失している.以上より,アルミニウム供試体では2.5MHz程度までの進行波成分
が観測されているが,2MHz以上ではノイズの影響が大きくなり,2.5MHz以上ではほとんど
有意な信号成分が含まれなていないことがわかる.
一方,図\ref{fig:fig9}に示した花崗岩供試体の結果では,
波数と周波数の比例関係は認められるが,その周波数帯域は0.6MHzから2.0MHz程度と
アルミニウム供試体に比べて狭い.また,標準偏差も$k,\alpha$ともアルミニウムの
場合に比べて大きく,0.5MHzから高周波側に向けて増加する.
特に波数ベクトルの方向$\alpha$に関して言えば,いずれの周波数でも配向の程度が
アルミニウム供試体と比べて低く,2.5Mz近傍では標準偏差が約90度に達し,配向性は
完全に消失する.
以上より,信号成分の含まれる周波数帯域$\Omega$は,アルミニウム供試体ではおよそ
0.5〜2.5MHz, 花崗岩供試体では0.6から2.0MHzと判定できる.
このような情報は,フーリエ変換による波数-周波数スペクトルからは得ることができず,
局所波数ベクトルの確率分布を見ることの一つの利点と言える.
ただし,ここで言う帯域とは, 配向性を消失することなく,x方向伝播する波動が
観測できる周波数範囲の目安であり,一定値以上の波動振幅の得られる帯域という通常の
定義とは異なることに注意が必要である.
%花崗岩のような不均質媒体では,計測波形やその周波数スペクトルは複雑な形を
%しており,振幅を基準にした議論は難しいことが多い.その点で,波数ベクトルの
%統計分布に基づく帯域の定義に従い,見るべき周波数範囲を定めることが
%より適切と考えられる.
\begin{figure}
\begin{center}
	\includegraphics[clip,scale=0.5]{Figs/pdf_kw_Al.eps}
	\caption{(a) 波数ベクトルの大きさと,(b)方向の分布(アルミニウム供試体).}
	\label{fig:fig8}
\end{center}
\end{figure}
\begin{figure}
\begin{center}
	\includegraphics[clip,scale=0.5]{Figs/pdf_kw_bar.eps}
	\caption{(a)波数ベクトルの大きさと,(b)方向の分布(花崗岩供試体).}
	\label{fig:fig9}
\end{center}
\end{figure}
%%%%%%%%%%%%%%%%%%%%%%%%%%
\subsection{到達時間関数}
%観測領域内の位置$\fat{x}$における波動は,$x=0$の側から伝播してくる.
観測領域${\cal R}$の左端$(x=0)$から伝播した超音波が,領域内の任意の点$\fat{x}\in {\cal R}$に
到達するために要する時間を$T_f(\fat{x},\omega)$とし,$T_f$を波数ベクトルから求めることを考える.
そこで,直線伝播距離が$a$となる$\cal R$内の点を,
\begin{equation}
	{\cal I}(a)={\cal R} \cup \left\{ x=a\right\}
	\label{eqn:def_I0}
\end{equation}
とし,ある$\fat{x}_0\in {\cal I}(0)$からの波動が,経路$\Gamma$を伝播して
$\fat{x}$に到達した場合について考える(図\ref{fig:fig12}).
このとき,始点から終点の間で生じる位相の変化は
\begin{equation}
	\phi(\fat{x},\omega)-\phi(\fat{x}_0,\omega)=
	2\pi \int_{\Gamma} \fat{k}(\fat{s},\omega)\cdot d\fat{s}
	\label{eqn:int_phi}
\end{equation}
で与えられる.ここで,$d\fat{s}$は経路$\Gamma$の微小接ベクトルで, 
波動伝播は位相が増加する方向に起きることから,$\Gamma$上では
\begin{equation}
	\fat{k}\cdot d\fat{s}\geq 0 \, {\rm on} \, \Gamma 
	\label{eqn:}
\end{equation}
であることが要請される.フェルマーの原理によれば,$\fat{x}_0$と$\fat{x}$を
結ぶ経路$\Gamma$のなかで,実際に選ばれる経路は,位相変化を最小にするものである.
ただし,$\fat{x}$に到達する 波動伝播の起点$\fat{x}_0$が, $\cal I$(0)上のどこかにあるは
前もってわからない.そこで,フェルマーの原理を適用するにあたり,$\Gamma$と$\fat{x}_0$が
式(\ref{eqn:int_phi})を最小にするように選ばれると考える.
すなわち,$\fat{x}$における位相は,
\begin{equation}
	\phi(\fat{x},\omega) - \phi(\fat{x}_0,\omega)
	=
	\min_{\left(\Gamma, \fat{x}_0\right)}
	\int_\Gamma 2\pi \fat{k}(\fat{s},\omega) \cdot d\fat{s}
	\label{eqn:Fermat}
\end{equation}
となると考える.これを周波数$\omega$で割れば,到達時間が次のように得られる.
\begin{equation}
	T_f(\fat{x},\omega) = \
	\frac{\phi(\fat{x},\omega)}{\omega}
	\label{eqn:def_Tf}
\end{equation}
実際には,観測波形は${\cal G}$でのみ与えられているため,経路$\Gamma$を
格子点を結ぶ折れ線$\tilde \Gamma$で近似する(図\ref{fig:fig12}).
\begin{equation}
	\fat{k}\cdot d\fat{s}
	=
	\left\{
	\begin{array}{c}
		\pm k_x \Delta x \\
		\pm k_y \Delta y
	\end{array}
	\right.
	\label{eqn:}
\end{equation}
さらに,各々の$\fat{x}$について経路$\Gamma$を直接特定するのでなく,
次のようにして,出発点から最小の位相変化で到達できる点を順次決定して
位相分布を決定する.
\subsection{到達時間の計算方法}
以下$T_f$の計算手順を述べる.\\
はじめに,$\fat{x}\in {\cal G}\cap {\cal I}(0)$を一つ選ぶ.
この点の位相を0とし,位相決定済み点としてリストに登録する.
次に,$\fat{x}$に隣接する格子点をリストアップし,これを近傍点と呼ぶ.
全ての近傍点には位相決定済みの隣接点が存在する.
そこで, 式(\ref{eqn:phi_int})を離散化した
\begin{equation}
	\phi(\fat{x}+\Delta \fat{s}) -\phi(\fat{x}_0) = 2\pi \fat{k} \cdot \Delta \fat{s}
	\label{eqn:inc_phase}
\end{equation}
を用い,近傍点の位相を計算して仮登録する.
ここで式(\ref{eqn:inc_phase})において,$\fat{x}$は位相決定済みの点に,
$\fat{x}+\Delta \fat{s}$はその近傍点に取り,
$\Delta \fat{s}$は2つの点の隣接関係(相対位置)に応じて次のいずれかで与える.
\begin{equation}
	\Delta \fat{s} = (\pm \Delta x,\,0) \, or \, (0,\pm \Delta y)
	\label{eqn:} \end{equation}
なお,波数ベクトル$\fat{k}$は,式(\ref{eqn:kx_FD})あるいは式(\ref{eqn:ky_FD})で計算した
ものを用いる.また, $\fat{k}\cdot \Delta \fat{s}<0$となる方向にある近傍点の位相は未定の
ままとし,複数の位相決定済み点に隣接した近傍点には,それら隣接点から計算された位相の
中で最小のものを仮登録する.
この作業を全ての近傍点で行った後,仮登録された位相の中で最小のものだけを採用し,
その格子点を位相決定済みのリストに加える.
位相決定済み格子点リストが更新された後は,近傍点のリストを再度作成し,位相の仮登録から
の手順を繰り返す.以上の作業を,新たに登録される近傍点がなくなるまで実行する.
これら一連の位相決定作業を,${\cal G}\cap {\cal I}(0)$に含まれる全ての点を開始点として
行えば,最終的には全ての点について,$\cal I$(0)から最小の位相変化で到達したときの位相が
登録されることになる.
%式(\ref{eqn:Fermat})で表される位相$\phi(\fat{x},\omega)$を求めることができる.
ただし,この方法では$\cal G$の全ての点で位相が決定できる保証は無く,与えられた波数ベクトル
場によっては,位相が未決定の点が残される.しかしながら,これは位相決定方法の不備ではなく,
波数ベクトル場の性質によるものと言える.観測データにはノイズが含まれ波数ベクトルが
完全に正確には求まらないこと,また,波動伝播経路が観測面内に限定されず,本来は
3次元的であることを考えると,いくつかの点で位相が未決定のまま残ることは不自然ことではない.
\begin{figure}
\begin{center}
	\includegraphics[clip,scale=0.7]{Figs/unwrap.eps}
	\caption{波数ベクトルの積分経路$\Gamma$とその折れ線近似$\tilde \Gamma$のイメージ.}
	\label{fig:fig12}
\end{center}
\end{figure}
\subsection{到達時間関数の計算例}
以上の方法で計算した到達時間関数$T(\fat{x},\omega)$の一例として,
周波数1MHzの場合に対する結果を図-\ref{fig:fig10}に示す. 
アルミニウム供試体に対する結果では,到達時間の分布は$y$方向にほぼ一定となり,
期待通り1次元的な波動伝播挙動が再現されている.
なお,到達時間が未定のまま残された格子点では,便宜上$T_f=-1$として表示している.
そのため到達時間未定の箇所は,図\ref{fig:fig10}において孤立した紺色のセルとして
示されている.アルミニウム供試体に対する結果において,領域中央部のドットが欠けたように
見える点は,このことによる.アルミニウム供試体でこのような点が生じた理由は,
供試体表面の性状が悪く,LDVでの波形計測が十分な強度の反射光を得ることが出来なかった
ためと考えられる.
図-\ref{fig:fig10}(b)の花崗岩供試体に対する結果でも,$x$>0の方向へ到達時間が遅れる
傾向ははっきりしている.一方で,$y$軸方向への分布を見たとき,到達時間は一様でなく
揺らぎがあることが分かる.また,到達時間未定の箇所もアルミニウム供試体の場合よりも多い.
これは主として,強い散乱により波数ベクトルの方向が3次元的に変動し,
供試体表面上の経路からは超音波が到達できない点が生じているための考えられる.
\begin{figure}
\begin{center}
	\includegraphics[clip,scale=0.4]{Figs/phi_R.eps}
	\caption{到達時間$T_f(\fat{x},\omega)$の空間分布. 
	(a)アルミニウム供試体,(b)花崗岩供試体.いずれも周波数1.0MHzの結果.}
	\label{fig:fig10}
\end{center}
\end{figure}
\subsection{到達時間の確率分布}
このようにして得られた到達時間の見積りを頻度分布として整理した結果を
図-\ref{fig:fig11}に示す.この図は,周波数帯域$\Omega$を
\[
	\Omega = \left[ 0.6, 2.0\right],\,\, [{\rm MHz}]
\]
として,頻度分布:
\begin{equation}
	P_{{\cal G \cap I}(x)\times \Omega }(T_f; \Delta T), \ \ \Delta T=0.1[\mu {\rm s}]
	\label{eqn:P_GIO}
\end{equation}
を示したものである.縦軸は伝播距離$x$を,横軸は到達時間$T_f$を表し,
$(T_f,x)$における頻度分布をカラーマップとして表示している.
$T_f-x$平面上の各点における頻度を評価する際のサンプルには,
位置$x$にある格子点${\cal G \cap I}(x)$において計算した,
帯域$\Omega$内の全ての周波数に対して計算した到達時間を用いている.
図\ref{fig:11}-(a)に示したアルミニウム供試体の場合,
到達時間の分布は右上がりの直線近傍の狭い範囲に集中し,
$T_f$の分散は伝播距離$x$によってほとんど変化しない.
この直線の傾きは伝播速度$c$を表すので,$T_f$の平均$\bar{T}_f(x)$を
直線近似して傾きを求めると,
\begin{equation}
	c=2.975[{\rm km/s}], \ \ (アルミニウム)
	\label{eqn:c_Al_ave}
\end{equation}
となる.この結果は,波数−周波数スペクトルのピークから求めた速度と
0.012[km/s]の差しかなく両者はよく一致する.このことは,
本研究で提案した到達時間関数が合理的なもので,その数値的な評価も
問題なく行われていることを意味する.
一方,\ref{fig:fig11}-(b))に示す花崗岩供試体に対する結果では,
伝播距離$x$の増加にともない,到達時間$T_f$の分散が大きくなっている.
このことを強調するため,図\ref{fig:fig11}-(b)には,
$T_f$の平均$\bar{T}_f$を白の実線で,標準偏差を$\delta T_f$として
$\bar{T}_f\pm \delta T_f$にあたる位置を白の破線でそれぞれ示している.
なお,アルミニウム供試体の場合は,これらの曲線は近接して図上では
互いに区別がつかないため,図\ref{fig:fig11}-(a)に平均と標準偏差を示す
カーブは示していない.
\begin{figure}
\begin{center}
	\includegraphics[clip,scale=0.5]{Figs/tof_hist.eps}
	\caption{到達時間$T_f$の正規化した頻度分布. (a)アルミニウム供試体,(b)花崗岩供試体.}
	\label{fig:fig11}
\end{center}
\end{figure}
図-\ref{fig:fig12}に,平均到達時間と$\bar{T}_f$と位置$x$の関係を示す.
このグラフには,アルミニウムと花崗岩供試体の結果を示している.
これまでは,花崗岩供試体として図\ref{fig:fig1}に示したブロック状の
ものに対する結果だけを示してきた.以下ではこれに加えて,同じ石切り場
で採取した万成花崗岩を円柱状に加工した供試体に対して得られた結果を併せて示す.
2つの花崗岩供試体を区別する場合,前者をブロック供試体,後者を円柱供試体等と呼ぶ.
なお,円柱供試体の直径は60mm,高さは50mmで,超音波計測は
コア上部のフラットな面で行った.送受信点の配置等,その他の計測条件は
ブロック供試体の場合と同様である.
図-\ref{fig:fig12}のグラフによれば,平均到達時間$\bar T_f$の大小関係は概ね
\begin{center}
	ブロック供試体 $<$ 円柱供試体 $<$ アルミニウム
\end{center}
となっているが,円柱供試体では$x$が大きなところで次第に$\bar T_f$が遅れ,
$x=17.5\mu$s付近ではアルミニウム供試体と同程度になっている.
これらの関係を直線近似し,花崗岩供試体における伝播速度$c$を求めると,
\begin{eqnarray}
	c=3.272[{\rm km/s}], &&(花崗岩ブロック)\\
	c=2.983[{\rm km/s}], &&(花崗岩円柱)
	\label{eqn:c_granite_ave}
\end{eqnarray}
となり,同じ花崗岩でも音速に差があることが示される.
花崗岩にはマイクロクラックの配向に起因した音響異方性があることはよく知られている.
ここで求めた,2つの花崗岩供試体に対する音速値の差は,既往の文献に報告されている
異方性によるS波速度の伝播方向による変動と同程度で,顕著に大きな差では無い.
なお,花崗岩ブロックの場合,波数-周波数スペクトルのピークと,平均到達時間から求めた
音速に乖離がある.しかしながら,波数-周波数スペクトルによる方法が,大きな振幅を
持つ波形成分の伝播挙動に影響を受けやすいのに対し,平均到達時間は位相だけに
着目して速度を算出しているた,振幅のばらつきには影響されにくいと考えられる.
従って,不均質材の場合2つの方法で求めた音速が一致すべき必要性は無い.
さらに,到達時間は平均値のまわりにばらつきがあるため,距離と到達時間に
一対一の関係を定めることは厳密には出来ず,走時曲線と速度の定義には任意性がある.
このような視点からすると,速度よりも,到達時間の分布自体がより
本質的なものであると言うことができる.
\begin{figure}
\begin{center}
	\includegraphics[clip,scale=0.5]{Figs/tof1d.eps}
	\caption{到達時間の平均$\bar T_f$と伝播距離$x$の関係. }
	\label{fig:fig12}
\end{center}
\end{figure}
\subsection{到達時間の不確実性}
%最後に,到達時間と伝播距離の不確実性について議論する.
図-\ref{fig:fig13}に,到達時間の標準偏差$\delta T_f$と距離$x$の関係を示す.
図-\ref{fig:fig13}(a)は横軸を位置$x$にとり,到達時間の標準偏差$\delta T_f$をプロットした
もので,3つの供試体に対する結果を比較している.花崗岩供試体の場合,伝播距離$x$が大きくなるに
伴い,到達時間の標準偏差も単調に増加している.一方,アルミニウム供試体では,
伝播時間に応じた$\delta T_f$の増減はほとんどなく,およそ0.1$\mu$s以下の範囲ににとどまっている.
2つの花崗岩供試体で比較すると,$x=15$mm辺りから若干の乖離が生じて,円柱供試体の場合に$\delta T_f$
が大きくなっている.これは,平均到達時間$T_f$がブロック供試体と比較して遅れる,すなわち,
速度が低下する領域に相当する.
図-\ref{fig:fig13}(b)は,同じ結果を縦軸を平均到達時間$\bar T_f$で正規化した
標準偏差$\frac{\delta T_f}{\bar{T}_f}$としたグラフで,
全ての供試体正規化した標準偏差が距離に対して単調に減少する様子が示されている.
この結果は,到達時間の不確実性(標準偏差)が到達時間そのものと同程度あるいはそれ以上の
割合で増加するわけではないことを示している.また,花崗岩供試体間で比較すると,
到達時間差が考慮された結果,位置による変化挙動は互いによく似た曲線となっている.
これは,2つの供試体の間でランダム不均質性に共通する面があることを示していると
理解することができる.
\\
次に,到達時間$T_f$に対応する伝播距離$x$の標準偏差$\delta x$を計算した結果を
図-\ref{fig:fig14}に示す.ここでも,(a)は標準偏差$\delta x$を,
(b)は伝播距離の平均$\bar{x}$で正規化した標準偏差$\frac{\delta x}{\bar{x}}$を示す.
$\delta T_f$が,位置$x$で観測した到達時間の不確実性を表すのに対し,
$\delta x$は,時刻$T_f$で観測行ったときの伝播距離の不確実性を表している.
アルミニウム供試体の場合,時間によって位置の不確実性はあまり変化せず,
$\delta x$は0.25mm程度となっている.
アルミニウム供試体の位相速度$c$は約3.0[km/s]だから,$\delta x/c$はおよそ
$0.8\mu$sとなり,これは図\ref{fig:fig13}(a)にある結果と整合する.
花崗岩供試体に対する結果では,位置の不確実性である$\delta x$は時刻$T_f$に応じて増加し,
2つの供試体で比べると$\delta x$はブロック供試体の方が一貫して大きな値となっている.
これは,ブロック供試体の音速が円柱供試体の音速よりも大きいためであり,実際,平均位置で正規化すると
両者の差は目立たなくなくなる.
\\
最後に,正規化した標準偏差$\frac{\delta T_f}{\bar{T}_f}$を,両対数軸上にプロットする.
その結果は,図\ref{fig:fig15}のようであり,花崗岩供試体に対する結果には伝播距離$x$と
正規化した標準偏差がほぼ直線的な関係にあることが分かる.
そこで,$\frac{\delta T_f}{\bar{T}_f}$を,べき関数
\begin{equation}
	\frac{\delta T_f }{\bar{T}_f} \simeq \frac{K}{x^m}
	\label{eqn:fit_power}
\end{equation}
で近似し,両対数グラフでの傾き$-m$を最小2乗法によって求めると,
それぞれの供試体で次に示すような評価
が得られる.
\begin{eqnarray}
	m &=& 1.073 (アルミニウム供試体) \\
	m &=& 0.537 (花崗岩ブロック)\\
	m &=& 0.497 (花崗岩円柱)\\
\end{eqnarray}
よって,花崗岩供試体の場合,正規化した到達時間の標準偏差がおよそ$x^{-1/2}$に比例することが分かる.
さらに,平均到達時間$\bar T_f$と伝播距離$x$の間に比例関係
\begin{equation}
	x=\tilde c \bar T_f
	\label{linfit_x}
\end{equation}
が仮定できるならば,式(\ref{eqn:fit_power})は
\begin{equation}
	\delta T_f \simeq \frac{K}{\sqrt{\tilde c}}\sqrt{ \bar{T}_f}
	\label{eqn:plaw} 
\end{equation}
となり,到達時間の標準偏差が,$\sqrt{\bar{T}_f}$に比例するとの
結果が得られ,不確実性の指標が一つのパラメータで表現できる可能性があることが示される.
%このことは,$\bar{T}_f(x)$と$\delta T_f(x)$の関係をモデル化しておけば,
%$\bar{T}_f(x)$と$m$を実測データから決めることで,
%$(x,T_f)$の同時確率分布を定めることができ,その結果から音速や到達位置の
%平均$\bar{x}$や不確実性$\delta x$の特徴を理解できることによる.
このような法則に普遍性があるか否かは,今後実測値との比較で検証を行う必要があるが,
べき則による不確実性の発展則は,シンプルで理解しやすく,今後,ランダム不均質
媒体中の波動伝播モデルを構築する上で,有用な考え方になると思われる.
\begin{figure}
\begin{center}
	\includegraphics[clip,scale=0.5]{Figs/sigma_t.eps}
	\caption{到達時間$T_f$の標準偏差$\delta T_f$と伝播距離の関係. }
	\label{fig:fig13}
\end{center}
\end{figure}
\begin{figure}
\begin{center}
	\includegraphics[clip,scale=0.5]{Figs/sigma_y.eps}
	\caption{到達距離$\bar{x}$の標準偏差$\delta \bar{x}$と伝播時間$T_f$の関係. }
	\label{fig:fig14}
\end{center}
\end{figure}
\begin{figure}
\begin{center}
	\includegraphics[clip,scale=0.5]{Figs/tsig_log.eps}
	\caption{
		両対数軸上グラフとしてプロットした正規化された到達時間の標準偏差
			$\frac{\delta T_f}{\bar{T}_f}$. 
		}
	\label{fig:fig15}
\end{center}
\end{figure}
