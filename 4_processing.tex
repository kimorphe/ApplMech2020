本節では局所的な波数を定義し,その統計分布に基づいて花崗岩中の
波動伝播挙動を調べることで,伝播速度と到達時間のゆらぎを定量的に評価する.
\subsection{局所波数ベクトル}
フーリエ変換によって定義される波数$(\xi_x,\xi_y)$は, 観測領域全体の変動を反映した大域的な量である.
花崗岩中を伝播する波動は,各所で伝播方向や波長が変化するため,その空間変動を評価するには
局所的な波数を用いることがよりふさわしい.そこで,位相の勾配で与えらる局所的な波数ベクトル
$\fat{k}=(k_x,k_y)$を新たに定義して,その確率的な分布を調べる.
ただし,式(\ref{eqn:def_Phi})で与えられる位相$\Phi$は$\pm \pi$
を挟んで跳躍があり,そのままでは微分出来ない点が現れる.
そこで,着目点$\fat{x}=(x,y)$において微分可能となるように,アンラップされた位相$\phi$を
考え,これを
\begin{equation}
	\phi(\fat{x},\omega)=\arg\left\{  A(\fat{x},\omega) \right\}
	\label{eqn:def_phi}
\end{equation}
と表す.$\phi$と$\Phi$の関係は
\begin{equation}
	\phi(x,\omega)=\Phi(\fat{x},\omega) + 2\pi n(\fat{x},\omega )
	\label{eqn:}
\end{equation}
である.ただし,$n$は$\fat{x}$において位相$\Phi$の跳躍を打ち消すように
選ばれた整数である.
$\phi$を用いて,位置に依存した波数ベクトルを次のように定義する。
\begin{equation}
	2\pi \fat{k}= 
	2\pi (k_x,k_y)=
	\left(
		\frac{\partial \phi}{\partial x}
		, \, 
		\frac{\partial \phi}{\partial y} 
	\right)
	\label{eqn:}
\end{equation}
\subsection{波数ベクトルの計算方法}
波数ベクトルをデータ$\cal D$を用いて数値的に評価する際には,次のようにして差分近似を行う.
$\cal R$上の関数$f(x,y)$の,格子点$(x_i,\, y_j)$における値を
\begin{equation}
	f_{i,j}=f(x_i,y_j)
	\label{eqn:}
\end{equation}
と表す.この表記を用い,$\phi$の$x$方向の偏微分を次のような中央差分で近似する.
\begin{equation}
	2\pi (k_x)_{i+\frac{1}{2},j}
	=
	\left( \frac{\partial \phi }{\partial x}\right) _{i+\frac{1}{2},j}
	\simeq 
	\frac{\phi_{i+1,j}-\phi_{i,j}}{\Delta x}
	\label{eqn:fd_x}
\end{equation}
式(\ref{eqn:fd_x})の最右辺分母の項は,
\begin{equation}
	\phi_{i+1,j}-\phi_{i,j}
	=
	\arg \left(\frac{ A_{i+1,j}}{A_{i,j}}\right)
\end{equation}
と書ける.$\Delta x$が十分に小さければ,
\begin{equation}
	\left| \phi_{i+1,j}-\phi_{i,j}\right| < \pi
	\label{eqn:phi_bound}
\end{equation}
であると考えてく,その場合
\begin{equation}
	\arg \left(\frac{ A_{i+1,j}}{A_{i,j}}\right)
	=
	{\rm Arg}\left(\frac{ A_{i+1,j}}{A_{i,j}}\right)
\end{equation}
とできる.
以上より,
\begin{equation}
	\left( k_x \right)_{i+\frac{1}{2},j} 
	\simeq 
	\frac{1}{2\pi \Delta x}
	{\rm Arg} \left( \frac{ A_{i+1,j}}{A_{i,j}} \right)
	\label{eqn:kx_FD}
\end{equation}
となる.$y$方向への微分についても同様にして考えれば,$k_y$は
\begin{equation}
	\left( k_y \right)_{i,j+\frac{1}{2}} 
	\simeq 
	\frac{1}{2\pi \Delta y}
	{\rm Arg} \left( \frac{ A_{i,j+1}}{A_{i,j}} \right)
	\label{eqn:ky_FD}
\end{equation}
で与えられることが分かる.$(x_i,y_j)$における波数ベクトル$\fat{k}_{i,j}$が
必要となる場合には,
\begin{eqnarray}
	(k_x)_{i,j} &=&
	\frac{1}{2}\left\{ (k_x)_{i+\frac{1}{2},j}+ (k_x)_{i-\frac{1}{2},j} \right\}
	\label{eqn:} \\
	(k_y)_{i,j} &=&
	\frac{1}{2}\left\{ (k_y)_{i,j+\frac{1}{2}}+ (k_y)_{i,j-\frac{1}{2}} \right\}
	\label{eqn:}
\end{eqnarray}
で代用する.
\subsection{波数ベクトルの確率分布}
花崗岩供試体の物性値にはランダムな不均質性があり、花崗岩供試体で計測された波形にも
ランダムな変動が現れる.ランダム変動について確率的な特徴を見るために,
ここでは波数ベクトルの頻度分布を求め,均質材での結果と比較する.
いま,$\cal G$上で得られた観測量$g(\fat{x})$とし,その全体
$\left\{ g(\fat{x}), \fat{x}\in {\cal G}) \right\}$
について,適当な階級幅$\Delta g$でカウントした頻度分布を
$P_{\cal G} [g;\Delta g]$と書き、
"${\cal G}$で観測されたデータから得た$g$の階級幅$\Delta g$で評価した頻度分布"と読むことにする。
なお,頻度分布は正規化されており,特に断りの無い限り
\begin{equation}
	\sum_{g} P_{\cal G}[g;\Delta g]=1
	\label{eqn:normalize}
\end{equation}
であるとする.

波数ベクトルの大きさと方向を
\begin{equation}
	k=\left| \fat{k} \right|=\sqrt{k_x^2+k_y^2},  \ \ 
	\alpha = \tan^{-1} \left( \frac{k_y}{k_x} \right)
	\label{eqn:kpol}
\end{equation}
と表す。これらの正規化した頻度分布:
\[
	P_{\cal G}[k;\Delta k], \ \ 
	P_{\cal G}[\alpha;\Delta \alpha]
\]
を図\ref{fig:fig8}と図\ref{fig:fig9}に示す。
これらの図は、アルミニウム供試体、花崗岩供試体に対するものである。
(a)は$k$の、(b)は$\alpha$の頻度分布を周波数との関係で示ししている。
階級幅は$\Delta k=0.024$mm$^{-1}$, $\Delta \alpha=$7.2度である。
図中の白の実線は各周波数における平均値を,破線は平均値$\pm$標準偏差
を示したものである。
アルミニウム供試体の結果を見ると,波数と周波数は0.2MHzから2.5MHz程度の
帯域では明確な比例関係にあることがわかる。また、波数ベクトルの方向
は、同じ周波数帯域において0度付近に集中しており,$x$軸方向に非常によく
配向している様子が示されている
ただし、2MHzを超える辺りからは0度に集中した分布はしているものの、
$\alpha$の標準偏差はが大きくなっている。
これは,2MHz程度以上の周波数からは、信号/雑音比が低下しはじめている
ことを示している。以上のことを踏まえると、アルミニウム供試体では、
2.5MHz程度までは波数と周波数が比例することから、進行波成分が観測されているが、
2MHz以上はノイズの影響が次第に大きくなり,2.5MHz以上ではほとんど有意な信号成分が
偉得ていないことがわかる。
図\ref{fig:fig9}に示した花崗岩供試体の結果でも、波数と周波数の比例関係は
見られるが、その範囲は0.6MHzから2.0MHz程度とアルミニウム供試体に比べて狭い。
また、標準偏差は波数ベクトルの大きさ、向きともアルミニウムの場合に比べて
大きく、高周波側に向けてほぼ一貫して増加する。波数ベクトルの方向について
言えば、いずれの周波数でも配向の程度は均質材に比べて明らかに低く、
2.5Mz近傍で標準偏差が約90度に達して飽和し,配向性が消失している。
以上より、信号成分の含まれる周波数帯域は,大きくみてアルミニウム供試体では
0.2〜2.5MHz, 花崗岩供試体で0.6から2.0MHzであると言える。
このような情報は、波数-周波数スペクトルからは得ることができず、
局所波数ベクトルの統計を見ることの一つの利点と言える。なお、ここで言う帯域とは、
配向性を消失することなく、x方向へ伝播する波動が観測できる周波数範囲という
意味であり、一定値以上の波動振幅の得られる帯域という意味ではない。
花崗岩のような不均質媒体では、計測波形やその周波数スペクトルは複雑な形を
しているため、振幅を基準にした議論は難しいことが多い。その点で、波数ベクトルの
統計分布に基づく帯域の定義に従い、見るべき周波数範囲を定めることが
より適切と考えられる。
\subsection{到達時間関数}
観測領域内の位置$\fat{x}$における波動は,$x=0$の側から伝播してくる.
$x=0$から$\fat{x}\in {\cal R}$に波動が到達するために要する時間を
$T_f(\fat{x},\omega)$とする.$T_f$を波数ベクトルから求めることを考える.
\begin{equation}
	{\cal I}(0)={\cal R} \cup \left\{ x=0\right\}
	\label{eqn:def_I0}
\end{equation}
とし,ある$\fat{x}_0\in {\cal I}(0)$からの波動が,経路$\Gamma$を伝播して
$\fat{x}$に到達したとする(図\ref{fig:fig12}).
このとき,始点から終点までの間で生じる位相の変化は
\begin{equation}
	\phi(\fat{x},\omega)-\phi(\fat{x}_0,\omega)=
	2\pi \int_{\Gamma} \fat{k}(\fat{s},\omega)\cdot d\fat{s}
	\label{eqn:int_phi}
\end{equation}
で与えられる.ただし,$d\fat{s}$は$\Gamma$上の各点における微小接ベクトルで, 
波動伝播は位相が増加する方向に起きることから,伝播経路$\Gamma$上では,
\begin{equation}
	\fat{k}\cdot d\fat{s}\geq 0 \, {\rm on} \, \Gamma 
	\label{eqn:}
\end{equation}
であることが要請される.フェルマーの原理によれば,$\fat{x}_0$と$\fat{x}$を
結ぶ経路$\Gamma$のなかで,実際に選ばれる経路は,位相の変化を最小にするものである.
ただし,$\fat{x}$に到達する 波動伝播の起点$\fat{x}_0$が, $\cal I$(0)上のどこかにあるは
前もってわからない.そこで,フェルマーの原理を適用するにあたり,
$\fat{x}_0$と$\Gamma$は式(\ref{eqn:int_phi})を最小にするように選ばれると考える.
すなわち,$\fat{x}$における位相は,
\begin{equation}
	\phi(\fat{x},\omega) - \phi(\fat{x}_0,\omega)
	=
	\min_{\left(\Gamma, \fat{x}_0\right)}
	\int_\Gamma 2\pi \fat{k}(\fat{s},\omega) \cdot d\fat{s}
	\label{eqn:Fermat}
\end{equation}
となると考える.これを周波数$\omega$で割れば,到達時間が次のように得られる.
\begin{equation}
	T_f(\fat{x},\omega) = \frac{\phi(\fat{x},\omega)}{\omega}
	\label{eqn:def_Tf}
\end{equation}
実際には,観測波形は${\cal G}$でのみ与えられているため,$\Gamma$を
格子点を結ぶ折れ線$\tilde \Gamma$で近似する(図\ref{fig:fig12}).
\begin{equation}
	\fat{k}\cdot d\fat{s}
	=
	\left\{
	\begin{array}{c}
		\pm k_x \Delta x \\
		\pm k_y \Delta y
	\end{array}
	\right.
	\label{eqn:}
\end{equation}
さらに,各々の$\fat{x}$について経路$\Gamma$を特定するのでなく,次のようにして
$\cal G$での位相を決定する.

はじめに,$\fat{x}\in {\cal G}\cap {\cal I}(0)$を一つ選ぶ.
この点の位相を0とし,位相決定済み点としてリストに登録する.
次に,$\fat{x}$に隣接する格子点をリストアップし,これを近傍点と呼ぶ.
全ての近傍点には位相決定済みの隣接点が存在する.そこで, 式(\ref{eqn:phi_int})を離散化した
\begin{equation}
	\phi(\fat{x}+\Delta \fat{s}) -\phi(\fat{x}_0) = 2\pi \fat{k} \cdot \Delta \fat{s}
	\label{eqn:inc_phase}
\end{equation}
を用い,近傍点の位相を計算して仮登録する.このとき,$\fat{x}$は位相決定済みの点に,
$\fat{x}+\Delta \fat{s}$はその近傍点に取り,$\Delta \fat{s}$は両者の相対位置関係に
応じて次のいずれかで与える.
\begin{equation}
	\Delta \fat{s} = (\pm \Delta x,\,0) \, or \, (0,\pm \Delta y)
	\label{eqn:}
\end{equation}
なお,波数ベクトル$\fat{k}$は,式(\ref{eqn:kx_FD})あるいは式(\ref{eqn:ky_FD})で計算した
ものを用いる.また, $\fat{k}\cdot \Delta \fat{s}<0$となる方向にある近傍点の位相は未定の
ままとし,複数の位相決定済み点に隣接する近傍点には,それらの隣接点から計算された位相の
中で最小のものを仮登録する.
この作業を全ての近傍点で行った後,仮登録された位相の中で最小のものだけを採用して,
その格子点を位相決定済みのリストに加える.
更新された位相決定済み格子点に対し,近傍点のリストを作成して,位相の仮登録からの手順を繰り返す.
以上の作業を新たに登録される近傍点がなくなるまで実行する.
これら一連の位相決定作業を,${\cal G}\cap {\cal I}(0)$に含まれる全ての点を開始点として
順次行えば,式(\ref{eqn:Fermat})で表される位相$\phi(\fat{x},\omega)$を求めることができる.
なお,この方法では$\cal G$の全ての点で位相が決定できる保証は無く,与えられた波数ベクトル
場によっては,位相が未決定の格子点が残される.しかしながら、これは位相決定方法の不備ではなく,
波数ベクトル場の性質によるものと言え,観測データにはノイズが含まれ波数ベクトルが
完全に正確には求まらないこと,また,波動伝播経路が観測面内に限定されず,厳密には
3次元的であることを考えると,いくつかの点で位相が未決定のまま残ることは何ら不自然ことではない.
\begin{figure}
\begin{center}
	\includegraphics[clip,scale=0.5]{Figs/pdf_kw_Al.eps}
	\caption{波数ベクトルの(a)大きさと(b)方向の分布(アルミニウム供試体).}
	\label{fig:fig8}
\end{center}
\end{figure}
\begin{figure}
\begin{center}
	\includegraphics[clip,scale=0.5]{Figs/pdf_kw_bar.eps}
	\caption{波数ベクトルの(a)大きさと(b)方向の分布(花崗岩供試体).}
	\label{fig:fig9}
\end{center}
\end{figure}
\begin{figure}
\begin{center}
	\includegraphics[clip,scale=0.8]{Figs/unwrap.eps}
	\caption{波数ベクトルの積分経路$\Gamma$とその折れ線近似$\tilde \Gamma$のイメージ.}
	\label{fig:fig12}
\end{center}
\end{figure}
以上の方法で計算した到達時間関数$T(\fat{x},\omega)$を周波数1MHzの場合について図-\ref{fig:fig10}に
示す。アルミニウム供試体に対する結果では、到達時間はy方向にほぼ一定であり、期待通り、1次元的な
伝播挙動が再現されている。なお、到達時間が未定のまま残された格子点に関しては
到達時間を-1として表示している。そのため、到達時間未定の箇所は、図\ref{fig:fig10}において、孤立した
紺のセルとして示されている。アルミニウム供試体に対する結果の中央付近でドットが欠けたように
見えるのはこのためで、供試体表面の性状が悪く、この点でのみLDVで波形が十分な精度で計測できなかった
ことが理由として考えられる。
花崗岩供試体に対する結果では、
$x$>0の方向に次第に到達時間が遅れる傾向ははっきりしているものの、
到達時間は$y$軸方向に見たときに一様でなくゆらぎがある。また、到達時間未定の
箇所もアルミニウム供したの場合に比べて多い。これは、計測上の問題が含まれる
可能性もあるが、主として、強い散乱により波数ベクトルの方向が複雑かつ急激に変化する
ことによるものと考えられる。
\begin{figure}
\begin{center}
	\includegraphics[clip,scale=0.4]{Figs/phi_R.eps}
	\caption{到達時間$T_f(\fat{x},\omega)$の空間分布. 
	(a)アルミニウム供試体,(b)花崗岩供試体.いずれも周波数1.0MHzの結果.}
	\label{fig:fig10}
\end{center}
\end{figure}
このようにして得られた到達時間の見積りを頻度分布として整理した結果を
図-\ref{fig:fig11}に示す。この図は、周波数帯域$\Omega$を
\[
	\Omega = \left[ 0.6, 2.0\right],\,\, [{\rm MHz}]
\]
として、頻度分布:
\begin{equation}
	P_{{\cal G \cap I}(x)\times \Omega }(T_f; \Delta T), \ \ \Delta T=0.1[\mu {\rm s}]
	\label{eqn:P_GIO}
\end{equation}
を示したものである。縦軸は伝播距離$x$を,横軸は到達時間$T_f$を表し,
$(T_f,x)$における頻度分布をカラーマップとして表示している.
$T_f-x$平面上の各点における頻度を評価する際のサンプルには,
位置$x$にある格子点${\cal G \cap I}(x)$において
帯域$\Omega$内の周波数に対して計算した全ての到達時間を用いている。
位置に応じて到達時間が遅れる。
図\ref{fig:11}-(a)に示したアルミニウム供試体の場合、
到達時間の分布は右上がりの直線近傍の狭い範囲に集中し,
$T_f$の分散はほとんど変化しない。
この直線の傾きは伝播速度$c$を表すので、$T_f$の平均$\bar{T}_f(x)$を
直線近似して傾きを求めると,
\begin{equation}
	c=2.975[{\rm km/s}], \ \ (アルミニウム)
	\label{eqn:c_Al_ave}
\end{equation}
となる.この結果は,波数−周波数スペクトルのピークから求めた速度と
0.012[km/s]の差しかなく両者はよく一致する.このことは,
本研究で提案した到達時間関数が合理的なもので、その数値的な評価も
問題なく行われていることを意味する。
一方,\ref{fig:fig11}-(b))に示す花崗岩供試体に対する結果では、
位置$x$が遠ざかるにつれて、到達時間$T_f$の分散が大きくなる.
このことを強調するため,図\ref{fig:fig11}-(b)には,
$T_f$の平均$\bar{T}_f$を白の実線で,標準偏差を$\delta T_f$として
$\bar{T}_f\pm \delta T_f$にあたる位置を白の破線でそれぞれ示している.
なお,アルミニウム供試体の場合は,これらの曲線は近接して図上では
互いに区別がつかないため,図\ref{fig:fig11}-(a)には示していない。
\begin{figure}
\begin{center}
	\includegraphics[clip,scale=0.5]{Figs/tof_hist.eps}
	\caption{到達時間$T_f$の正規化した頻度分布. (a)アルミニウム供試体,(b)花崗岩供試体.}
	\label{fig:fig11}
\end{center}
\end{figure}
図\ref{fig:fig12}に,平均到達時間と$T_f$と位置$x$の関係を示す.
このグラフには、アルミニウムと花崗岩供試体の結果を示している。
これまでは、花崗岩供試体として図\ref{fig:fig1}に示したブロック状の
ものに対する結果だけを示してきた。以下ではこれに加えて,同じ石切り場
で採取した万成花崗岩を円柱コア状に加工した供試体に対して
得られた結果も併せて示す。2つの花崗岩供試体を区別する場合、
前者をブロック供試体,後者を円柱供試体等と呼ぶことにする。
なお、円柱供試体の直径は60mm,高さは50mmで,超音波計測は
コア上部のフラットな面で行った。送受信点の配置等,その他の計測条件は
ブロック供試体の場合と同様である。
到達時間は,概ね
\begin{center}
ブロック供試体 $<$ 円柱供試体 $<$ アルミニウム
\end{center}
となっているが,円柱供試体では$x$が大きなところで、次第に遅れ
$x=17.5\mu$s付近でアルミニウム供試体と同程度になっている。
これらの関係を直線近似し,花崗岩供試体における
伝播速度$c$を求めると,
\begin{eqnarray}
	c=3.272[{\rm km/s}], &&(花崗岩ブロック)\\
	c=2.983[{\rm km/s}], &&(花崗岩円柱)
	\label{eqn:c_granite_ave}
\end{eqnarray}
となり、同じ花崗岩でも音速に差があることが示される。
花崗岩はマイクロクラックの配向に起因した異方性があることはよく知られている。
2つの花崗岩供試体に対する音速値の差は、既往の文献に報告されている
異方性によるS波速度の伝播方向による差と同程度になっている。
\begin{figure}
\begin{center}
	\includegraphics[clip,scale=0.5]{Figs/tof1d.eps}
	\caption{到達時間の平均$\bar T_f$と伝播距離の関係. }
	\label{fig:fig12}
\end{center}
\end{figure}

図-\ref{fig:fig13}に、到達時間の標準偏差$\delta T_f$の観測位置に応じた変化を示す。
図-\ref{fig:fig13}(a)は、横軸を位置$x$にとり、標準偏差$\delta T_f$をプロットした
もので、3つの供試体に対する結果を比較している。花崗岩供試体の場合、$x$が大きくなるに
ともない、到達時間の標準偏差も単調に増加している。一方、アルミニウム供試体の場合、
観測位置に対して、$\delta T_f$はほとんど変化せず、0.1$\mu$s以下にとどまっている。
この差は、明らかに岩石供試体の不均質性に起因したものである。
2つの花崗岩供試体で比較すると、$x=15$mm辺りから若干の乖離が生じ、
円柱供試体の場合の$\delta T_f$が大きくなっている。これは、
平均到達時間$T_f$がブロック供試体と比較して遅れる、すなわち、速度が低下する領域に相当する。
図-\ref{fig:fig13}(b)は、同じ結果を縦軸を平均到達時間$\bar T_f$で正規化した
標準偏差$\frac{\delta T_f}{\bar{T}_f}$を示したものである。
いずれの供試体に対する結果も単調に減少し、到達時間の不確実性(標準偏差)が
本質的に増加し続けるわけではないことが分かる。また、
花崗岩供試体間では、到達時間の差が考慮された結果、位置による変化挙動の差がほとんどなくな
っている。これは、2つの供試体における不均質性の共通する面を反映していると考えらる点で重要
な結果であると言える。

最後に、到達時間$T_f$に対応する位置$x$の標準偏差$\delta x$を計算した結果を図-\ref{fig:fig14}に示す。
ここでも(a)は標準偏差$\delta x$を(b)は、平均$\bar{x}$で正規化した標準偏差$\frac{\delta x}{\bar{x}}$
を示している。
$\delta T_f$が、位置$x$で観測した到達時間の不確実性を表すのに対し、
$\delta x$は、時刻$T_f$で観測したときの波動の到達位置に関する不確実性を表している。
アルミニウム供試体の場合、時間によって位置の不確実性はあまり変化せず、
$\delta x$は0.25mm程度となっている。
アルミニウム供試体の位相速度$c$は約3.0[km/s]だから、$\delta x/c$はおよそ
$0.8\mu$sとなり、これは図\ref{fig:fig13}(a)にある結果と整合する。
花崗岩供試体に対する結果では、位置の不確実性である$\delta x$は時刻$T_f$に応じて
増加し、2つの供試体で比べると$\delta x$はブロック供試体の方が一貫して大きな値となっている。
これは、ブロック供試体の音速が円柱供試体の音速よりも大きいためであり、
実際、平均位置で正規化すると、両者の差は目立たなくなくなる。
以上、図-\ref{fig:fig13},図-\ref{fig:fig14}の結果から、

花崗岩供試体の不均質性を
見る上で最も本質的な指標となるのは、正規化した到達時間の標準偏差$\frac{\delta T/}{\bar T}$
であると考えられる。
これは、$\bar{T}_f(x)$と$\delta T_f(x)$の関係をモデル化しておけば、
$(x,T_f)$の同時確率分布を定めることができ、その結果から音速や
到達位置の平均$\bar{x}$や不確実性$\delta x$を推定できることによる。

位置に関する不確実性は、これに
\begin{figure}
\begin{center}
	\includegraphics[clip,scale=0.5]{Figs/sigma_t.eps}
	\caption{到達時間$T_f$の標準偏差$\delta T_f$と伝播距離の関係. (a)アルミニウム供試体,(b)花崗岩供試体.}
	\label{fig:fig13}
\end{center}
\end{figure}
\begin{figure}
\begin{center}
	\includegraphics[clip,scale=0.5]{Figs/sigma_y.eps}
	\caption{到達距離$\bar{x}$の標準偏差$\delta \bar{x}$と伝播時間$T_f$の関係. (a)アルミニウム供試体,(b)花崗岩供試体.}
	\label{fig:fig14}
\end{center}
\end{figure}
