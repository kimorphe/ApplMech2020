%%#!platex
%
% Example of Japanese Paper of JSCE
% for LaTeX2e users
%
% revised on 4/25/2014
%
%%%%%%%%%%%%%%%%%%%%%%%%%%%%%%%%%%%%%%%%%%%
%
% もし jis フォントメトリックを使う場合は,以下をアンコメントしてください.
% \DeclareFontShape{JY1}{mc}{m}{n}{<-> s * jis}{}
% \DeclareFontShape{JY1}{gt}{m}{n}{<-> s * jisg}{}
%
\documentclass{jsce}
%
\usepackage{epic,eepic,eepicsup}
%\usepackage{graphicx,multicol}
\usepackage{graphicx}
\usepackage{multicol}
\usepackage{amsmath}
%\usepackage{showkeys}
\usepackage{setspace}
%  amsを使う方は以下をアンコメントしてください.
%\usepackage{amssymb,amsmath}
% 英語はサポートしているかどうか不明
% \inenglish
% 学会サンプルに times とあるので指定しておきます
\usepackage{times}
%
\finalversion
\pagestyle{empty}
%
\title{
	超音波計測に基づく\\花崗岩中の表面波伝播特性に関する研究
}%
\endtitle{
A STUDY ON THE PROPAGATION CHARACTERISTICS OF SURFACE WAVES IN A GRANITE BASED ON ULTRASONIC MEASUREMENT
}
%
% emailアドレスのフォントをタイプライター体にしたい方は次行をアンコメント
% \emailstyle{\ttfamily}
% emailアドレスを公開される方は,
%% \thanks{○○○○○○\email{your_name@foo.ac.jp}}のようにしてください.
%
\author{木本 和志\thanks{正会員 博士(工学) 岡山大学 環境生命科学研究科(〒700-8530 岡山県岡山市北区3丁目1番地1号)\email{kimoto@cc.okayama-u.ac.jp}}・
岡野 蒼\thanks{学生会員 岡山大学環境生命科学研究科 (〒700-8530 岡山県岡山市北区3丁目1番地1号)}・
斎藤 隆泰\thanks{正会員 博士(工学)群馬大学大学院理工学府 環境創生部門(〒376-8515 群馬県桐生市天神町 1-5-1)}・
佐藤 忠信\thanks{正会員 博士(工学)神戸学院大・現代社会学部 (〒000-0000 *******************************)}・
松井 裕哉\thanks{正会員 XX(xx)日本原子力研究開発機構・幌延深地層研究センター(〒098-3224 北海道天塩郡幌延町北進432番地2)}
}
\endauthor{Kazushi KIMOTO, Aoi OKANO, Takahiro SAITOH, Hiroya MATSUI and Tadanobu SATO}
%
\abstract{
\small
本研究は,ランダムな非均質媒体における表面波の伝播挙動を,超音波計測と波形解析によって調べたものである.
実験では,粗粒結晶質岩である花崗岩をランダム非均質媒体として用い, 線集束型の圧電探触子で励起した表面波
による振動をレーザードップラー振動計で計測する.計測波形の解析は周波数領域において行い,
フェルマーの原理に基づき,各波形観測点における到達時間を周波数毎に求める.
この方法で得られた到達時間のアンサンブルから,到達時間が従う確率分布を伝播距離の関数として評価する.
このようにして求めた確率分布の標準偏差を到達時間の不確実性の指標として用い,伝播距離に応じた到達時間の
ゆらぎの進展挙動を調べる.以上の波形解析の結果,本研究に用いた花崗岩試料における到達時間の不確実性は,
概ね伝播距離の-1/2乗と平均到達時間の積に比例することを明らかとする.
このことは,非均質材に波動伝播の統計的モデルリングにおいて有用な知見となる.
}
%
\keywords{random media, heterogeneity, ultrasonic wave,uncertainty, travel time}
%
\endabstract{% Yes blank line
\normalsize
This study investigates the propagation characteristics of surface wave traveling in a random heterogenious medium. 
For this purpose, ultrasonic measurements are perfomed on a coarese-grained granite block as a typical randomly
heterogeneous medium. In the ultrasonic testing, a line-focus transducer is used to excite ultrasonic waves, 
whereas a laser Doppler vibrometer is used to pick-up the ultrasonic motion on the surface of the granite block.
The measured waveforms are analysed in the frequency domain to evaluate the travel-time 
-or each measurement point based on the Fermet's preiciple. From the ensemble of travel-times obtained thus, 
the probablity distribtuion of the travel-time is established as a function of travel-distance. 
The uncertainty of the travel-time and its spatial evolution are then investigated using the standard deviation 
of the travel-time as a measure of the uncertainty. As a result, it was found that the uncertainty is 
approximately proportional to the mean travel-time multiplied by the square root of the travel-distance. 
This is a finding that would be useful for the stochastic modeling of the wave propagation in random heterogeneous media. 
}
%
% \titlepagecontrol{1}
%
%\receivedate{2019.7.19}
% \receivedate{January 15, 1991}
%
% \def\theenumi{\alph{enumi}}  % もし enumerate 最初の箇条を (a) と
% \def\labelenumi{(\theenumi)} % したい場合・・・
%
\begin{document}
\maketitle
%%%%%%%%%%%%%%%%%%%%%%%%%%%%%%%%%%%%%%%%%%%%%%%%%%%%%%%%%%%%%%%%%%%%%%%%%%%%%
\section{はじめに}
	%\section{はじめに}
弾性波試験は材料物性や形状の非破壊評価にしばしば用いられている.建設材料の多くは,物性や形状が不均一なランダムかつ多孔質な媒体である.例えば,岩石は,複数の鉱物種で構成された,き裂やボイドを含む典型的なランダム多孔質媒体である.このような不均質媒体では,弾性波は屈折や散乱により非常に複雑な伝播挙動を示す.そのため,不均質材の弾性波試験の高精度化や信頼性の担保には,材料の不均質性を考慮した波動伝播モデルによるデータ解釈が必要となる.またそのようなモデルは,媒体のランダム性も適切に表現できるものでなければならない.波動媒体の不均質性を表現する際,物性値を平均と偏差成分に分け,偏差成分が既知の確率分布に従うとして解析が行われる.例えば,時間調和な波動場を考える場合,支配波動方程式に含まれる波数の項を,平均値と確率変数としての偏差成分に分離して記述する.このとき,波数の偏差成分がどのような確率分布に従うかは自明でなく,材料に応じて異なる可能性がある.また,確率分布が材料のどのよう性質を反映して決まるかということも,弾性波試験の観点からは重要である.これらのことは,理論解析だけで明らかにすることはできず,現実の材料をランダム不均質媒体としてモデル化するための確率分布は,弾性波の実測データを基に構築する必要がある.
 以上を踏まえ本研究では,花崗岩試料用いた超音波計測を行い,実測データから波数ベクトルが従う確率分布を推定する.花崗岩は数mm~数cm程度の結晶粒から成る多結晶質体で,超音波はき裂や粒界で散乱を起こす.そのため,材料の不均質部と強く相互作用する弾性波を室内試験で観測するための実験供試体として利用し易い.ここでは,圧電超音波探触子で花崗岩試料に表面波を励起し,その伝播挙動をレーザー振動計で可視化するとともに,波動場の位相と波数の構造を調べる.以下,超音波計測方法と計測結果,推定した位相の空間分布と,波数ベクトルの確率密度分布を順に示す.
\cite{Fujinawa})
\hspace{\parindent}
以上のことを踏まえ,本研究では,間隙スケールでの幾何形状や物性値をもとに,
不飽和多孔質体における巨視的な物質拡散挙動を評価するための数値解析
手法の開発を行う.具体的には,マルコフ連鎖モンテカルロ法を用い,任意の
形状と粒径分布をもつ固相粒子で構成される不飽和多孔質体の数値モデル
を作成する.作成した不飽和多孔質体モデルはランダムウォークによる分子拡散
シミュレーションに用い,マクロな拡散係数を求めるとともにマクロスケールでみた場合の
拡散が通常の拡散方程式に従うものであるか否かを調べる.
分子拡散による物質輸送は非常に遅いプロセスだが,そのメカニズムは拡散物質に
常に作用する.そこで本研究では,多孔質体における分子拡散による輸送特性を調べることを
第一に考え,間隙水の移動は考慮していない.また,多孔質構造の影響を見るためには,
複数の多孔質体モデル間での比較が必要となることから,計算負荷の低い2次元問題での
シミュレーションを行う.
%ただし,本論文で述べる方法は,原理的には3次元問題にもそのまま適用することができ,
%間隙水の移動による効果をランダムウォークに取り込むことは可能であることを指摘しておく.\\
ただし,本論文で述べる方法は,原理的には3次元問題にもそのまま適用することができる.
また,間隙水の移動による機械的分散効果も,拡散粒子の移動確率を異方的に
することで,ランダムウォーク・シミュレーションに取り込むことも可能である.
これらの拡張は,本稿で提案するシミュレーション手法をより現実的な問題へ
適用し,今後,実測データとの比較を行う上での次に取り組むべき課題である.\\
\hspace{\parindent}
以下,本論文では,第2節において不飽和多孔質体における2次元拡散問題の設定を示し,
続く第3節において,本研究で用いる一連の数値解析手法について述べる.
第4節では,ランダムウォークによる数値拡散解析に用いたシミュレーションモデルとその
諸元,各種解析条件を示す.第5節では,シミュレーション結果を示し,不飽和多孔質体
におけるマクロ拡散挙動に与える,固相粒子の充填構造,粒子形状,粒径分布と飽和度の
影響を調べる.最後に,本研究のまとめを今後の課題とともに述べる.\\
\hspace{\parindent}
本研究と同様なアプローチは,多孔質体における熱や物質の輸送特性を調べることを
目的とした研究にこれまでにも用いられている.例えば,X線CT試験や統計的な方法で作成した
多孔質体モデルは,間隙中の流速や物質輸送特性を調べるためにしばしば用いられてきた
\cite{Liang}$^-$\cite{Narsilio}.また,間隙水の配置を決定するためにモンテカルロ法を
用いた例も,これまでの研究でいくつか報告が行われている\cite{Berkowitz,MC}.
本研究は,これらの既往の研究で提案された手法とアイデアに負うところが大きく,個々の
数値解析手法を新規に開発するものでない.しかしながら,本研究独自の取り組みと
貢献は以下の3点にあると考えられる.1点目として,本研究では任意の形状と粒径分布を
持つ固相粒子が充填された多孔質構造モデルを作成して,マクロ拡散挙動を調べていること
が挙げられる.
これにより,水分量だけでなく,固相構造が間隙水中の物質拡散挙動に与える影響を詳細に調べることが
可能となっている.2点目は,マクロ拡散係数と水分量の関係だけでなく,異常拡散の
程度が飽和度に対してどのように変化するかを調べた点にある.また3点目として,固相粒子の形状
と粒径分布,充填構造が拡散挙動に与える影響を調べた結果,異常拡散の起源
が拡散経路の屈曲にあることを明らかにしたことが挙げられる.これら第2,第3点目に述べた知見は,
本研究で行った数値シミュレーションの結果として因果関係が明らかになったという点で
独自の貢献といえる.

\section{超音波計測実験}
	%\section{超音波計測実験}
\subsection{実験供試体}
図-\ref{fig:fig1}に実験に用いた花崗岩供試体を示す.
この試験片は岡山県万成地域の採石場で採取した万成花崗岩をブロック状に切断加工したものである.
試験片表面には,目視で認められるような欠けや割れ,風化はない.
主要鉱物は,カリ長石,ナトリウム長石,石英および雲母の四種類で,
特徴的な桃色の色合いをした箇所がカリ長石である。
試験片のサイズはは長さ$L=178$mm, 幅$W=56$mm,厚さ$H=30$mmで,
図-\ref{fig:fig1}のような$xyz$座標系において$x$方向へ伝播する
超音波を計測する.全ての計測は,試験片の上面$(z=0)$mmにおいて行う.
また,均質材における波動伝播挙動との比較を行うため,同様な
超音波計測は,アルミニウムブロックの供試体でも行う.
アルミニウム供試体のサイズは,長さ200mm,幅150mm,厚さ50mmの
直方体で,音響異方性がほとんど無いことを予め確認している。
\begin{figure}
\begin{center}
\includegraphics[clip,scale=0.5]{Figs/samples.eps}
\caption{
	超音波計測に用いた花崗岩供試体.
}
\label{fig:fig1}
\end{center}
\end{figure}
\subsection{超音波探触子(送信子)}
表面波の励起は,供試体表面に接触させた圧電超音波探触子で行う.
送信に用いた超音波探触子の外観は図-\ref{fig:fig3}のようである。
圧電素子を収納した筐体部分と,圧電素子がマウントされた
ウェッジ(シュー)部分が示されている。
圧電素子は曲率半径26.1mm,投影面積が25mm×40mmの瓦状で、
共振周波数は2MHzのものを用いている。
圧電素子は同じ曲率半径を持つウェッジ上縁部に接着されており、
ウェッジ内部を伝播した縦波が先端部に集束するように設計されている。
ウェッジ先端部を試験片に接触させて用いれば、
試験片内部を円筒状に広がる弾性波が,ウェッジ先端部から励起される。
ウェッジの先端部は幅と長さが1×50mmである。
このような線集束型の探触子を用いることで,入射位置と伝播方向を
定義して制御することができる
また,試験片表面から強い半円筒波状の超音波を励起することで,
点波源から半球状の球面波を励起した場合に比べ,幾何減衰の影響を
小さくすることができる.
\begin{figure}[h]
\begin{center}
\includegraphics[clip,scale=1.0]{Figs/fig3.eps}
\caption{
	超音波の送信に用いた線集束探触子の外観.(a)正面,(b)側面から見た様子.
}
\label{fig:fig2}
\end{center}
\end{figure}
\subsection{超音波計測装置の構成}
図-\ref{fig:fig3}に,超音波計測装置の構成を示す.
計測装置は,3軸ステージ,レーザードップラー振動計(LDV),オシロスコープ,および
高周波スクウェア−ウェーブパルサーで構成されている.
試験片は,位置を精確に調整するために,水平2軸,回転1軸の3軸ステージ上に固定する.
送信に用いる探触子は,試験片表面に接触させて固定し,
400Vの矩形電圧パルスを,スクウェア−ウェーブパルサで印加して駆動する.
受信にはLDVを用い,受信波形はオシロスコープへ転送して4,096回平均化した後,
デジタル波形としてPCに収録する.なお,サンプリング周波数は15MHz,
計測時間範囲は200$\mu$秒とした.送信波形は2MHz程度のパルス状の波形だが,
岩石供試体では低い周波数成分が主として透過する.さらに,多重散乱により
振動の継続時間が送信パルス幅より長くなることから,サンプリングレートを
若干低めにし,振動が十分に小さくなる時間まで計測時間を長めにとるように条件を
設定した。
\begin{figure}[t]
\begin{center}
\includegraphics[clip,scale=0.5]{Figs/ut_setup.eps}
\caption{
	超音波計測装置の構成.
}
\label{fig:fig3}
\end{center}
\end{figure}
\subsection{送受信位置}
図-\ref{fig:fig4}に,送信および受信領域の配置を示す.
図中の${\cal S}$は送信位置,すなわち,線集束探触子のウェッジ先端部が接触する位置を表しており,
この部分で試験片に鉛直動が加えられる.${\cal R}$はLDVでスキャンする波形観測領域を示す.
$\cal R$は20mm$\times$30mmの矩形領域にとり,計測ピッチは$x$,$y$方向とも0.5mmとすることで,
$\cal R$上の正方格子状に配置された観測点で計41×61=2,501の供試体表面における鉛直動の時刻歴波形を取得した.
ここで,観測点の$x$および$y$軸方向の間隔を,それぞれ$\Delta x,\Delta y$とすれば,
$x$方向に$i$番目,$y$方向に$j$番目の観測点の座標$(x_i,\, y_j)$は
\begin{equation}
	(x_i,\, y_j)=(x_0+i\Delta x,\, y_0+j\Delta y)
	\label{eqn:x_ij}
\end{equation}
と書くことができる.また,観測点全体がなす格子を${\cal G}$とすれば,
\begin{equation}
	{\cal G} = \left\{ 
	(x_i,\, y_j)\left| i=0,\dots N_x, \, j=0,\dots N_y  \right.
	\right\}
	\label{eqn:Grid}
\end{equation}
と表される.ただし,$N_x$と$N_y$は$x$および$y$軸方向の観測点数を表し,
格子点数や格子間隔は実際には
\begin{equation}
	\Delta x=\Delta y=0.5 {\rm mm}, \ \ N_X=41, \, N_y=61
	\label{eqn:grid_prms}
\end{equation}
\begin{equation}
	(x_0,y_0)=(0,-15){\rm mm}
	\label{eqn:grid_corner}
\end{equation}
である.ここで,$t$を時間変数として,位置$(x,y)$において観測した時刻歴波形を
$a(x,y,t)$と表す.以下では, 簡単のため$x,y$および$t$はいずれも
連続変数として扱い,$a(x,y,t)$に関する微分や積分は,観測データを使って評価する
際には,適宜離散化して評価する.
\begin{figure}[t]
\begin{center}
\includegraphics[clip,scale=0.5]{Figs/cod.eps}
\caption{
	超音波の送信位置$\cal S$と受信領域$\cal R$の配置.
}
\label{fig:fig4}
\end{center}
\end{figure}

\section{計測結果}
	%%%%%%%%%%%%%%%%%%%%%%%%%%%%%%%%%%%%%%%%%%%%%%%%%%%%%%%%%%%%%%%%%%%%%%%
実験で得られた波形データの全体を,
\begin{equation}
	{\cal D}=\left\{
		a(x,y,t)\left| (x,y,t)\in {\cal R}\times [0,T_d] \right.
	\right\}
	\label{eqn:dataset}
\end{equation}
と表す.ここに,$T_d$は計測時間範囲を意味する.
図-\ref{fig:fig4}は,データセット$\cal D$から同一の時刻$t$における振幅を取り出して作成した
鉛直振動場のスナップショットを,$t=21\mu$秒,$t=23\mu$秒について示したものである.
上段は花崗岩供試体,下段は均質材であるアルミニウムブロックを用いて計測した結果を示し,
いずれもオシロスコープで計測された振幅値[mV]でをカラ−表示したものである.
均質なアルミニウム供試体の場合,若干のゆらぎはあるものの,概ね波形を保ったまま
右($x>0$)方向に超音波が伝播し,鉛直方向に伸びる直線的な波面がはっきりと観察されている.
一方,強い不均質性を持つ花崗岩供試体では,アルミニウムと同程度の速度で
波動場が右方向へ進展していることは分かるものの,振幅のゆらぎが非常に大きく,
初動の到達位置は明確でなく,大きな振幅を持つ波動の通過後も,振動が継続する様子が
見られる.
ここで,$a(x,y,t)$の時間$t$に関するフーリエ変換を
\begin{equation}
	A(x,y,\omega)=\int_{-\infty}^{\infty} a(x,y,t)e^{i\omega t} dt
	\label{eqn:Fourier_t}
\end{equation}
とし,フーリエ変換$A(x,y,\omega)$の位相を
\begin{equation}
	\phi(x,y,\omega)=\arg\left\{ A(x,y,\omega) \right\}
	\label{eqn:phase}
\end{equation}
と表す.なお,$\omega$は角周波数を表し,位相$\phi$は$A$と
\begin{equation}
	A=\left| A \right|e^{i\phi}
	\label{eqn:phi2A}
\end{equation}
の関係にあり,$\phi$の範囲は
\begin{equation}
	\phi \in (-\pi,\pi]
	\label{eqn:dom_phi}
\end{equation}
に取る.図\ref{fig:fig5}は,$\cal D$からFFTによって求めた位相$\phi(x,y,\omega)$
の受信領域$\cal R$における空間分布を,周波数0.7MHzと1.0MHzについて示したものである.
同一の位相となる点を結ぶ曲線は波面を表し,アルミニウム供試体の場合,上下にほぼ直線的
に伸びる波面群が現れていることが分かる.0.7MHzの場合,$x=0$付近で若干波面が屈曲
している用に見える。これは,送信周波数帯域の下限に近い周波数のため,
ノイズの影響が1.0MHzの場合よりも強く現れるためと考えられる.
これに対して花崗岩供試体では,いずれの周波数においても,場所によらず波面は著しく屈曲
している.また,波面は$y$方向へ伸びる傾向は観察でき,配向性を示すことは明らかな
ものの,特定の個々の波面がどのような曲線を描いているかを視認することは困難である。
なお,図-\ref{fig:fig4}に示されるように,超音波は$x>0$の方向に進行している.
そのため,観測領域全体でみたとき,位相は$x>0$方向に増加傾向を示す。
ただし,ここでは位相の範囲を式(\ref{eqn:dom_phi})のようにしていることに注意が必要である。
その結果,位相は$-\pi$から$\pi$の間で増加し,$\pi$を超えたところで負の側に折り返される.
これにより,アルミニウム供試体に対する結果では$x$方向に鋸刃状の変動を繰返すパターンが
赤と青の縞模様となって示されている.
\begin{figure}
\begin{center}
\includegraphics[clip,scale=0.5]{Figs/snapshot.eps}
\caption{
	振動速度分布のスナップショット.
}
\label{fig:fig4}
\end{center}
\end{figure}
\begin{figure}
\begin{center}
	\includegraphics[clip,scale=0.5]{Figs/phase_xy.eps}
	\caption{位相の空間分布.}
	\label{fig:fig5}
\end{center}
\end{figure}
\begin{equation}
	\fat{k}= (k_x,k_y)= \frac{1}{2\pi} \nabla \phi (x,y,\omega)
	\label{eqn:}
\end{equation}
\begin{equation}
	{\rm Prob} [k](\omega), \ \ 
	{\rm Prob} [\theta](\omega)
	\label{eqn:}
\end{equation}
\subsection{分散関係}
波動場の分散挙動を調べるために,波数-周波数スペクトルを求める.
波数-周波数スペクトルは$x$および$y$方向のフーリエ変換により
\begin{equation}
	\hat{\hat {A}}(\xi_x,\xi_y,\omega) =
	\iint A(x,y,\omega)e^{-i(\xi_x x +\xi_y y)}dxdy
	\label{eqn:Fkk_spctr}
\end{equation}
で与えられる.ここでは、主たる波動の伝播方向は$x$方向のため,
\begin{equation}
	\hat{\hat {A}}(\xi_x,0,\omega) =
	\iint A(x,y,\omega)e^{-i\xi_x x}dxdy
	\label{eqn:Fk_spctr}
\end{equation}
を見ることで,$x$方向への分散関係を調べる.
以下では,$\xi_x=2\pi k_x$と置き,
式(\ref{eqn:Fk_spctr})の波数−周波数スペクトルを
\begin{equation}
	\bar{A}(k_x,\omega)=\hat{\hat {A}}(\xi_x,0,\omega)
	\label{eqn:def_Ak}
\end{equation}
と書き,$k_x$を$x$方向の波数と呼ぶ.
\begin{figure}
\begin{center}
	\includegraphics[clip,scale=0.32]{Figs/fkplot_Al.eps}
	\caption{(a)波数−周波数スペクトル.(b)伝播速度と周波数の関係. (アルミニウム供試体)}
	\label{fig:}
\end{center}
\end{figure}
\begin{figure}
\begin{center}
	\includegraphics[clip,scale=0.32]{Figs/fkplot_bar2.eps}
	\caption{(a)波数−周波数スペクトル.(b)伝播速度と周波数の関係. (花崗岩供試体)}
	\label{fig:}
\end{center}
\end{figure}
% メモ
%Fig.4に周波数0.4,0.6,0.9および1.2MHzの波形成分に対する位相の空間分布を示す.
%相対的に低周波の0.4と0.6MHzでは,等位相線(波面)がy軸方向に伸びる1次元的な構造を示している.
%ただし,波面は屈曲して直線的ではない.一方,0.9および1.2MHzでは,y方向への位相の揺らぎが大きく,
%平面波的な構造が見られない.Fig.5に,位相分布の勾配を中央差分で近似して求めた,
%波数ベクトルkの確率密度分布を示す.Fig.5-(a)は,波数ベクトルの大きさkに関する,
%(b)はx軸方向から測ったkの方向に関する確率密度を示している.(a)の図にあるように,k
%の確率密度は非対称かつ有限な幅をもち,ガウス分布的でもない.また,周波数が大きく
%なるにつれ分散が増加している.これは,波数と周波数の関係が確定的に定められないこと,
%高周波になる程波動場の分散性が強まるが,波数の大きさは一定範囲に留まることを示している.
%また,波数ベクトルは入射方向に配向するが,相対的に高周波の1.0MHzと1.2MHzでは配向性が低下し,
%伝搬経路の屈曲が強まることを示している.これら波数ベクトルの確率密度と周波数の関係を記述する
%法則を見出すことは今後の課題だが,花崗岩のランダム媒体としてのモデル化においては,
%このような波数ベクトルの特徴を反映する必要がある.

\section{到達時間と距離の関係 }
	本節では局所的な波数を定義し,その統計分布に基づいて花崗岩中の
波動伝播挙動を調べることで,伝播速度と到達時間のゆらぎを定量的に評価する.
\subsection{局所波数ベクトル}
フーリエ変換によって定義される波数$(\xi_x,\xi_y)$は, 観測領域全体の変動を反映した大域的な量である.
花崗岩中を伝播する波動は,各所で伝播方向や波長が変化するため,その空間変動を評価するには
局所的な波数を用いることがよりふさわしい.そこで,位相の勾配で与えらる局所的な波数ベクトル
$\fat{k}=(k_x,k_y)$を新たに定義して,その確率的な分布を調べる.
ただし,式(\ref{eqn:def_Phi})で与えられる位相$\Phi$は$\pm \pi$
を挟んで跳躍があり,そのままでは微分出来ない点が現れる.
そこで,着目点$\fat{x}=(x,y)$において微分可能となるように,アンラップされた位相$\phi$を
考え,これを
\begin{equation}
	\phi(\fat{x},\omega)=\arg\left\{  A(\fat{x},\omega) \right\}
	\label{eqn:def_phi}
\end{equation}
と表す.$\phi$と$\Phi$の関係は
\begin{equation}
	\phi(x,\omega)=\Phi(\fat{x},\omega) + 2\pi n(\fat{x},\omega )
	\label{eqn:}
\end{equation}
である.ただし,$n$は$\fat{x}$において位相$\Phi$の跳躍を打ち消すように
選ばれた整数である.
$\phi$を用いて,位置に依存した波数ベクトルを次のように定義する。
\begin{equation}
	2\pi \fat{k}= 
	2\pi (k_x,k_y)=
	\left(
		\frac{\partial \phi}{\partial x}
		, \, 
		\frac{\partial \phi}{\partial y} 
	\right)
	\label{eqn:}
\end{equation}
\subsection{波数ベクトルの計算方法}
波数ベクトルをデータ$\cal D$を用いて数値的に評価する際には,次のようにして差分近似を行う.
$\cal R$上の関数$f(x,y)$の,格子点$(x_i,\, y_j)$における値を
\begin{equation}
	f_{i,j}=f(x_i,y_j)
	\label{eqn:}
\end{equation}
と表す.この表記を用い,$\phi$の$x$方向の偏微分を次のような中央差分で近似する.
\begin{equation}
	2\pi (k_x)_{i+\frac{1}{2},j}
	=
	\left( \frac{\partial \phi }{\partial x}\right) _{i+\frac{1}{2},j}
	\simeq 
	\frac{\phi_{i+1,j}-\phi_{i,j}}{\Delta x}
	\label{eqn:fd_x}
\end{equation}
式(\ref{eqn:fd_x})の最右辺分母の項は,
\begin{equation}
	\phi_{i+1,j}-\phi_{i,j}
	=
	\arg \left(\frac{ A_{i+1,j}}{A_{i,j}}\right)
\end{equation}
と書ける.$\Delta x$が十分に小さければ,
\begin{equation}
	\left| \phi_{i+1,j}-\phi_{i,j}\right| < \pi
	\label{eqn:phi_bound}
\end{equation}
であると考えてく,その場合
\begin{equation}
	\arg \left(\frac{ A_{i+1,j}}{A_{i,j}}\right)
	=
	{\rm Arg}\left(\frac{ A_{i+1,j}}{A_{i,j}}\right)
\end{equation}
とできる.
以上より,
\begin{equation}
	\left( k_x \right)_{i+\frac{1}{2},j} 
	\simeq 
	\frac{1}{2\pi \Delta x}
	{\rm Arg} \left( \frac{ A_{i+1,j}}{A_{i,j}} \right)
	\label{eqn:kx_FD}
\end{equation}
となる.$y$方向への微分についても同様にして考えれば,$k_y$は
\begin{equation}
	\left( k_y \right)_{i,j+\frac{1}{2}} 
	\simeq 
	\frac{1}{2\pi \Delta y}
	{\rm Arg} \left( \frac{ A_{i,j+1}}{A_{i,j}} \right)
	\label{eqn:ky_FD}
\end{equation}
で与えられることが分かる.$(x_i,y_j)$における波数ベクトル$\fat{k}_{i,j}$が
必要となる場合には,
\begin{eqnarray}
	(k_x)_{i,j} &=&
	\frac{1}{2}\left\{ (k_x)_{i+\frac{1}{2},j}+ (k_x)_{i-\frac{1}{2},j} \right\}
	\label{eqn:} \\
	(k_y)_{i,j} &=&
	\frac{1}{2}\left\{ (k_y)_{i,j+\frac{1}{2}}+ (k_y)_{i,j-\frac{1}{2}} \right\}
	\label{eqn:}
\end{eqnarray}
で代用する.
\subsection{波数ベクトルの確率分布}
花崗岩供試体の物性値にはランダムな不均質性があり、花崗岩供試体で計測された波形にも
ランダムな変動が現れる.ランダム変動について確率的な特徴を見るために,
ここでは波数ベクトルの頻度分布を求め,均質材での結果と比較する.
いま,$\cal G$上で得られた観測量$g(\fat{x})$とし,その全体
$\left\{ g(\fat{x}), \fat{x}\in {\cal G}) \right\}$
について,適当な階級幅$\Delta g$でカウントした頻度分布を
$P_{\cal G} [g;\Delta g]$と書き、
"${\cal G}$で観測されたデータから得た$g$の階級幅$\Delta g$で評価した頻度分布"と読むことにする。
なお,頻度分布は正規化されており,特に断りの無い限り
\begin{equation}
	\sum_{g} P_{\cal G}[g;\Delta g]=1
	\label{eqn:normalize}
\end{equation}
であるとする.

波数ベクトルの大きさと方向を
\begin{equation}
	k=\left| \fat{k} \right|=\sqrt{k_x^2+k_y^2},  \ \ 
	\alpha = \tan^{-1} \left( \frac{k_y}{k_x} \right)
	\label{eqn:kpol}
\end{equation}
と表す。これらの正規化した頻度分布:
\[
	P_{\cal G}[k;\Delta k], \ \ 
	P_{\cal G}[\alpha;\Delta \alpha]
\]
を図\ref{fig:fig8}と図\ref{fig:fig9}に示す。
これらの図は、アルミニウム供試体、花崗岩供試体に対するものである。
(a)は$k$の、(b)は$\alpha$の頻度分布を周波数との関係で示ししている。
階級幅は$\Delta k=0.024$mm$^{-1}$, $\Delta \alpha=$7.2度である。
図中の白の実線は各周波数における平均値を,破線は平均値$\pm$標準偏差
を示したものである。
アルミニウム供試体の結果を見ると,波数と周波数は0.2MHzから2.5MHz程度の
帯域では明確な比例関係にあることがわかる。また、波数ベクトルの方向
は、同じ周波数帯域において0度付近に集中しており,$x$軸方向に非常によく
配向している様子が示されている
ただし、2MHzを超える辺りからは0度に集中した分布はしているものの、
$\alpha$の標準偏差はが大きくなっている。
これは,2MHz程度以上の周波数からは、信号/雑音比が低下しはじめている
ことを示している。以上のことを踏まえると、アルミニウム供試体では、
2.5MHz程度までは波数と周波数が比例することから、進行波成分が観測されているが、
2MHz以上はノイズの影響が次第に大きくなり,2.5MHz以上ではほとんど有意な信号成分が
偉得ていないことがわかる。
図\ref{fig:fig9}に示した花崗岩供試体の結果でも、波数と周波数の比例関係は
見られるが、その範囲は0.6MHzから2.0MHz程度とアルミニウム供試体に比べて狭い。
また、標準偏差は波数ベクトルの大きさ、向きともアルミニウムの場合に比べて
大きく、高周波側に向けてほぼ一貫して増加する。波数ベクトルの方向について
言えば、いずれの周波数でも配向の程度は均質材に比べて明らかに低く、
2.5Mz近傍で標準偏差が約90度に達して飽和し,配向性が消失している。
以上より、信号成分の含まれる周波数帯域は,大きくみてアルミニウム供試体では
0.2〜2.5MHz, 花崗岩供試体で0.6から2.0MHzであると言える。
このような情報は、波数-周波数スペクトルからは得ることができず、
局所波数ベクトルの統計を見ることの一つの利点と言える。なお、ここで言う帯域とは、
配向性を消失することなく、x方向へ伝播する波動が観測できる周波数範囲という
意味であり、一定値以上の波動振幅の得られる帯域という意味ではない。
花崗岩のような不均質媒体では、計測波形やその周波数スペクトルは複雑な形を
しているため、振幅を基準にした議論は難しいことが多い。その点で、波数ベクトルの
統計分布に基づく帯域の定義に従い、見るべき周波数範囲を定めることが
より適切と考えられる。
\subsection{到達時間関数}
観測領域内の位置$\fat{x}$における波動は,$x=0$の側から伝播してくる.
$x=0$から$\fat{x}\in {\cal R}$に波動が到達するために要する時間を
$T_f(\fat{x},\omega)$とする.$T_f$を波数ベクトルから求めることを考える.
\begin{equation}
	{\cal I}(0)={\cal R} \cup \left\{ x=0\right\}
	\label{eqn:def_I0}
\end{equation}
とし,ある$\fat{x}_0\in {\cal I}(0)$からの波動が,経路$\Gamma$を伝播して
$\fat{x}$に到達したとする(図\ref{fig:fig12}).
このとき,始点から終点までの間で生じる位相の変化は
\begin{equation}
	\phi(\fat{x},\omega)-\phi(\fat{x}_0,\omega)=
	2\pi \int_{\Gamma} \fat{k}(\fat{s},\omega)\cdot d\fat{s}
	\label{eqn:int_phi}
\end{equation}
で与えられる.ただし,$d\fat{s}$は$\Gamma$上の各点における微小接ベクトルで, 
波動伝播は位相が増加する方向に起きることから,伝播経路$\Gamma$上では,
\begin{equation}
	\fat{k}\cdot d\fat{s}\geq 0 \, {\rm on} \, \Gamma 
	\label{eqn:}
\end{equation}
であることが要請される.フェルマーの原理によれば,$\fat{x}_0$と$\fat{x}$を
結ぶ経路$\Gamma$のなかで,実際に選ばれる経路は,位相の変化を最小にするものである.
ただし,$\fat{x}$に到達する 波動伝播の起点$\fat{x}_0$が, $\cal I$(0)上のどこかにあるは
前もってわからない.そこで,フェルマーの原理を適用するにあたり,
$\fat{x}_0$と$\Gamma$は式(\ref{eqn:int_phi})を最小にするように選ばれると考える.
すなわち,$\fat{x}$における位相は,
\begin{equation}
	\phi(\fat{x},\omega) - \phi(\fat{x}_0,\omega)
	=
	\min_{\left(\Gamma, \fat{x}_0\right)}
	\int_\Gamma 2\pi \fat{k}(\fat{s},\omega) \cdot d\fat{s}
	\label{eqn:Fermat}
\end{equation}
となると考える.これを周波数$\omega$で割れば,到達時間が次のように得られる.
\begin{equation}
	T_f(\fat{x},\omega) = \frac{\phi(\fat{x},\omega)}{\omega}
	\label{eqn:def_Tf}
\end{equation}
実際には,観測波形は${\cal G}$でのみ与えられているため,$\Gamma$を
格子点を結ぶ折れ線$\tilde \Gamma$で近似する(図\ref{fig:fig12}).
\begin{equation}
	\fat{k}\cdot d\fat{s}
	=
	\left\{
	\begin{array}{c}
		\pm k_x \Delta x \\
		\pm k_y \Delta y
	\end{array}
	\right.
	\label{eqn:}
\end{equation}
さらに,各々の$\fat{x}$について経路$\Gamma$を特定するのでなく,次のようにして
$\cal G$での位相を決定する.

はじめに,$\fat{x}\in {\cal G}\cap {\cal I}(0)$を一つ選ぶ.
この点の位相を0とし,位相決定済み点としてリストに登録する.
次に,$\fat{x}$に隣接する格子点をリストアップし,これを近傍点と呼ぶ.
全ての近傍点には位相決定済みの隣接点が存在する.そこで, 式(\ref{eqn:phi_int})を離散化した
\begin{equation}
	\phi(\fat{x}+\Delta \fat{s}) -\phi(\fat{x}_0) = 2\pi \fat{k} \cdot \Delta \fat{s}
	\label{eqn:inc_phase}
\end{equation}
を用い,近傍点の位相を計算して仮登録する.このとき,$\fat{x}$は位相決定済みの点に,
$\fat{x}+\Delta \fat{s}$はその近傍点に取り,$\Delta \fat{s}$は両者の相対位置関係に
応じて次のいずれかで与える.
\begin{equation}
	\Delta \fat{s} = (\pm \Delta x,\,0) \, or \, (0,\pm \Delta y)
	\label{eqn:}
\end{equation}
なお,波数ベクトル$\fat{k}$は,式(\ref{eqn:kx_FD})あるいは式(\ref{eqn:ky_FD})で計算した
ものを用いる.また, $\fat{k}\cdot \Delta \fat{s}<0$となる方向にある近傍点の位相は未定の
ままとし,複数の位相決定済み点に隣接する近傍点には,それらの隣接点から計算された位相の
中で最小のものを仮登録する.
この作業を全ての近傍点で行った後,仮登録された位相の中で最小のものだけを採用して,
その格子点を位相決定済みのリストに加える.
更新された位相決定済み格子点に対し,近傍点のリストを作成して,位相の仮登録からの手順を繰り返す.
以上の作業を新たに登録される近傍点がなくなるまで実行する.
これら一連の位相決定作業を,${\cal G}\cap {\cal I}(0)$に含まれる全ての点を開始点として
順次行えば,式(\ref{eqn:Fermat})で表される位相$\phi(\fat{x},\omega)$を求めることができる.
なお,この方法では$\cal G$の全ての点で位相が決定できる保証は無く,与えられた波数ベクトル
場によっては,位相が未決定の格子点が残される.しかしながら、これは位相決定方法の不備ではなく,
波数ベクトル場の性質によるものと言え,観測データにはノイズが含まれ波数ベクトルが
完全に正確には求まらないこと,また,波動伝播経路が観測面内に限定されず,厳密には
3次元的であることを考えると,いくつかの点で位相が未決定のまま残ることは何ら不自然ことではない.
\begin{figure}
\begin{center}
	\includegraphics[clip,scale=0.5]{Figs/pdf_kw_Al.eps}
	\caption{波数ベクトルの(a)大きさと(b)方向の分布(アルミニウム供試体).}
	\label{fig:fig8}
\end{center}
\end{figure}
\begin{figure}
\begin{center}
	\includegraphics[clip,scale=0.5]{Figs/pdf_kw_bar.eps}
	\caption{波数ベクトルの(a)大きさと(b)方向の分布(花崗岩供試体).}
	\label{fig:fig9}
\end{center}
\end{figure}
\begin{figure}
\begin{center}
	\includegraphics[clip,scale=0.8]{Figs/unwrap.eps}
	\caption{波数ベクトルの積分経路$\Gamma$とその折れ線近似$\tilde \Gamma$のイメージ.}
	\label{fig:fig12}
\end{center}
\end{figure}
以上の方法で計算した到達時間関数$T(\fat{x},\omega)$を周波数1MHzの場合について図-\ref{fig:fig10}に
示す。アルミニウム供試体に対する結果では、到達時間はy方向にほぼ一定であり、期待通り、1次元的な
伝播挙動が再現されている。なお、到達時間が未定のまま残された格子点に関しては
到達時間を-1として表示している。そのため、到達時間未定の箇所は、図\ref{fig:fig10}において、孤立した
紺のセルとして示されている。アルミニウム供試体に対する結果の中央付近でドットが欠けたように
見えるのはこのためで、供試体表面の性状が悪く、この点でのみLDVで波形が十分な精度で計測できなかった
ことが理由として考えられる。
花崗岩供試体に対する結果では、
$x$>0の方向に次第に到達時間が遅れる傾向ははっきりしているものの、
到達時間は$y$軸方向に見たときに一様でなくゆらぎがある。また、到達時間未定の
箇所もアルミニウム供したの場合に比べて多い。これは、計測上の問題が含まれる
可能性もあるが、主として、強い散乱により波数ベクトルの方向が複雑かつ急激に変化する
ことによるものと考えられる。

このようにして得られた到達時間の見積を、頻度分布として整理した結果を
図-\ref{fig:fig11}に示す。この図は、周波数帯域を
\[
	\Omega = \left[ 2\pi f_{min}, 2\pi f_{max} \right]
\]
とするとき、頻度分布:
\begin{equation}
	P_{{\cal G \cap I}(x)\times \Omega }(T_f; \Delta T)
\end{equation}
を示したものである。
${\cal G \cap I}(x)$は,$x$同一の$x$座標をもつ格子点全体を表すため、
各$x$における$T_f$の頻度分布を全ての帯域内の全ての周波数における
データを使って求めたもので、図-\ref{fig:fig11}では
縦軸を$x$座標、横軸を到達時間$T_f$として正規化頻度をカラーマップとして
示している。
位置におうじて到達時間が遅れる。
アルミは、到達時間の分散はアルミでは位置によってほとんど変化しない。
花崗岩では、位置が遠くなるにつれて、到達時間の分散が大きくなり、
不確実性がます。
花崗岩での到達時間の平均と平均プラスマイナス標準偏差が
それぞれ白の実線と破線で示してある。
アルミの場合はこれらの曲線は互いに重なりほとんど区別がつかないので
この図には示していない。
\begin{figure}
\begin{center}
	\includegraphics[clip,scale=0.4]{Figs/phi_R.eps}
	\caption{到達時間$T_f(\fat{x},\omega)$の空間分布. 
	(a)アルミニウム供試体,(b)花崗岩供試体.いずれも周波数1.0MHzの結果.}
	\label{fig:fig10}
\end{center}
\end{figure}
\begin{figure}
\begin{center}
	\includegraphics[clip,scale=0.5]{Figs/tof_hist.eps}
	\caption{到達時間$T_f$の正規化した頻度分布. (a)アルミニウム供試体,(b)花崗岩供試体.}
	\label{fig:fig11}
\end{center}
\end{figure}
\begin{figure}
\begin{center}
	\includegraphics[clip,scale=0.5]{Figs/tof1d.eps}
	\caption{到達時間の平均$\bar T_f$と伝播距離の関係. }
	\label{fig:fig12}
\end{center}
\end{figure}
\begin{figure}
\begin{center}
	\includegraphics[clip,scale=0.5]{Figs/sigma_t.eps}
	\caption{到達時間$T_f$の標準偏差$\delta T_f$と伝播距離の関係. (a)アルミニウム供試体,(b)花崗岩供試体.}
	\label{fig:fig13}
\end{center}
\end{figure}
\begin{figure}
\begin{center}
	\includegraphics[clip,scale=0.5]{Figs/sigma_y.eps}
	\caption{到達距離$\bar{x}$の標準偏差$\delta \bar{x}$と伝播時間$T_f$の関係. (a)アルミニウム供試体,(b)花崗岩供試体.}
	\label{fig:fig14}
\end{center}
\end{figure}

\section{まとめ}
本研究では,非均質材における波動伝播特性を理解することを目的に,
花崗岩供試体を用いた超音波計測の結果から表面波の到達時間と位置の関係
を調べた.到達時間の評価は,計測波形のフーリエ位相から求めた波数ベクトル場を
経路積分することによって行う.受信点位置と周波数に関するアンサンブルから
到達時間の頻度分布を得ることで,波動の到達位置と到達時間の平均と偏差を
評価した.この方法では,波形の立ち上がりやピークを読むといった作業は必要なく,
全ての計算は波数や位相等の明確に定義された量で行うことができる.
以上の方法で,到達時間と到達位置の同時確率分布を推定したところ,均質材である
アルミニウム供試体では,到達時間のゆらぎは,到達位置にほとんど依存しないが,
花崗岩供試体では,到達位置が遠方になるに従い,多重散乱の影響を受け,
到達時刻のゆらぎが単調に増加することが示された.しかしながら,
平均到達時間で正規化した標準偏差で,到達時間のゆらぎを見ると,距離に
応じて単調減少を示し,その割合は,距離の$-1/2$乗に比例することが見出された.
言い換えれば,到達時間の不確実性は,平均到達時間と伝播距離の-1/2乗に比例する.
従って,もし伝播距離が平均到達時間の1次式となるならば,
到達時間の不確実性は到達時間の1/2乗すなわち距離の1/2乗に比例すると結論できる.
今後は,到達時間と位置の同時確率分布の詳細な分布形状を調べること,
分布を特徴付ける平均や標準偏差の物理的な起源について明らかにすることが課題となる.
特に,後者は非均質材における探査や非破壊検査への応用の上で重要な課題である.
また,ここでは到達時間に関する議論だけを行ったが,到達時間のゆらぎと散乱減衰の
関係付けや,それに基づく減衰モデリングを行うことも,ランダム非均質材中の
波動伝播挙動をより深く理解する上での重要なテーマになる.
%本研究では,花崗岩中を伝播する表面波を計測して位相の空間分布を求め,波数ベクトルの確率密度分布を推定した.その結果,媒体の不均質性によって波面が屈曲すること,波数ベクトルの確率密度分布は周波数に強く依存すること,高周波になるにつれ確率密度がブロードになること,波数は非対称かつ有限な幅の非ガウス的な確率分布に従うことが明らかとなった.今後は,波数ベクトルの空間分布構造とその周波数依存性,岩石物性との関係について調べることが,工学的な応用を展開する上での課題となる.
%%%%%%%%%%%%%%%%%%%%%%%%%%%%%%%%%%%%%%%%%%%%%%%%%%%%%%%%%%%%%%%%%%%%%%%%%%%%%
%%%%%%%%%%%%%%%%%%%%%%%%%%%%%%%%%%%%%%%%%%%%%%%%%%%%%%%%%%%%%%%%%%%%%%%%%%%%%
%\newpage
%\lastpagecontrol[2cm]{13.7cm}
\vspace{0mm}
\begin{thebibliography}{99}
%\vspace{5mm}
\begin{spacing}{1.175}
\bibitem{JAEA}
	日本原子力研究開発機構 福島研究開発部門 福島研究開発拠点 福島環境安全センター:  
	福島における放射性セシウムの環境動態研究の現状(平成30年度版),JAEA-Research,
	2019-002,DOI:10.11484/jaea-research-2019-002,2019.
\bibitem{NUMO}
	原子力発電環境整備機構:地層処分事業の安全確保(2010年度版)-確かな技術による安全な地層処分の実現のために-, NUMO-TR-11-01, 2011.
\bibitem{Bear1}
	Bear,J.:Dynamics of fluids in porous media,Dover,1972.
\bibitem{Bear2}
	Bear,J.and Bachmat,Y.:Introduction to modeling of transport phenomena in porous media, Kluwer Academic Publishers,1990.
\bibitem{Nakano}
	中野政詩:土の物質移動学,東京大学出版会, 1991.
\bibitem{Fujinawa}
	藤縄克之:環境地下水学,共立出版,2010.
\bibitem{Rubin}
	Rubin,S., Dror,I. and Berkowitz, B.: Experimental and 
		modeling analysis of coupled non-Fickian transport and 
		sorption in natural soils,Journal of Contaminant Hydrology, vol.132,pp.28-36,2012.
\newpage
\bibitem{Masuda}
	Masuda,A.,Ushida,K. and Okamoto, T. :Direct observation of spatiotemporal dependence of anomalous diffusion in 
		inhomogeneous 
		fluid by sampling-volume-controlled fluorescence correlation spectroscopy, Physical Review E, vol.72,pp.060101-1-4,2005.
\bibitem{Non_Fick_review}
	Berkowitz,B., Cortis,A. and Scher,A.:Modeling non-Fickian transport in geological formations as a continuous time random walk, Reviews of Geophysics,vol.44,pp.1-49,2006.
\bibitem{Upscaling_review}
	Frippiat,C.C. and Holeyman,A.E.:A comparative review of upscaling methods for solute transport in heterogeneous porous media, Journal of Hydrology, vol.262,pp.150-176,2008.
%\bibitem{Cuadros}
%	Cuadros, J. : Clay as sealing material in nuclear waste repositories, %Geology Today, Vol.24, No.3, pp.99-103, 2008.
\bibitem{Liang}
	Liang,Z.,Ioannidis,M.and Chatzis,I.:Permeability and electrical conductivity of porous media from 3D stochastic replicas of the microstructure, Chemical Engineering Science,vol.55,pp.5247-5262,2000. 
\bibitem{chalk_model}
	Talkudar,M.S., Torsaeter, O. and Howard, J.J.:
	Stochastic reconstruction, 3D characterization and network modeling of chalk, Journal of Petroleum Science and Engineering, 
	vol.35, pp.1-21,2002.
\bibitem{Digital_porous}
	Kainourgiakis,M.E.,Kikkinides,E.S.,Galani,Charalambop-oulou,G.C. and Stubos,A.K.:
	Digitally reconstructed porous media: Transport and sorption properties,
	Transport in Porous Media, Vol.58, pp.43-62, 2005.
\bibitem{Kim}
	Kim,D.H., Young,J.K., Lee, J.-S. and Yun, T.S..:Thermal and electrical response of unsaturated hydropholic and hydrophobic granular materials,
	 Geotechnical Testing Journal,Vol.34,No.5,pp.1-9,2011.
\bibitem{Narsilio}
	Narsilio,G.A., Kress,J. and Yun, T. S.:
	Characterisation of conduction phenomena in soils at particle-scale: 
	Finite element analyses in conjunction with syhtetic 3D imaging, Computers and Geotechnics, vol.38, pp.828-836, 2010.
\bibitem{Berkowitz}   
        Berkowitz,B. and Hansen,P.D.:A Numerical Study of the Distribution of 	Water in Partially Saturated Porous Rock, 
	Transport in Porous Media, Vol.45, pp.303-319, 2001.
\bibitem{MC}  
        Lu, N.,Zeidman, B. D., Lusk, M. T., Willson, C. S. and Wu, D. T.: 
	A Monte Carlo paradigm for capillarity in porous media, Geophysical Research Letters, Vol.37, L23402, 2010.
\bibitem{comp_phys}  
	Metropolis,N.,Rosenbluth,A.W.,Rosenbluth,M. N.,Teller,A. H. and	Teller,E.:
		Equation of state calculations by fast computing machines, 
	J.Chem.Phys., 21(6), pp.1087-1092, 1953.
\bibitem{SurfChem}
	Butt, H.-J., Graf, K. and Kappl, M., 
	"Physics and chemistry of interfaces", WILEY-VCH, 2006.
\bibitem{Kimoto}
	中島 唯一,木本 和志,河村 雄行:間隙水分布を考慮した不飽和多孔質体の熱伝導解析, 
	土木学会論文集A2, Vol.74, No.2, pp.I105-I114,2018.
\bibitem{Toda}
	戸田 盛和,物理学30講シリーズ 分子運動30講,朝倉書店,1996.
\bibitem{Rwk_textbook}
	Klaffter, J., and Sololov,I.M.,秋元 琢磨(訳): 
	ランダムウォークはじめの一歩,共立出版, 2018.
\bibitem{Gennes}
ドゥジェンヌ,ブロシェール ヴィアール,ケレ:表面張力の物理学, 吉岡書店, 2003.
%\bibitem{Terada}
%	寺田 賢二郎, 菊池 昇: 均質化法入門, 丸善株式会社, 2003.
%\bibitem{Zohdi}
%        Zohdi, T. I. and Wriggers, P. : An Intruduction to Computational Micromechanics,Springer, 2008.
%\bibitem{Iwasaki}
%	岩崎 佳介, 木本 和志, 市川 康明: 含水した砂質土の熱応答特性に関する実験および数値解析,  土木学会論文集A2, Vol.70, No.2,  ppI115-I124, 2014.
%\bibitem{Yun}
%	Yun, T. S., and Evans, T. M. : Three-dimensional random walk network model %for thermal conductivity in pariculate materials, 
%	Computers and Geotechnics, vol.37, pp.991-998,2010.
\end{spacing}
\end{thebibliography}
\vspace{-5mm}
\begin{flushright}
	\small
	\bf{ (Received July 19, 2019)\\
	(Accepted December 24, 2019)}
\end{flushright}
%\lastpagesettings
\newpage
%\lastpagecontrol[1cm]{13.7cm}
\lastpagecontrol[1cm]{25.7cm}
\end{document}

%\begin{minipage}[c]{13.7cm}
%\end{minipage}
%\lastpagecontrol[0cm]{13.7cm}
%\newpage
%\begin{multicols}{1}
%-------------------------------------------------
%-------------------------------------------------
%\end{multicols}

