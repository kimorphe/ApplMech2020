%%#!platex
%
% Example of Japanese Paper of JSCE
% for LaTeX2e users
%
% revised on 4/25/2014
%
%%%%%%%%%%%%%%%%%%%%%%%%%%%%%%%%%%%%%%%%%%%
%
% もし jis フォントメトリックを使う場合は,以下をアンコメントしてください.
% \DeclareFontShape{JY1}{mc}{m}{n}{<-> s * jis}{}
% \DeclareFontShape{JY1}{gt}{m}{n}{<-> s * jisg}{}
%
\documentclass{jsce}
%
\usepackage{epic,eepic,eepicsup}
%\usepackage{graphicx,multicol}
\usepackage{graphicx}
\usepackage{multicol}
\usepackage{amsmath}
%\usepackage{showkeys}
\usepackage{setspace}
%  amsを使う方は以下をアンコメントしてください.
%\usepackage{amssymb,amsmath}
% 英語はサポートしているかどうか不明
% \inenglish
% 学会サンプルに times とあるので指定しておきます
\usepackage{times}
%
\finalversion
\pagestyle{empty}
%
\title{
	超音波計測に基づく\\花崗岩中の表面波伝播特性に関する研究
}%
\endtitle{
A STUDY ON THE PROPAGATION CHARACTERISTICS OF SURFACE WAVES IN GRANITE BASED ON ULTRASONIC MEASUREMENTS
}
%
% emailアドレスのフォントをタイプライター体にしたい方は次行をアンコメント
% \emailstyle{\ttfamily}
% emailアドレスを公開される方は,
%% \thanks{○○○○○○\email{your_name@foo.ac.jp}}のようにしてください.
%
\author{木本 和志\thanks{正会員 博士(工学) 岡山大学 環境生命科学研究科(〒700-8530 岡山県岡山市北区3丁目1番地1号)\email{kimoto@okayama-u.ac.jp}}・
岡野 蒼\thanks{学生会員 岡山大学環境生命科学研究科 (〒700-8530 岡山県岡山市北区3丁目1番地1号)}・
斎藤 隆泰\thanks{正会員 博士(工学)群馬大学大学院理工学府 環境創生部門(〒376-8515 群馬県桐生市天神町 1-5-1)}・
佐藤 忠信\thanks{正会員 博士(工学)神戸学院大学・現代社会学部・防災社会学科(〒650-8586神戸市中央区港島1-1-3)}・
松井 裕哉\thanks{正会員 修士(工学)日本原子力研究開発機構・幌延深地層研究センター・堆積岩処分技術開発Gr
(〒098-3224 北海道天塩郡幌延町北進432番地2)}
}
\endauthor{Kazushi KIMOTO, Aoi OKANO, Takahiro SAITOH,\\ Tadanobu SATO and Hiroya MATSUI }
%
\abstract{
\small
本研究は,ランダム不均質媒体における表面波の伝播挙動を,超音波計測と波形解析によって調べたものである.
超音波計測実験には,粗粒結晶質岩である花崗岩をランダム不均質媒体として用い,線集束型の圧電探触子で励起した
表面波をレーザードップラー振動計で計測する.計測波形の解析は周波数領域で行い,フェルマーの原理に基づいて
各波形観測点での到達時間を求める.この方法で得られた到達時間のアンサンブルから,到達時間が従う確率分布を伝播距離の関数として評価する.次に,確率分布の標準偏差を到達時間の不確実性(ゆらぎ)の指標として用い,
伝播距離に応じたゆらぎの伝播挙動を調べる.以上の波形解析結果から,本研究に用いた花崗岩試料における
到達時間のゆらぎは,概ね伝播距離の$-1/2$乗と平均到達時間の積に比例することを明らかにする.
このことは,ランダム不均質媒体における統計的波動伝播モデリングにおいて有用な知見となる.
}
%
\keywords{random media, heterogeneity, ultrasonic wave, uncertainty, travel time}
%
\endabstract{% Yes blank line
\normalsize
This study investigates the propagation characteristics of surface wave traveling in a random heterogeneous medium. 
For this purpose, ultrasonic measurements are performed on a coarse-grained granite block as a typical randomly heterogeneous medium. In the ultrasonic testing, a line-focus transducer is used to excite ultrasonic waves, whereas a laser Doppler vibrometer is used to pickup the ultrasonic motion on the surface of the granite block. The measured waveforms are analysed in the frequency domain to evaluate the travel-time for each measurement point based on the Fermat's principle. From the ensemble of travel-times obtained thus, 
the probability distribution of the travel-time is established as a function of travel-distance. The uncertainty of the travel-time and its spatial evolution are then investigated using the standard deviation of the travel-time as a measure of the uncertainty. As a result, it was found that the uncertainty is 
approximately proportional to the mean travel-time divided by the square root travel-distance.This is a finding that would be useful for stochastic modeling of the waves in random heterogeneous media. 
}
%
% \titlepagecontrol{1}
%
%\receivedate{2019.7.19}
% \receivedate{January 15, 1991}
%
% \def\theenumi{\alph{enumi}}  % もし enumerate 最初の箇条を (a) と
% \def\labelenumi{(\theenumi)} % したい場合・・・
%
\begin{document}
\maketitle
%%%%%%%%%%%%%%%%%%%%%%%%%%%%%%%%%%%%%%%%%%%%%%%%%%%%%%%%%%%%%%%%%%%%%%%%%%%%%
\section{はじめに}
	建設分野において弾性波検査の対象となる材料の多くは,形状や物性が不規則かつ不均一なランダム媒体である.
例えば,コンクリートは,粒径や形状が異なる骨材と気泡がランダムに分布した非均質媒体である.
また,岩盤や岩石も断層や節理系から,岩石を構成する鉱物粒や介在物,マイクロクラックに至るまで,
各種の空間スケールで多様な非均質性を有する\cite{RockPhys}.
このような非均質ランダム媒体において,弾性波は不均質部との相互作用によって散乱や屈折を起こし,
複雑な伝播挙動を示す.そのため,地震探査や岩石コア,多結晶質材の弾性波検査やイメージング
には,緻密な金属材料のような均質材に対する非破壊検査には無い困難が伴う\cite{Sato,Borcea,Thompson}.
特に,弾性波が強い多重散乱を起こしながら媒体を伝播する場合,著しい減衰や波形の変化のために,
計測で得られた波形から有用な情報を取り出すことが,一般に難しい.このような困難を克服し,
ランダム不均質媒体に対する信頼性や精度の高い弾性波検査技術を開発するには,多重散乱効果を
考慮した波動伝播モデルの構築が必要となる.

物理探査や非破壊検査において反射源位置を特定する際,弾性波速度が既知である必要がある\cite{Etgen, Schmitz}.
この理由から,種々の弾性波伝播特性の中でも伝播速度は重要と言え,このことは不均質媒体
でも,少なくともバックグラウンドの弾性波速度が必要となる点では同様である\cite{Langenberg, Bleistein}.
ただし,ランダム媒体の場合,媒体の物性値が場所によって異なるため,計測点毎に弾性波の到達時間と,
そこから見積もられる弾性波速度には必然的にばらつきが生じる.弾性波速度のばらつきは反射源位置の同定精度と
不確実性に影響するため,ランダム媒体に関しては弾性波速度の平均値だけでなくばらつきも重要な情報となる.
また,弾性波速度のばらつきは媒体の不均質性を反映したものであることから,ランダムな不均質性を弾性波計測
データから調べる目的においては,速度のばらつきを含め,弾性波速度や到達時間が従う統計分布自体が
興味の対象となる\cite{Yu,Li}.

ランダム不均質媒体における弾性波速度のばらつきを調べるために,これまで,種々の理論,数値解析および実験的
研究が行われてきた\cite{NishizawaI}.例えば,理論および数値解析的な研究には,波線理論や1次散乱理論を用いて
伝播時間解析を行ったもの\cite{Muller, Korn, Spetzler2001}や,差分法モデルでランダム媒体中を伝播する波動の
解析を行いその結果を理論解析と比較したもの\cite{Spetzler}などがある.一方,実験的な研究には,
岩石試料を透過する超音波をレーザードップラー振動計で計測し,弾性波速度のばらつきと鉱物粒径との関係を調べたもの
\cite{Nishizawa1996,Nishizawa2001}や,個々の計測波形と平均波形の乖離やP-S波間でのエネルギー分配の挙動を
不均質性スケールとの関係で調べたものなど\cite{Sivaji,Fukushima}がある.
これら実験的な研究の成果は,弾性波計測結果から不均質性のスケールや強度を推定する上で重要なものと言える.
一方で,伝播時間や伝播速度の揺らぎと,伝播距離や方向の関係は実験的には調べられておらず,
例えば波線理論や散乱理論による予測と一致するかどうかはこれまで明らかにされていない.

以上を踏まえ本研究では,弾性波伝播時間のばらつきが伝播距離に応じてどのように変化するかを明らかにすることを目的に,
超音波計測を実施する.実験では,典型的なランダム不均質媒体である花崗岩を供試体として用い,
圧電トランスデューサで励起した表面波の振動を,レーザードップラー振動計(LDV)で多点計測する.
LDVを用いて表面波を対象とした計測を行う理由は,試料表面の超音波振動を高い時空間解像度と広い周波数帯域で
観測することにより,波動場の伝播状況を精確に捉えることを意図したものである.
超音波の送信には,接触型の線集束トランスデューサを用い試料内部に円筒波を励起する.
これにより,強い超音波を送信できるだけでなく,入射方向と伝播距離を明確に定義することが可能となる.
一連の計測で得られた波形は,周波数領域において解析し,フェルマーの原理に基づき各観測点における到達時間を求める.
このようにして得られた到達時間のアンサンブルから,到達時間の確率分布を,伝播距離の関数として求める.
最後に,到達時間の平均と標準偏差を評価し,到達時間のばらつきが距離に応じてどのような法則に従い変化するかを
明らかにする.

以下では,はじめに超音波計測の方法について述べる.次に,計測で得られた波形データから表面振動の様子を可視化し,
どのような波動場が供試体表面に形成されているかを示す.続いて,各観測点と周波数における到達時間を求める波形解析
方法を示し,計測波形から求めた到達時間の空間分布を示す.最後に,到達時間の確率分布とその平均,標準偏差を
伝播距離の関数として求めた結果を示し,到達時間の不確実性が空間的にどのように発展するかを考察する.

\section{超音波計測実験}
	%\section{超音波計測実験}
\subsection{実験供試体}
実験に用いた花崗岩供試体を図-\ref{fig:fig1}に示す.
この供試体は,岡山県万成地域の採石場で採取した万成花崗岩をブロック状に切断加工したもので,
供試体表面には,目視で認められるような欠けや割れ,明らかな風化はない.
万成花崗岩の主要造岩鉱物は,カリ長石,ナトリウム長石,石英および雲母の四種類で,
特徴的な桃色の色合いをした箇所がカリ長石である.
試験片のサイズは長さ$L=178$mm, 幅$W=56$mm,厚さ$H=30$mmで,
計測位置は図-\ref{fig:fig1}のような$xyz$座標系で表す.
超音波の送信と受信は,試験片の上面$(z=0)$mmにおいて行い,$x$軸の正方向へ伝播する表面波を計測する.
また,均質材における波動伝播挙動との比較を行うため,同様な計測を,アルミニウムブロック供試体でも行う.
アルミニウム供試体のサイズは,長さ200mm,幅150mm,厚さ50mmの直方体で,後に述べる送受信位置のとり方は,花崗岩供試体の場合と同様である.
\begin{figure}
\begin{center}
\includegraphics[clip,scale=0.5]{Figs/samples.eps}
\caption{
	超音波計測に用いた花崗岩供試体.
}
\label{fig:fig1}
\end{center}
\end{figure}
\subsection{超音波探触子(送信子)}
超音波の送信は,供試体表面に接触させた圧電超音波探触子で行う.
実験に用いた超音波探触子の外観は図-\ref{fig:fig3}のようであり,この図には
圧電素子を収納した筐体部分と,圧電素子がマウントされたウェッジ(シュー)部分が示されている.
内部に収納された圧電素子は,曲率半径が26.1mm,投影面積が25mm×40mmの瓦状のもので,共振周波数は2MHzである.
圧電素子は,曲率半径を合わせて作成されたウェッジ上縁部に接着されており,
圧電素子で励起した縦波がウェッジ内部を伝播して先端部に集束するよう設計されている.
従ってウェッジ先端部を供試体に接触させて用いることで,供試体内部を円筒状に広がる弾性波が,線状の接触部から励起される.
なお,供試体に接触させるウェッジの先端部は幅と長さは1×50mmとなっている.
このような線集束型の探触子を用いることで,入射点と伝播方向が明確に設定される.
また,供試体表面から強い半円筒波状の超音波を励起することで,点波源から半球状の球面波を励起した場合に比べ,幾何減衰の影響も小さくすることができ,信号/雑音比の点で有利になる.
\begin{figure}[h]
\begin{center}
\includegraphics[clip,scale=1.0]{Figs/fig3.eps}
\caption{
	超音波の送信に用いた線集束探触子の外観.(a)正面,(b)側面から見た様子.
}
\label{fig:fig2}
\end{center}
\end{figure}
\subsection{超音波計測装置の構成}
実験に用いた超音波計測装置の構成を図-\ref{fig:fig3}に示す.計測装置は,3軸ステージ,レーザードップラー振動計(LDV),オシロスコープ,および高周波スクウェア−ウェーブパルサーで構成されている.
供試体は水平2軸,回転1軸の3軸ステージ上に固定し ,LDVによるレーザー照射位置を精確に調整する.
その際,送信探触子は,試験片表面に接触させて固定し,供試体とともに移動させる.
探触子の駆動はスクウェア−ウェーブパルサーを用いて行い,400Vの矩形パルス電圧を印加する.
受信にはLDVを用い,受信波形をオシロスコープへ転送し,4,096回の平均化を行った後,デジタル波形としてPCに収録する.サンプリング周波数は15MHz,計測時間範囲は200$\mu$秒とし,全ての計測は同じ条件で行った.
送信探触子の公称周波数は2MHzであるため,これに対してサンプリング周波数はやや低めに設定されている.
しかしながら,花崗岩供試体では低い周波数成分が主として透過し,多重散乱により振動の継続時間も送信パルス幅より長くなる傾向にある.このことに配慮し,ここではサンプリングレートを若干低めにし,計測時間範囲を余裕をもって設定することとした.
\begin{figure}[t]
\begin{center}
\includegraphics[clip,scale=0.5]{Figs/ut_setup.eps}
\caption{
	超音波計測装置の構成.
}
\label{fig:fig3}
\end{center}
\end{figure}
\subsection{送受信位置}
図-\ref{fig:fig4}に,送信および受信領域の配置を示す.
ここで,${\cal S}$は送信位置,すなわち,線集束探触子のウェッジ先端が接触する位置を表し,この部分で試験片に鉛直動が加えられる.
${\cal R}$はLDVでスキャンする波形観測領域を表し,その大きさと形状は20mm$\times$30mmの矩形領域になっている.計測ピッチは$x$方向,$y$方向とも0.5mmとし,$\cal R$上の正方格子状に配置された観測点で計41×61=2,501の超音波時刻歴波形を取得する.
なお,送信位置と受信領域の距離は20mmとしている.
これは,送信探触子の筐体に遮蔽され,レーザー光を直接照射することのできない領域が存在するためである.
アルミニウム供試体における観測では,座標原点を供試体表面の中央に取る他は,花崗岩供試体の場合に同じとした.

ここで,観測点格子の$x$および$y$軸方向間隔を,それぞれ,$\Delta x,\Delta y$とすれば,
$x$方向に$i$番目,$y$方向に$j$番目の観測点座標$(x_i,\, y_j)$は
\begin{equation}
	(x_i,\, y_j)=(x_0+i\Delta x,\, y_0+j\Delta y)
	\label{eqn:x_ij}
\end{equation}
と書くことができる.また,観測点が成す格子全体を${\cal G}$とすれば,
\begin{equation}
	{\cal G} = \left\{ 
	(x_i,\, y_j)\left| i=0,\dots N_x, \, j=0,\dots N_y  \right.
	\right\}
	\label{eqn:Grid}
\end{equation}
と表される.ただし,$N_x$と$N_y$は$x$および$y$軸方向の観測点数を表す.
実際の格子(観測)点数や格子間隔は,既に述べた通りであり,それらをまとめて示すと以下の通りとなる.
\begin{equation}
	\Delta x=\Delta y=0.5 {\rm mm}
	\label{eqn:grid_prms}
\end{equation}
\begin{equation}
	N_X=41, \, N_y=61
	\label{eqn:grid_nums}
\end{equation}
\begin{equation}
	(x_0,y_0)=(0,-15){\rm mm}
	\label{eqn:grid_corner}
\end{equation}
である.以下では,$t$を時間変数とし,位置$(x,y)$において観測した時刻歴波形を$a(x,y,t)$と表す.
簡単のため,$x,y$および$t$はいずれも連続変数として表記するが,$a(x,y,t)$に関する微分や積分などの演算を観測データに施す場合,観測点位置での値を使い,適宜離散化して評価する.
\begin{figure}[t]
\begin{center}
\includegraphics[clip,scale=0.5]{Figs/cod.eps}
\caption{
	超音波の送信位置$\cal S$と受信領域$\cal R$の配置.
}
\label{fig:fig4}
\end{center}
\end{figure}

\section{計測結果}
	%%%%%%%%%%%%%%%%%%%%%%%%%%%%%%%%%%%%%%%%%%%%%%%%%%%%%%%%%%%%%%%%%%%%%%%
実験で得られた波形データの全体を,
\begin{equation}
	{\cal D}=\left\{
		a(x,y,t)\left| (x,y,t)\in {\cal R}\times [0,T_d] \right.
	\right\}
	\label{eqn:dataset}
\end{equation}
と表す.ここに,$T_d$は計測時間範囲を意味する.
図-\ref{fig:fig4}は,データセット$\cal D$から同一の時刻$t$における振幅を取り出して作成した
鉛直振動場のスナップショットを,$t=21\mu$秒,$t=23\mu$秒について示したものである.
上段は花崗岩供試体,下段は均質材であるアルミニウムブロックを用いて計測した結果を示し,
いずれもオシロスコープで計測された振幅値[mV]でをカラ−表示したものである.
均質なアルミニウム供試体の場合,若干のゆらぎはあるものの,概ね波形を保ったまま
右($x>0$)方向に超音波が伝播し,鉛直方向に伸びる直線的な波面がはっきりと観察されている.
一方,強い不均質性を持つ花崗岩供試体では,アルミニウムと同程度の速度で
波動場が右方向へ進展していることは分かるものの,振幅のゆらぎが非常に大きく,
初動の到達位置は明確でなく,大きな振幅を持つ波動の通過後も,振動が継続する様子が
見られる.
ここで,$a(x,y,t)$の時間$t$に関するフーリエ変換を
\begin{equation}
	A(x,y,\omega)=\int_{-\infty}^{\infty} a(x,y,t)e^{i\omega t} dt
	\label{eqn:Fourier_t}
\end{equation}
とし,フーリエ変換$A(x,y,\omega)$の位相を
\begin{equation}
	\phi(x,y,\omega)=\arg\left\{ A(x,y,\omega) \right\}
	\label{eqn:phase}
\end{equation}
と表す.なお,$\omega$は角周波数を表し,位相$\phi$は$A$と
\begin{equation}
	A=\left| A \right|e^{i\phi}
	\label{eqn:phi2A}
\end{equation}
の関係にあり,$\phi$の範囲は
\begin{equation}
	\phi \in (-\pi,\pi]
	\label{eqn:dom_phi}
\end{equation}
に取る.図\ref{fig:fig5}は,$\cal D$からFFTによって求めた位相$\phi(x,y,\omega)$
の受信領域$\cal R$における空間分布を,周波数0.7MHzと1.0MHzについて示したものである.
同一の位相となる点を結ぶ曲線は波面を表し,アルミニウム供試体の場合,上下にほぼ直線的
に伸びる波面群が現れていることが分かる.0.7MHzの場合,$x=0$付近で若干波面が屈曲
している用に見える。これは,送信周波数帯域の下限に近い周波数のため,
ノイズの影響が1.0MHzの場合よりも強く現れるためと考えられる.
これに対して花崗岩供試体では,いずれの周波数においても,場所によらず波面は著しく屈曲
している.また,波面は$y$方向へ伸びる傾向は観察でき,配向性を示すことは明らかな
ものの,特定の個々の波面がどのような曲線を描いているかを視認することは困難である。
なお,図-\ref{fig:fig4}に示されるように,超音波は$x>0$の方向に進行している.
そのため,観測領域全体でみたとき,位相は$x>0$方向に増加傾向を示す。
ただし,ここでは位相の範囲を式(\ref{eqn:dom_phi})のようにしていることに注意が必要である。
その結果,位相は$-\pi$から$\pi$の間で増加し,$\pi$を超えたところで負の側に折り返される.
これにより,アルミニウム供試体に対する結果では$x$方向に鋸刃状の変動を繰返すパターンが
赤と青の縞模様となって示されている.
\begin{figure}
\begin{center}
\includegraphics[clip,scale=0.5]{Figs/snapshot.eps}
\caption{
	振動速度分布のスナップショット.
}
\label{fig:fig4}
\end{center}
\end{figure}
\begin{figure}
\begin{center}
	\includegraphics[clip,scale=0.5]{Figs/phase_xy.eps}
	\caption{位相の空間分布.}
	\label{fig:fig5}
\end{center}
\end{figure}
\begin{equation}
	\fat{k}= (k_x,k_y)= \frac{1}{2\pi} \nabla \phi (x,y,\omega)
	\label{eqn:}
\end{equation}
\begin{equation}
	{\rm Prob} [k](\omega), \ \ 
	{\rm Prob} [\theta](\omega)
	\label{eqn:}
\end{equation}
\subsection{分散関係}
波動場の分散挙動を調べるために,波数-周波数スペクトルを求める.
波数-周波数スペクトルは$x$および$y$方向のフーリエ変換により
\begin{equation}
	\hat{\hat {A}}(\xi_x,\xi_y,\omega) =
	\iint A(x,y,\omega)e^{-i(\xi_x x +\xi_y y)}dxdy
	\label{eqn:Fkk_spctr}
\end{equation}
で与えられる.ここでは、主たる波動の伝播方向は$x$方向のため,
\begin{equation}
	\hat{\hat {A}}(\xi_x,0,\omega) =
	\iint A(x,y,\omega)e^{-i\xi_x x}dxdy
	\label{eqn:Fk_spctr}
\end{equation}
を見ることで,$x$方向への分散関係を調べる.
以下では,$\xi_x=2\pi k_x$と置き,
式(\ref{eqn:Fk_spctr})の波数−周波数スペクトルを
\begin{equation}
	\bar{A}(k_x,\omega)=\hat{\hat {A}}(\xi_x,0,\omega)
	\label{eqn:def_Ak}
\end{equation}
と書き,$k_x$を$x$方向の波数と呼ぶ.
\begin{figure}
\begin{center}
	\includegraphics[clip,scale=0.32]{Figs/fkplot_Al.eps}
	\caption{(a)波数−周波数スペクトル.(b)伝播速度と周波数の関係. (アルミニウム供試体)}
	\label{fig:}
\end{center}
\end{figure}
\begin{figure}
\begin{center}
	\includegraphics[clip,scale=0.32]{Figs/fkplot_bar2.eps}
	\caption{(a)波数−周波数スペクトル.(b)伝播速度と周波数の関係. (花崗岩供試体)}
	\label{fig:}
\end{center}
\end{figure}
% メモ
%Fig.4に周波数0.4,0.6,0.9および1.2MHzの波形成分に対する位相の空間分布を示す.
%相対的に低周波の0.4と0.6MHzでは,等位相線(波面)がy軸方向に伸びる1次元的な構造を示している.
%ただし,波面は屈曲して直線的ではない.一方,0.9および1.2MHzでは,y方向への位相の揺らぎが大きく,
%平面波的な構造が見られない.Fig.5に,位相分布の勾配を中央差分で近似して求めた,
%波数ベクトルkの確率密度分布を示す.Fig.5-(a)は,波数ベクトルの大きさkに関する,
%(b)はx軸方向から測ったkの方向に関する確率密度を示している.(a)の図にあるように,k
%の確率密度は非対称かつ有限な幅をもち,ガウス分布的でもない.また,周波数が大きく
%なるにつれ分散が増加している.これは,波数と周波数の関係が確定的に定められないこと,
%高周波になる程波動場の分散性が強まるが,波数の大きさは一定範囲に留まることを示している.
%また,波数ベクトルは入射方向に配向するが,相対的に高周波の1.0MHzと1.2MHzでは配向性が低下し,
%伝搬経路の屈曲が強まることを示している.これら波数ベクトルの確率密度と周波数の関係を記述する
%法則を見出すことは今後の課題だが,花崗岩のランダム媒体としてのモデル化においては,
%このような波数ベクトルの特徴を反映する必要がある.

\section{到達時間と距離の関係 }
	本節では,局所量としての波数ベクトルをあらためて定義し,
その統計分布に基づいて花崗岩中の波動伝播挙動を調べることで,
伝播速度,到達時間とゆらぎを定量的に評価する.
\subsection{局所波数ベクトル}
フーリエ変換によって定義される波数$(\xi_x,\xi_y)$は,観測領域全体の情報を反映した
大域的な量と言える.花崗岩中を伝播する波動は,各所で伝播方向や波長が変化するため,
そのような空間変動を評価するには局所的な量として定義した波数を用いることがより適切である.
そこで,位相の勾配に基づく波数ベクトル$\fat{k}=(k_x,k_y)$を新たに定義し,
その統計的な分布を調べる.ただし,式(\ref{eqn:def_Phi})で与えられる位相$\Phi$は$\pm \pi$
を挟んで跳躍があり,そのままでは微分出来ない点が現れる.
そこで,着目点$\fat{x}=(x,y)$において微分可能となるようにアンラップされた位相$\phi$を
考え,これを式(\ref{eqn:def_Phi})と区別し
\begin{equation}
	\phi(\fat{x},\omega)=\arg\left\{  A(\fat{x},\omega) \right\}
	\label{eqn:def_phi}
\end{equation}
と表す.$\phi$と$\Phi$の間には
\begin{equation}
	\phi(\fat{x},\omega)=\Phi(\fat{x},\omega) + 2\pi n(\fat{x},\omega )
	\label{eqn:unwrap}
\end{equation}
の関係があり,$n$は$\fat{x}$において位相$\Phi$の跳躍を打ち消すように
選ばれた整数を表す.ここで,位相$\phi$を用いて,
局所量としての波数ベクトル$\fat{k}=(k_x,k_y)$を
\begin{equation}
	2\pi \fat{k}= 
	2\pi (k_x,k_y)=
	\left(
		\frac{\partial \phi}{\partial x}
		, \, 
		\frac{\partial \phi}{\partial y} 
	\right)
	\label{eqn:kvec_def}
\end{equation}
で定義する.このように定義した波数ベクトルは,
位置$\fat{x}=(x,y)$と角周波数$\omega$の関数となる.
なお,$2\pi$の因子を式(\ref{eqn:kvec_def})のようにつけることで,
波数ベクトル$(k_x,\,k_y)$は局所的な波長の逆数,すなわち,
空間周波数としての意味を持つことになる.
\subsection{波数ベクトルの計算方法}
波数ベクトルを観測データ$\cal D$から数値的に評価するには,
以下のようにして差分近似を行う.

$\cal R$上の関数$f(x,y)$の,格子点$(x_i,\, y_j)$における値を
\begin{equation}
	f_{i,j}=f(x_i,y_j)
	\label{eqn:}
\end{equation}
と表す.この表記を用い,$\phi$の$x$方向への偏微分を次のように中央差分で近似する.
\begin{equation}
	2\pi (k_x)_{i+\frac{1}{2},j}
	=
	\left( \frac{\partial \phi }{\partial x}\right) _{i+\frac{1}{2},j}
	\simeq 
	\frac{\phi_{i+1,j}-\phi_{i,j}}{\Delta x}
	\label{eqn:fd_x}
\end{equation}
ここで,式(\ref{eqn:fd_x})最右辺の分子の項は,
\begin{equation}
	\phi_{i+1,j}-\phi_{i,j}
	=
	\arg \left(\frac{ A_{i+1,j}}{A_{i,j}}\right)
\end{equation}
と書くことができる.格子間隔$\Delta x$が十分に小さければ,この項は
\begin{equation}
	\left| \phi_{i+1,j}-\phi_{i,j}\right| < \pi
	\label{eqn:phi_bound}
\end{equation}
としてよく,その場合
\begin{equation}
	\arg \left(\frac{ A_{i+1,j}}{A_{i,j}}\right)
	=
	{\rm Arg}\left(\frac{ A_{i+1,j}}{A_{i,j}}\right)
\end{equation}
とできる.以上より,
\begin{equation}
	\left( k_x \right)_{i+\frac{1}{2},j} 
	\simeq 
	\frac{1}{2\pi \Delta x}
	{\rm Arg} \left( \frac{ A_{i+1,j}}{A_{i,j}} \right)
	\label{eqn:kx_FD}
\end{equation}
となり,式(\ref{eqn:unwrap})の$n$を$(\fat{x},\omega)$毎に求めることなく
波数$k_x$を得ることができる.
$y$方向への微分についても同様に考えれば,$k_y$は
\begin{equation}
	\left( k_y \right)_{i,j+\frac{1}{2}} 
	\simeq 
	\frac{1}{2\pi \Delta y}
	{\rm Arg} \left( \frac{ A_{i,j+1}}{A_{i,j}} \right)
	\label{eqn:ky_FD}
\end{equation}
となることが分かる.なお,$(x_i,y_j)$における波数ベクトル$\fat{k}_{i,j}$が
必要な場合は,
\begin{eqnarray}
	(k_x)_{i,j} &=&
	\frac{1}{2}\left\{ (k_x)_{i+\frac{1}{2},j}+ (k_x)_{i-\frac{1}{2},j} \right\}
	\label{eqn:} \\
	(k_y)_{i,j} &=&
	\frac{1}{2}\left\{ (k_y)_{i,j+\frac{1}{2}}+ (k_y)_{i,j-\frac{1}{2}} \right\}
	\label{eqn:}
\end{eqnarray}
で代用する.
\subsection{波数ベクトルの確率分布}
波数ベクトル$\fat{k}$の統計的な特徴を見るために,$\fat{k}$の頻度分布を求める.
ここで,格子$\cal G$上で得られた観測量を$g(\fat{x})$とし,その全体
$\left\{ g(\fat{x}), \fat{x}\in {\cal G}) \right\}$について,階級幅$\Delta g$で
カウントした頻度分布を$P_{\cal G} [g;\Delta g]$と書くことにする.
またこれを,"${\cal G}$で観測したデータ$g$の,階級幅$\Delta g$で評価した頻度分布"と読むことにする.
なお,頻度分布は次のように正規化されているものとする.
\begin{equation}
	\sum_{g} P_{\cal G}[g;\Delta g]=1
	\label{eqn:normalize}
\end{equation}
頻度分布と確率密度関数は厳密には同じでないが,本研究では頻度分布の階級幅を
一定とするため,分布形状や平均値等の値は,確率密度関数を使って全ての議論を行った場合と変わらない.
このことから,正規化した頻度分布と確率分布を同一視して議論を行う.
\subsection{波数ベクトルの分布特性}
波数ベクトルの大きさと方向を
\begin{equation}
	k=\left| \fat{k} \right|=\sqrt{k_x^2+k_y^2},  \ \ 
	\alpha = \tan^{-1} \left( \frac{k_y}{k_x} \right)
	\label{eqn:kpol}
\end{equation}
と表す.これら正規化した頻度分布:
\[
	P_{\cal G}[k;\Delta k], \ \ 
	P_{\cal G}[\alpha;\Delta \alpha]
\]
を,図-\ref{fig:fig9}と図-\ref{fig:fig10}に示す.
これらの図は,それぞれアルミニウム供試体,花崗岩供試体に対する結果で,
(a)は$k$の,(b)は$\alpha$の頻度分布を周波数との関係で示している.
階級幅は$\Delta k=0.024$mm$^{-1}$, $\Delta \alpha=$7.2度で,
図中,白の実線は平均値を,破線は平均値$\pm$標準偏差を示したものである.
アルミニウム供試体の結果を見ると,波数と周波数は0.2MHzから2.5MHz程度の
帯域で明確な比例関係にあることがわかる.また,波数ベクトルの方向は,その帯域で0度付近に
集中し,$x$軸方向へ強く配向している.ただし,2MHz程度から$\alpha$の標準偏差が
次第に大きくなり,2.5MHzを超える頃には配向性が消失している.以上より,
アルミニウム供試体では2.5MHz程度までの進行波成分が観測されているが,
2MHz程度からノイズの影響が大きくなり,2.5MHz以上にはほとんど有意な信号成分
が含まれないことがわかる.
一方,図-\ref{fig:fig10}に示した花崗岩供試体の結果では,
波数と周波数の比例関係は認められるが,その周波数帯域は0.6MHzから2.0MHz程度と
アルミニウム供試体に比べて狭い.また,標準偏差も$k,\alpha$ともアルミニウムの
場合に比べて大きく,0.5MHzから高周波側では単調に増加する.
方向$\alpha$に関して言えば,いずれの周波数でも配向の程度がアルミニウム供試体と
比べて低く,2.5MHz近傍では標準偏差が約90度に達し,配向性が完全に消失する.
以上より,信号成分の含まれる周波数帯域$\Omega$は,アルミニウム供試体ではおよそ
0.5$\sim$2.5MHz, 花崗岩供試体では0.6$\sim$2.0MHzと判定できる.
これらの帯域や配向状況の差は,花崗岩供試体では高い周波数ほど鉱物粒による
散乱の影響を受け易いことに起因すると考えられる.
%
このような情報はフーリエ変換による波数-周波数スペクトルからは得ることができず,
局所波数ベクトルの確率分布を用いる一つの利点と言える.
なお,ここで言う帯域とは, 配向性を消失することなく$x$方向へ伝播する波動が
観測できる周波数範囲の目安で,一定値以上の波動振幅が得られる帯域という通常の
定義と異なることに注意が必要である.
%花崗岩のような不均質媒体では,計測波形やその周波数スペクトルは複雑な形を
%しており,振幅を基準にした議論は難しいことが多い.その点で,波数ベクトルの
%統計分布に基づく帯域の定義に従い,見るべき周波数範囲を定めることが
%より適切と考えられる.
\begin{figure}
\begin{center}
	\includegraphics[clip,scale=0.4]{Figs/pdf_kw_Al.eps}
	\caption{(a) 波数ベクトルの大きさ$k$と,(b)方向$\alpha$の分布(アルミニウム供試体).}
	\label{fig:fig9}
\end{center}
	\vspace{-10mm}
\end{figure}
\begin{figure}
\begin{center}
	\includegraphics[clip,scale=0.4]{Figs/pdf_kw_bar.eps}
	\caption{(a)波数ベクトルの大きさ$k$と,(b)方向$\alpha$の分布(花崗岩供試体).}
	\label{fig:fig10}
\end{center}
	\vspace{-10mm}
\end{figure}
%%%%%%%%%%%%%%%%%%%%%%%%%%
\subsection{到達時間関数$T_f$}
観測領域${\cal R}$の左端$(x=0)$から伝播した超音波が,
領域内の任意の点$\fat{x}\in {\cal R}$に
到達するために要する時間を$T_f(\fat{x},\omega)$とし,$T_f$を波数ベクトルから求めることを考える.
そこで,直線伝播距離が$a$となる$\cal R$内の点を,
\begin{equation}
	{\cal I}(a)={\cal R} \cup \left\{ x=a\right\}
	\label{eqn:def_I0}
\end{equation}
とし,ある$\fat{x}_0\in {\cal I}(0)$からの波動が,経路$\Gamma$を伝播して
$\fat{x}$に到達した場合について考える(図-\ref{fig:fig11}).
このとき,始点から終点の間で生じる位相の変化は
\begin{equation}
	\phi(\fat{x},\omega)-\phi(\fat{x}_0,\omega)=
	2\pi \int_{\Gamma} \fat{k}(\fat{s},\omega)\cdot d\fat{s}
	\label{eqn:int_phi}
\end{equation}
で与えられる.ここで,$d\fat{s}$は経路$\Gamma$の微小接ベクトルである.
波動伝播は位相が増加する方向に起きることから,$\Gamma$上では
\begin{equation}
	\fat{k}\cdot d\fat{s}\geq 0 \, {\rm on} \, \Gamma 
	\label{eqn:}
\end{equation}
が要請される.フェルマーの原理によれば,$\fat{x}_0$と$\fat{x}$を
結ぶ経路$\Gamma$のうち,実際に選ばれる経路は位相変化が最小となるものである.
ただし,$\fat{x}$に到達する 波動伝播の起点$\fat{x}_0$が, $\cal I$(0)上のどこかにあるのかは
前もってわからない.そこで,フェルマーの原理を適用するにあたり,$\Gamma$と$\fat{x}_0$が
式(\ref{eqn:int_phi})を最小にするように選ばれると考える.
すなわち,$\fat{x}$における位相は,
\begin{equation}
	\phi(\fat{x},\omega) - \phi(\fat{x}_0,\omega)
	=
	\min_{\left(\Gamma, \fat{x}_0\right)}
	\int_\Gamma 2\pi \fat{k}(\fat{s},\omega) \cdot d\fat{s}
	\label{eqn:Fermat}
\end{equation}
となると考える.これを周波数$\omega$で割れば,到達時間が次のように得られる.
\begin{equation}
	T_f(\fat{x},\omega) = \
	\frac{\phi(\fat{x},\omega)}{\omega}
	\label{eqn:def_Tf}
\end{equation}
実際には,観測波形は${\cal G}$でのみ与えられているため,経路$\Gamma$を
格子点を結ぶ折れ線$\tilde \Gamma$で経路を近似する(図-\ref{fig:fig11}).
\begin{equation}
	\fat{k}\cdot d\fat{s}
	=
	\left\{
	\begin{array}{c}
		\pm k_x \Delta x \\
		\pm k_y \Delta y
	\end{array}
	\right.
	\label{eqn:}
\end{equation}
さらに,各々の$\fat{x}$について経路$\Gamma$を直接特定するのでなく,
次のようにして,出発点から最小の位相変化で到達できる点を順次決定して位相分布を求める.
\subsection{到達時間の計算方法}
はじめに,$\fat{x}\in {\cal G}\cap {\cal I}(0)$を一つ選ぶ.
この点の位相を0とし,位相決定済み点としてリストに登録する.
次に,$\fat{x}$に隣接する格子点をリストアップし,これを近傍点と呼ぶ.
全ての近傍点には位相決定済みの隣接点が存在する.
そこで, 式(\ref{eqn:int_phi})を離散化した
\begin{equation}
	\phi(\fat{x}+\Delta \fat{s}) -\phi(\fat{x}_0) = 2\pi \fat{k} \cdot \Delta \fat{s}
	\label{eqn:inc_phase}
\end{equation}
を用い,近傍点の位相を計算して仮登録する.
ただし,式(\ref{eqn:inc_phase})において,$\fat{x}$は位相決定済みの点に,
$\fat{x}+\Delta \fat{s}$はその近傍点に取る.また,
$\Delta \fat{s}$は2つの点の隣接関係(相対位置)に応じて次のいずれかで与える.
\begin{equation}
	\Delta \fat{s} = (\pm \Delta x,\,0) \, or \, (0,\pm \Delta y)
	\label{eqn:} \end{equation}
波数ベクトル$\fat{k}$は,式(\ref{eqn:kx_FD})あるいは式(\ref{eqn:ky_FD})で計算した
ものを用いる.なお, $\fat{k}\cdot \Delta \fat{s}<0$となる方向にある近傍点の位相は未定の
ままとし,複数の位相決定済み点に隣接した近傍点には,それら隣接点から計算された位相の
中で最小のものを仮登録する.
この作業を全ての近傍点で行った後,仮登録された位相の中で最小のものだけを採用し,
その格子点を位相決定済みのリストに加える.
位相決定済み格子点のリストが更新された後は,近傍点のリストを再度作成し,位相の仮登録から
の手順を繰り返す.以上の作業を,新たに登録される近傍点がなくなるまで実行する.
これら一連の位相決定作業を,${\cal G}\cap {\cal I}(0)$に含まれる全ての点を開始点として
行えば,最終的には位相が決定された全ての点で,$\cal I$(0)から最小の位相変化で到達したときの
位相が登録されることになる.
ただし,この方法では$\cal G$の全ての点で位相が決定できる保証は無く,与えられた波数ベクトル
場によっては,位相が未決定の点が残される.しかしながら,これは位相決定方法の不備ではなく,
波数ベクトル場の性質によるものと言える.観測データにはノイズが含まれ波数ベクトルが
完全に正確には求まらないこと,また,波動伝播経路が観測面内に限定されず,本来は
3次元的であることを考えると,位相が未決定の点が残ることは不自然なことではない.
\begin{figure}
\begin{center}
	\includegraphics[clip,scale=0.8]{Figs/unwrap.eps}
	\caption{波数ベクトルの積分経路$\Gamma$とその折れ線近似$\tilde \Gamma$のイメージ.}
	\label{fig:fig11}
	\vspace{-5mm}
\end{center}
	\vspace{-12mm}
\end{figure}
\subsection{到達時間関数の計算例}
以上の方法で計算した到達時間関数$T(\fat{x},\omega)$の一例として,
周波数$\frac{\omega}{2\pi}=$1MHzの結果を図-\ref{fig:fig12}に示す. 
アルミニウム供試体に対する結果では,到達時間の分布は$y$方向にほぼ一定となり,
期待通り1次元的な波動伝播挙動が再現されている.
なお,到達時間が未定のまま残された格子点では,便宜上$T_f=-1$として表示している.
そのため到達時間未定の箇所は,図-\ref{fig:fig12}において孤立した紺色のセルとして
示されている.アルミニウム供試体に対する結果で,領域中央部のピクセルが欠けたように
見える点はこのことによる.アルミニウム供試体でこのような点が生じた理由は,
供試体表面の性状が悪く,LDVでの波形計測が十分な強度の反射光を得ることが
出来なかったためと考えられる.
図-\ref{fig:fig12}(b)の花崗岩供試体に対する結果でも,$x>0$の方向へ到達時間が遅れる
傾向がはっきりしている.一方で,$y$軸方向に到達時間は一様でなく,ゆらぎがある.
また,到達時間未定の箇所もアルミニウム供試体の場合よりも多い.
これは主として,強い散乱により波数ベクトルの方向が3次元的に変動し,
供試体表面上の経路からは超音波が到達できない点が生じているためと考えられる.
\begin{figure}
\begin{center}
	\includegraphics[clip,scale=0.4]{Figs/phi_R.eps}
	\caption{到達時間$T_f(\fat{x},\omega)$の空間分布. 
	(a)アルミニウム供試体,(b)花崗岩供試体.いずれも周波数1.0MHzの結果.}
	\label{fig:fig12}
\end{center}
	\vspace{-5mm}
\end{figure}
\subsection{到達時間の確率分布}
到達時間を頻度分布として整理した結果を図-\ref{fig:fig13}に示す.
この図は,周波数帯域$\Omega$を
\[
	\Omega = \left[ 0.6, 2.0\right],\,\, [{\rm MHz}]
\]
として計算した頻度分布:
\begin{equation}
	P_{{\cal G \cap I}(x)\times \Omega }(T_f; \Delta T), \ \ \Delta T=0.1[\mu {\rm s}]
	\label{eqn:P_GIO}
\end{equation}
を示したものである.縦軸は伝播距離$x$を,横軸は到達時間$T_f$を表し,
$(T_f,x)$における頻度分布をカラーマップとして表示している.
$T_f-x$平面上の各点における頻度を評価する際のサンプルには,
位置$x$にある格子点${\cal G \cap I}(x)$で,帯域$\Omega$内の全ての
周波数に対して計算した到達時間を用いている.
図\ref{fig:fig13}-(a)に示したアルミニウム供試体の場合,
到達時間の分布は右上がりの直線近傍の狭い範囲に集中し,
$T_f$の分散は伝播距離$x$によってほとんど変化しない.
この直線の傾きは伝播速度$c$を表すので,$T_f$の平均$\bar{T}_f(x)$を
直線近似して傾きを求めると,
\begin{equation}
	c=2.975[{\rm km/s}], \ \ (アルミニウム)
	\label{eqn:c_Al_ave}
\end{equation}
となる.この結果は,波数−周波数スペクトルのピークから求めた速度と
0.012[km/s]の差しかなく,両者はよく一致する.このことは,
本研究で提案した到達時間関数が合理的なもので,その数値的な評価も
問題なく行われていることを意味する.
一方,図-\ref{fig:fig13}(b)に示す花崗岩供試体に対する結果では,
伝播距離$x$の増加にともない,到達時間$T_f$の分散が大きくなっている.
このことを強調するため,図-\ref{fig:fig13}(b)には,
$T_f$の平均$\bar{T}_f$を白の実線で,標準偏差を$\delta T_f$として
$\bar{T}_f\pm \delta T_f$にあたる位置を白の破線でそれぞれ示している.
なお,アルミニウム供試体の場合は,これらの曲線が近接し,図上で互いに区別がつかないため,
図-\ref{fig:fig13}(a)には平均と標準偏差を示すカーブは示していない.
\begin{figure}
\begin{center}
	\includegraphics[clip,scale=0.45]{Figs/tof_hist.eps}
	\caption{到達時間$T_f$の正規化した頻度分布. (a)アルミニウム供試体,(b)花崗岩供試体.}
	\label{fig:fig13}
\end{center}
	\vspace{-10mm}
\end{figure}
そこで,平均到達時間と$\bar{T}_f$と位置$x$の関係を,
図-\ref{fig:fig14}に示す.このグラフには,アルミニウムと花崗岩供試体の
結果を同時に示している.
これまでは,花崗岩供試体として図-\ref{fig:fig1}に示したブロック状の
ものに対する結果だけを示してきた.この図ではこれに加えて,同じ石切り場
で採取した万成花崗岩を,円柱状に加工した供試体に対して得られた結果を
併せて示している.2つの花崗岩供試体を区別する場合,前者をブロック供試体,
後者を円柱供試体と呼ぶ.なお,円柱供試体の直径は60mm,高さは50mmで,
超音波計測は円柱上部のフラットな面で行った.送受信点の配置等,計測条件は
ブロック供試体の場合と同じである.
図-\ref{fig:fig14}のグラフによれば,平均到達時間$\bar T_f$の大小関係は概ね
\begin{center}
	ブロック供試体 $<$ 円柱供試体 $<$ アルミニウム
\end{center}
となる.ただし,円柱供試体では$x$が大きなところで次第に$\bar T_f$が遅れ,
$x=17.5\mu$s付近ではアルミニウム供試体と同程度になっている.
これらの関係を図-\ref{fig:fig14}のグラフ全体を直線近似し,
花崗岩供試体における平均的な伝播速度$c$を求めると,
\begin{eqnarray}
	c=3.272[{\rm km/s}], &&(花崗岩ブロック)\\
	c=2.983[{\rm km/s}], &&(花崗岩円柱)
	\label{eqn:c_granite_ave}
\end{eqnarray}
となり,同じ花崗岩でも平均伝播速度に差があることが示される.
ただし,花崗岩供試体では超音波の距離減衰が大きく側面や底面からの反射波は観測されないため,
この差は供試体形状によるものではなく,鉱物分布の不均一性に起因すると考えられる.
また,花崗岩にはマイクロクラックの配向に起因した音響異方性があること
はよく知られている\cite{Takagi, Kudo1, Kudo2}.
ここで求めた,2つの花崗岩供試体に対する音速値の差は,既往の文献\cite{Sano1,Sano2}
に報告されている異方性によるS波速度の伝播方向による変動と同程度で,顕著に大きな差では無い.
なお,花崗岩ブロックの場合,波数-周波数スペクトルのピークと,平均到達時間から求めた
超音波伝播速度に乖離がある.これは,波数-周波数スペクトルによる方法が,大きな振幅を
持つ波形成分の伝播挙動に影響を受けやすいのに対し,平均到達時間は位相だけに
着目して速度を算出され,振幅のばらつきには影響されにくいためと考えられる.
従って,不均質材の場合2つの方法で求めた伝播速度が高精度に一致すべき必要性は無い.
さらに,不均質材では,到達時間には必ずばらつきが出るため,距離と到達時間に
一対一の関係を定めることが厳密には出来ず,走時曲線と速度の定義に任意性がある.
このような視点からすると,速度よりも,到達時間の分布自体がより本質的なものであると
言うことができる.
\begin{figure}
\begin{center}
	\includegraphics[clip,scale=0.45]{Figs/tof1d.eps}
	\caption{到達時間の平均$\bar T_f$と伝播距離$x$の関係. }
	\label{fig:fig14}
\end{center}
	\vspace{-8mm}
\end{figure}
\subsection{到達時間のゆらぎ}
図-\ref{fig:fig15}に,到達時間の標準偏差$\delta T_f$と距離$x$の関係を示す.
図-\ref{fig:fig15}(a)は横軸を伝播距離$x$,縦軸を到達時間の標準偏差$\delta T_f$として
プロットしたもので,3つの供試体に対する結果を比較している.花崗岩供試体の場合,
伝播距離$x$が大きくなるに伴い,到達時間の標準偏差も単調に増加している.
一方,アルミニウム供試体では,伝播時間に応じた$\delta T_f$の増減はほとんどなく,
およそ0.1$\mu$s以下の範囲ににとどまっている.
2つの花崗岩供試体で比較すると,$x=15$mm辺りから若干の乖離が生じ,
円柱供試体の場合に$\delta T_f$がより大きくなっている.
この距離では,平均到達時間$T_f$がブロック供試体と比較して遅れが生じ,
速度が低下する領域に相当している.
図-\ref{fig:fig15}(b)は,同じ結果を縦軸を平均到達時間$\bar T_f$で正規化した
標準偏差$\frac{\delta T_f}{\bar{T}_f}$としてグラフ化したもので,
全ての供試体で正規化した標準偏差が単調に減少することが示されている.
これは,到達時間のゆらぎ(標準偏差)が到達時間そのものと同程度あるいはそれ以上の
割合で増加するわけではないことを示している.また,花崗岩供試体間で比較すると,
到達時間差が考慮された結果,距離による変化挙動が互いによく似た曲線となっている.
これは,2つの供試体の間でランダム不均質性に共通する面があることを示していると
理解することができる.\\
\hspace{\parindent}
次に,到達時間$T_f$に対応する伝播距離$x$の標準偏差$\delta x$を
図-\ref{fig:fig16}に示す.ここでも,(a)は標準偏差$\delta x$を,
(b)は伝播距離の平均$\bar{x}$で正規化した標準偏差$\frac{\delta x}{\bar{x}}$を示す.
$\delta T_f$が,位置$x$で観測した到達時間のゆらぎを表すのに対し,
$\delta x$は時刻$T_f$で観測を行ったときの伝播距離のゆらぎと理解することができる.
アルミニウム供試体の場合,時間によって位置のゆらぎはあまり変化せず,
$\delta x$は0.25mm程度となっている.アルミニウム供試体の位相速度$c$は
約3.0[km/s]だから,$\delta x/c$はおよそ$0.8\mu$sとなり,
これは図-\ref{fig:fig15}(a)にある結果と整合する.
花崗岩供試体に対する結果では,位置のゆらぎである$\delta x$が時刻$T_f$に応じて増加し,
2つの供試体で比べると$\delta x$はブロック供試体の方が一貫して大きくなっている.
これは,ブロック供試体の音速が円柱供試体の音速よりも大きいためであり,実際,
平均距離$\bar{x}$で正規化すると両者の差は目立たなくなる.\\
\hspace{\parindent}
最後に,正規化した標準偏差$\frac{\delta T_f}{\bar{T}_f}$を,両対数軸上にプロットする.
その結果は,図-\ref{fig:fig17}のようであり,花崗岩供試体に対する結果では,伝播距離$x$と
正規化した標準偏差がほぼ直線関係にあることが分かる.
そこで,$\frac{\delta T_f}{\bar{T}_f}$を,べき関数
\begin{equation}
	\frac{\delta T_f }{\bar{T}_f} \simeq \frac{K}{x^m}
	\label{eqn:fit_power}
\end{equation}
で近似し,両対数グラフの傾き$-m$を最小2乗法で求めると,
それぞれの供試体で次のような評価が得られる.
\begin{eqnarray}
	m &=& 1.073 (アルミニウム供試体) \\
	m &=& 0.537 (花崗岩ブロック)\\
	m &=& 0.497 (花崗岩円柱)\\
\end{eqnarray}
これより,万成花崗岩供試体では,正規化した到達時間の標準偏差がおよそ$x^{-1/2}$に比例することが分かる.
さらに,平均到達時間$\bar T_f$と伝播距離$x$の間に比例関係
\begin{equation}
	x=\tilde c \bar T_f
	\label{linfit_x}
\end{equation}
が仮定できるならば,式(\ref{eqn:fit_power})は
\begin{equation}
	\delta T_f \simeq \tilde K \sqrt{ \bar{T}_f}, \  \ 
	\left( \tilde K=\frac{K}{\sqrt{\tilde c}}\right)
	\label{eqn:plaw} 
\end{equation}
となり,今回の供試体に関しては到達時間の標準偏差が$\sqrt{\bar{T}_f}$に比例するとの結果が得られる.
このことは,不確実性の指標が一つのパラメータ($\tilde K$)で表現できる可能性があることを
示している.
%このことは,$\bar{T}_f(x)$と$\delta T_f(x)$の関係をモデル化しておけば,
%$\bar{T}_f(x)$と$m$を実測データから決めることで,
%$(x,T_f)$の同時確率分布を定めることができ,その結果から音速や到達位置の
%平均$\bar{x}$や不確実性$\delta x$の特徴を理解できることによる.
このような法則に普遍性があるか否かは,今後実測値との比較で検証を行う必要があるが,
べき則による不確実性の発展則はシンプルで理解しやすく,ランダム不均質
媒体中の波動伝播モデルを構築する上で,有用な考え方になると思われる.
\begin{figure}
\begin{center}
	\includegraphics[clip,scale=0.425]{Figs/sigma_t.eps}
	\vspace{-2mm}
	\caption{到達時間$T_f$の標準偏差$\delta T_f$と伝播距離の関係. }
	\label{fig:fig15}
\end{center}
	\vspace{-5mm}
\end{figure}
\begin{figure}
\begin{center}
	\includegraphics[clip,scale=0.425]{Figs/sigma_y.eps}
	\vspace{-2mm}
	\caption{到達距離$\bar{x}$の標準偏差$\delta \bar{x}$と伝播時間$T_f$の関係. }
	\label{fig:fig16}
\end{center}
	\vspace{-5mm}
\end{figure}
\begin{figure}
\begin{center}
	\includegraphics[clip,scale=0.425]{Figs/tsig_log.eps}
	\vspace{-2mm}
	\caption{
		両対数軸上グラフとしてプロットした正規化された到達時間の標準偏差
			$\frac{\delta T_f}{\bar{T}_f}$. 
		}
	\label{fig:fig17}
\end{center}
	\vspace{-8mm}
\end{figure}

\section{まとめ}
本研究では,不均質材における波動伝播特性を理解することを目的に,
万成花崗岩供試体を用いた超音波計測の結果から,表面波の到達時間と位置の関係を調べた.
到達時間の評価には,計測波形のフーリエ位相から求めた波数ベクトル場を経路積分する
方法を提案し,受信点位置と周波数に関する到達時間のアンサンブルから頻度分布を
得ることで,波動の到達位置と到達時間の平均と偏差を評価した.
この方法では,波形の立ち上がりやピークを読むといった作業は必要なく,
全ての計算は波数や位相等の明確に定義された量で行うことができることが一つの
利点となっている.以上の方法で,到達時間と到達位置の同時確率分布を推定したところ,
均質材であるアルミニウム供試体では,到達時間のゆらぎが伝播距離にほとんど依存しない
ことが分かった.一方,万成花崗岩供試体では,伝播距離が大きくなるに従い,多重散乱の影響によって
到達時刻のゆらぎが単調に増加することが示された.ただし,平均到達時間で正規化した標準偏差
によって到達時間のゆらぎを見ると,その結果は距離に応じて単調に減少し,
減少割合が概ね距離の$-1/2$乗に比例することが見出された.従って,平均到達時間が伝播距離の
1次式で近似できると仮定すれば,今回の花崗岩供試体では到達時間のゆらぎが到達時間の1/2乗,あるいは,
距離の1/2乗に比例して増加すると結論することができる.
今後は,到達時間と伝播距離に関する同時確率分布の詳細な分布形状を調べること,
分布を特徴付ける平均や標準偏差の物理的な起源について明らかにすることが課題となる.
特に,後者は不均質材における探査や,岩石の風化や損傷を調べる非破壊検査への応用の上で重要な課題である.
また,ここでは到達時間に関する議論だけを行ったが,到達時間のゆらぎと散乱減衰の
関係付けや,それに基づく減衰モデリングを行うことも,ランダム不均質材中の
波動伝播挙動をより深く理解する上での重要なテーマになると考えられる.
%%%%%%%%%%%%%%%%%%%%%%%%%%%%%%%%%%%%%%%%%%%%%%%%%%%%%%%%%%%%%%%%%%%%%%%%%%%%%
\\

{\gt 謝辞:}
本実験に用いた花崗岩供試体は浮田石材店代表浮田隆司氏に提供頂いた.
また本研究の推進には,科学研究費補助金(基盤研究©課題番号\#18K04334)の補助を受けた.
併せて謝意を表す.
%%%%%%%%%%%%%%%%%%%%%%%%%%%%%%%%%%%%%%%%%%%%%%%%%%%%%%%%%%%%%%%%%%%%%%%%%%%%%
%\newpage
%\lastpagecontrol[2cm]{13.7cm}
\begin{thebibliography}{99}
\begin{spacing}{1.175}
\bibitem{RockPhys}
	ゲガーン, Y., パルシアウスカス, V.: 
	岩石物性入門, シュプリンガー・ジャパン, 2008. 
\bibitem{Sato}
	Sato, H., Fehler, M.C., and Maeda, T.:Seismic wave propagation and scattering in the heterogeneous earth, 
	Springer, 2012.
\bibitem{Borcea}
	Borcea, L.:Imaging with waves in random media, {\it Notices of the American Mathematical Society}, Vol.66, No.11, 
	pp.1800-1812, 2019.
\bibitem{Thompson}
	Thompson, B. R.:Elastic-wave propagation in random polycrystals: 
	fundamentals and application to nondestructive evaluation, 
	in Imaging of Complex Media with Acoustic and Seismic Waves, Topics in Applied Physics 84, Springer, pp.233-256, 2002.
\bibitem{Etgen}
	Etgen, J., Gray, S. H., and Zhang, Y.:An overview of depth imaging in exploration geophysics, 
	{\it Geophysics}, Vol.74, No.6, pp.WCA5-WCA17, 2009.
\bibitem{Schmitz}
	Schmitz, V.:Nondestructive acoustic imaging techniques, in Imaging of Complex Media with Acoustic and Seismic Waves, 
	Topics in Applied Physics 84, Springer, pp.167-189, 2002.
\bibitem{Langenberg}
	Shlivinski, A. and Langenberg, K. J.:Defect imaging with elastic waves in inhomogeneous-anisotropic materials with composite geometries, {\it Ultrasonics}, Vol.46, pp.89-104, 2007.
\bibitem{Bleistein}
	Bleistein, N., Cohen, J. K. and Stockwell, Jr. J. W.:Mathematics of multidimensional seismic imaging, migration, and inversion, Springer, 2000, pp.220.
\bibitem{Yu}
	Yu, L., Thompson, R. B., Mrgentan, F. J., and Wang, Y.:A Monte-Carlo model for microstructure-induced ultrasonic signal fluctuations in titanium alloy inspections,
	{\it Review of Progress in Quantitative Nondestructive Evaluation}, Vol.23, pp.1170-1177, 2004.
\bibitem{Li}
	Li, A., Roberts, R., Mrgentan, F. J., and Thompson, R. B.:A 2-D numerical simulation study of microstructure-induced  ultrasonic beam distortions,
	{\it Review of Progress in Quantitative Nondestructive Evaluation}, Vol.23, pp.1178-1186, 2004.
\bibitem{NishizawaI}
	西澤修:岩石中の地震波伝播I:不均質媒体のモデル化と弾性波速度, 地学雑誌, 第114巻, 第6号,  pp.921-948,  2005.
\bibitem{Muller}
	Muller, G., Roth, M. and Korn, M.:Seismic-wave traveltimes in random media,
	{\it Geophys. J. Int.}, Vol.110, pp.29-41, 1992. 
\bibitem{Korn}
	Korn, M.:Seismic waves in random media, 
	{\it Journal of Applied Geophysics}, Vol.29, pp.247-269, 1993.
\bibitem{Spetzler2001}
	Spetzler, J. and Snieder, R.:The effect of small-scale heterogeneity on the arrival time of waves, 
	{\it Geophys. J. Int.}, Vol.145, pp.786-796, 2001. 
\bibitem{Spetzler}
	Spetzler, J., Sivaji, C., Nishizawa, O., and Fukushima, Y.:A test of ray theory and scattering theory based on
	a laboratory experiment using ultrasonic waves and numerical 
\lastpagecontrol[0.0cm]{9.0cm}
\newpage
	simulation by finite-difference method, 
	{\it Geophys. J. Int.}, Vol.148, pp.165-178, 2002. 
\bibitem{Nishizawa1996}
	西澤修, 雷興林, 佐藤隆司:不均質媒体での地震波伝モデル実験-レーザードップラー速度計を用いた波動計測-
	, 地震調査所月報, 第47巻, 第4号, pp.209-222, 1996.
\bibitem{Nishizawa2001}
	西澤修, 雷興林, チャダラム シバジ:不均質媒質での地震波伝播モデル実験, 
	地震, 第54巻, pp.171-183, 2001.
\bibitem{Sivaji}
	Sivaji, C., Nishizawa, O., Kitagawa, G., and Fukushima, Y.:A physical-model study of the statistics of seismic waveform fluctuation in random heterogeneous media, 
	{\it Geophys. J. Int.}, Vol.148, pp.575-595, 2002. 
\bibitem{Fukushima}
	Fukushima, Y., Nishizawa, O., Sato, H., and Ohtake, M.:Laboratory study on scattering characteristics of shear waves 
	in rock samples, {\it Bulltine of Seismological Society of America}, Vol.93, No.1, pp.253-263, 2003.
%\bibitem{Okubo2012}
%	大久保 慎人, 雑賀 敦, 鈴木 貞臣, 中島 唯貴: 地震動観測による地震波速度と岩石物性試験による弾性波速度の関係
%	-段発発破波形の相関による地震波速度構造推定-: 地震, 第65巻, pp.21-30, 2012.
\bibitem{Kudo1}
	工藤洋三, 橋本堅一, 佐野修, 中川浩二:花崗岩の力学的異方性と岩石組織欠陥の分布,
	土木学会論文集, 第370号/III-5, pp.189-197, 1986.
\bibitem{Kudo2}
	工藤洋三, 橋本堅一, 佐野修, 中川浩二:瀬戸内地方の採石場における花崗岩石の異方性, 
	土木学会論文集, 第382号/III-7, pp.45-53, 1987.
\bibitem{Sano1}
	佐野修, 工藤洋三, 河嶋智, 水田義明:異方性体としての花崗岩の弾性率に関する実験的研究, 
	材料, 第37巻, 第418号, pp.84-90, 1987.
\bibitem{Sano2}
	佐野修, 民部雅史, 平野亮, 工藤洋三, 水田義明:弾性的対称性未知の岩石の弾性定数決定に関する研究, 
	材料, 第40巻, 第449号,pp.96-102, 1990.
\bibitem{Takagi}
	高木秀雄, 三輪成徳, 横溝佳侑, 西嶋圭, 円城寺守, 水野崇, 天野健治:土岐花崗岩中の石英に発達するマイクロクラックの三次元
	方位分布による古応力場の復元と生成環境, 地質学雑誌, 第114巻, 第7号, pp.321-335, 2008.
%\bibitem{Rwk_textbook}
%	Klaffter, J., and Sololov,I.M.,秋元 琢磨(訳): ランダムウォークはじめの一歩,共立出版, 2018.
\end{spacing}
\end{thebibliography}
\begin{flushright}
	\small
	\bf{ (Received July 24, 2020)\\
	(Accepted???????????)}
\end{flushright}
\end{document}

%\lastpagesettings
%\begin{minipage}[c]{13.7cm}
%\end{minipage}
%\lastpagecontrol[0cm]{13.7cm}
%\begin{multicols}{1}
%-------------------------------------------------
%-------------------------------------------------
%\end{multicols}

