%%#!platex
%
% Example of Japanese Paper of JSCE
% for LaTeX2e users
%
% revised on 4/25/2014
%
%%%%%%%%%%%%%%%%%%%%%%%%%%%%%%%%%%%%%%%%%%%
%
% もし jis フォントメトリックを使う場合は,以下をアンコメントしてください.
% \DeclareFontShape{JY1}{mc}{m}{n}{<-> s * jis}{}
% \DeclareFontShape{JY1}{gt}{m}{n}{<-> s * jisg}{}
%
\documentclass{jsce}
%
\usepackage{epic,eepic,eepicsup}
\usepackage{graphicx,multicol}
\usepackage{setspace}
%  amsを使う方は以下をアンコメントしてください.
%\usepackage{amssymb,amsmath}
% 英語はサポートしているかどうか不明
% \inenglish
% 学会サンプルに times とあるので指定しておきます
\usepackage{times}
%
\finalversion
\pagestyle{empty}
%
\title{
	不飽和多孔質体中の物質拡散に関する\\数値解析的研究
}
%
\endtitle{
	%Numerical Study on the Material Diffusion in Unsaturated Porous Media 
	%Numerical Study on the Mass Diffusion in Unsaturated Porous Media 
	A NUMERICAL STUDY ON THE MASS DIFFUSION \\IN UNSATURATED POROUS MEDIA 
}
%
% emailアドレスのフォントをタイプライター体にしたい方は次行をアンコメント
% \emailstyle{\ttfamily}
% emailアドレスを公開される方は,
%% \thanks{○○○○○○\email{your_name@foo.ac.jp}}のようにしてください.
%
\author{木本 和志\thanks{正会員 博士(工学) 岡山大学 環境生命科学研究科
(〒700-8530 岡山県岡山市北区3丁目1番地1号)
\email{kimoto@cc.okayama-u.ac.jp}}
}
%
\endauthor{Kazushi KIMOTO}
%
\abstract{
\small
本研究は,不飽和多孔質体における物質拡散挙動を調べることを目的に,ランダムウォークをベースとした
数値拡散解析手法の開発を行ったものである.ここで提案する方法では,
固相粒子を充填して作成した周期多孔質構造のユニットセルにおいて,
モンテカルロ法を用いて界面エネルギーが極小値をとるように間隙水を配置する.
作成した数値不飽和多孔質媒体は,液相を拡散媒体としたランダムウォークによる
物質拡散解析に用いる.ランダムウォーク・シミュレーションで得られた
拡散粒子変位の時刻歴から,拡散係数と平均2乗変位に関する時間のべき指数を評価し,
多孔質媒体が示すマクロな拡散特性を定量化する.本稿では,以上の方法で2次元拡散解析を行い,
飽和度や,固相粒子の形状と粒径分布が不飽和多孔質体のマクロ拡散に与える影響を調べた.
その結果,主として粒径分布が拡散係数の大きさに,固相粒子の充填構造が異常拡散の程度に
寄与することが明らかとなった.
}
%
\keywords{unsaturated porous media, Monte Carlo method, 
pore water distribution, random walk, anomalous diffusion}
%
\endabstract{% Yes blank line
\normalsize
This study develops a numerical technique for the analysis of mass diffusion in unsaturated porous media. 
In the present approach, a periodic porous solid model is generated numerically by packing arbitrary 
shaped particles into the unit cell. 
%
The pore water of predetermined amount is then distributed over 
the pore space so that the total interfacial free energy is minimized. Finally, a random walk particle 
tracking is implemented on the rectangular grid generated uniformly over the fluid phase domain. 
To characterize the macroscopic diffusion, an overall diffusion constant and the abnormality of the 
macro diffusion are evaluated quantitatively based on the temporal evolution of random walkers' mean-square 
displacement. With the numerical methods, the effects of such parameters as the degree of saturation, 
solid phase packing structure, and the solid particle size distribution on the macro diffusion characteristics 
are investigated. It was found, as the results, that the origin of the anomalous diffusion is the 
tortuosity of the diffusion paths,
%randomness of the packing structure, 
while the grain size distribution has more to do with the magnitude of the macro diffusion constant.  
}
%
% \titlepagecontrol{1}
%
%\receivedate{2019.7.19}
% \receivedate{January 15, 1991}
%
% \def\theenumi{\alph{enumi}}  % もし enumerate 最初の箇条を (a) と
% \def\labelenumi{(\theenumi)} % したい場合・・・
%
\begin{document}
\maketitle
%%%%%%%%%%%%%%%%%%%%%%%%%%%%%%%%%%%%%%%%%%%%%%%%%%%%%%%%%%%%%%%%%%%%%%%%%%%%%
\section{はじめに}
	建設分野において弾性波検査の対象となる材料の多くは,形状や物性が不規則かつ不均一なランダム媒体である.
例えば,コンクリートは,粒径や形状が異なる骨材と気泡がランダムに分布した非均質媒体である.
また,岩盤や岩石も断層や節理系から,岩石を構成する鉱物粒や介在物,マイクロクラックに至るまで,
各種の空間スケールで多様な非均質性を有する\cite{RockPhys}.
このような非均質ランダム媒体において,弾性波は不均質部との相互作用によって散乱や屈折を起こし,
複雑な伝播挙動を示す.そのため,地震探査や岩石コア,多結晶質材の弾性波検査やイメージング
には,緻密な金属材料のような均質材に対する非破壊検査には無い困難が伴う\cite{Sato,Borcea,Thompson}.
特に,弾性波が強い多重散乱を起こしながら媒体を伝播する場合,著しい減衰や波形の変化のために,
計測で得られた波形から有用な情報を取り出すことが,一般に難しい.このような困難を克服し,
ランダム不均質媒体に対する信頼性や精度の高い弾性波検査技術を開発するには,多重散乱効果を
考慮した波動伝播モデルの構築が必要となる.

物理探査や非破壊検査において反射源位置を特定する際,弾性波速度が既知である必要がある\cite{Etgen, Schmitz}.
この理由から,種々の弾性波伝播特性の中でも伝播速度は重要と言え,このことは不均質媒体
でも,少なくともバックグラウンドの弾性波速度が必要となる点では同様である\cite{Langenberg, Bleistein}.
ただし,ランダム媒体の場合,媒体の物性値が場所によって異なるため,計測点毎に弾性波の到達時間と,
そこから見積もられる弾性波速度には必然的にばらつきが生じる.弾性波速度のばらつきは反射源位置の同定精度と
不確実性に影響するため,ランダム媒体に関しては弾性波速度の平均値だけでなくばらつきも重要な情報となる.
また,弾性波速度のばらつきは媒体の不均質性を反映したものであることから,ランダムな不均質性を弾性波計測
データから調べる目的においては,速度のばらつきを含め,弾性波速度や到達時間が従う統計分布自体が
興味の対象となる\cite{Yu,Li}.

ランダム不均質媒体における弾性波速度のばらつきを調べるために,これまで,種々の理論,数値解析および実験的
研究が行われてきた\cite{NishizawaI}.例えば,理論および数値解析的な研究には,波線理論や1次散乱理論を用いて
伝播時間解析を行ったもの\cite{Muller, Korn, Spetzler2001}や,差分法モデルでランダム媒体中を伝播する波動の
解析を行いその結果を理論解析と比較したもの\cite{Spetzler}などがある.一方,実験的な研究には,
岩石試料を透過する超音波をレーザードップラー振動計で計測し,弾性波速度のばらつきと鉱物粒径との関係を調べたもの
\cite{Nishizawa1996,Nishizawa2001}や,個々の計測波形と平均波形の乖離やP-S波間でのエネルギー分配の挙動を
不均質性スケールとの関係で調べたものなど\cite{Sivaji,Fukushima}がある.
これら実験的な研究の成果は,弾性波計測結果から不均質性のスケールや強度を推定する上で重要なものと言える.
一方で,伝播時間や伝播速度の揺らぎと,伝播距離や方向の関係は実験的には調べられておらず,
例えば波線理論や散乱理論による予測と一致するかどうかはこれまで明らかにされていない.

以上を踏まえ本研究では,弾性波伝播時間のばらつきが伝播距離に応じてどのように変化するかを明らかにすることを目的に,
超音波計測を実施する.実験では,典型的なランダム不均質媒体である花崗岩を供試体として用い,
圧電トランスデューサで励起した表面波の振動を,レーザードップラー振動計(LDV)で多点計測する.
LDVを用いて表面波を対象とした計測を行う理由は,試料表面の超音波振動を高い時空間解像度と広い周波数帯域で
観測することにより,波動場の伝播状況を精確に捉えることを意図したものである.
超音波の送信には,接触型の線集束トランスデューサを用い試料内部に円筒波を励起する.
これにより,強い超音波を送信できるだけでなく,入射方向と伝播距離を明確に定義することが可能となる.
一連の計測で得られた波形は,周波数領域において解析し,フェルマーの原理に基づき各観測点における到達時間を求める.
このようにして得られた到達時間のアンサンブルから,到達時間の確率分布を,伝播距離の関数として求める.
最後に,到達時間の平均と標準偏差を評価し,到達時間のばらつきが距離に応じてどのような法則に従い変化するかを
明らかにする.

以下では,はじめに超音波計測の方法について述べる.次に,計測で得られた波形データから表面振動の様子を可視化し,
どのような波動場が供試体表面に形成されているかを示す.続いて,各観測点と周波数における到達時間を求める波形解析
方法を示し,計測波形から求めた到達時間の空間分布を示す.最後に,到達時間の確率分布とその平均,標準偏差を
伝播距離の関数として求めた結果を示し,到達時間の不確実性が空間的にどのように発展するかを考察する.

	\vspace{-2mm}
\section{2次元拡散問題の設定}
	%\vspace{-4mm}
	%\section{超音波計測実験}
\subsection{実験供試体}
図-\ref{fig:fig1}に実験に用いた花崗岩供試体を示す.
この試験片は岡山県万成地域の採石場で採取した万成花崗岩をブロック状に切断加工したものである.
試験片表面には,目視で認められるような欠けや割れ,風化はない.
主要鉱物は,カリ長石,ナトリウム長石,石英および雲母の四種類で,
特徴的な桃色の色合いをした箇所がカリ長石である。
試験片のサイズはは長さ$L=178$mm, 幅$W=56$mm,厚さ$H=30$mmで,
図-\ref{fig:fig1}のような$xyz$座標系において$x$方向へ伝播する
超音波を計測する.全ての計測は,試験片の上面$(z=0)$mmにおいて行う.
また,均質材における波動伝播挙動との比較を行うため,同様な
超音波計測は,アルミニウムブロックの供試体でも行う.
アルミニウム供試体のサイズは,長さ200mm,幅150mm,厚さ50mmの
直方体で,音響異方性がほとんど無いことを予め確認している。
\begin{figure}
\begin{center}
\includegraphics[clip,scale=0.5]{Figs/samples.eps}
\caption{
	超音波計測に用いた花崗岩供試体.
}
\label{fig:fig1}
\end{center}
\end{figure}
\subsection{超音波探触子(送信子)}
表面波の励起は,供試体表面に接触させた圧電超音波探触子で行う.
送信に用いた超音波探触子の外観は図-\ref{fig:fig3}のようである。
圧電素子を収納した筐体部分と,圧電素子がマウントされた
ウェッジ(シュー)部分が示されている。
圧電素子は曲率半径26.1mm,投影面積が25mm×40mmの瓦状で、
共振周波数は2MHzのものを用いている。
圧電素子は同じ曲率半径を持つウェッジ上縁部に接着されており、
ウェッジ内部を伝播した縦波が先端部に集束するように設計されている。
ウェッジ先端部を試験片に接触させて用いれば、
試験片内部を円筒状に広がる弾性波が,ウェッジ先端部から励起される。
ウェッジの先端部は幅と長さが1×50mmである。
このような線集束型の探触子を用いることで,入射位置と伝播方向を
定義して制御することができる
また,試験片表面から強い半円筒波状の超音波を励起することで,
点波源から半球状の球面波を励起した場合に比べ,幾何減衰の影響を
小さくすることができる.
\begin{figure}[h]
\begin{center}
\includegraphics[clip,scale=1.0]{Figs/fig3.eps}
\caption{
	超音波の送信に用いた線集束探触子の外観.(a)正面,(b)側面から見た様子.
}
\label{fig:fig2}
\end{center}
\end{figure}
\subsection{超音波計測装置の構成}
図-\ref{fig:fig3}に,超音波計測装置の構成を示す.
計測装置は,3軸ステージ,レーザードップラー振動計(LDV),オシロスコープ,および
高周波スクウェア−ウェーブパルサーで構成されている.
試験片は,位置を精確に調整するために,水平2軸,回転1軸の3軸ステージ上に固定する.
送信に用いる探触子は,試験片表面に接触させて固定し,
400Vの矩形電圧パルスを,スクウェア−ウェーブパルサで印加して駆動する.
受信にはLDVを用い,受信波形はオシロスコープへ転送して4,096回平均化した後,
デジタル波形としてPCに収録する.なお,サンプリング周波数は15MHz,
計測時間範囲は200$\mu$秒とした.送信波形は2MHz程度のパルス状の波形だが,
岩石供試体では低い周波数成分が主として透過する.さらに,多重散乱により
振動の継続時間が送信パルス幅より長くなることから,サンプリングレートを
若干低めにし,振動が十分に小さくなる時間まで計測時間を長めにとるように条件を
設定した。
\begin{figure}[t]
\begin{center}
\includegraphics[clip,scale=0.5]{Figs/ut_setup.eps}
\caption{
	超音波計測装置の構成.
}
\label{fig:fig3}
\end{center}
\end{figure}
\subsection{送受信位置}
図-\ref{fig:fig4}に,送信および受信領域の配置を示す.
図中の${\cal S}$は送信位置,すなわち,線集束探触子のウェッジ先端部が接触する位置を表しており,
この部分で試験片に鉛直動が加えられる.${\cal R}$はLDVでスキャンする波形観測領域を示す.
$\cal R$は20mm$\times$30mmの矩形領域にとり,計測ピッチは$x$,$y$方向とも0.5mmとすることで,
$\cal R$上の正方格子状に配置された観測点で計41×61=2,501の供試体表面における鉛直動の時刻歴波形を取得した.
以下では,位置$(x,y)$において観測した時刻歴波形を$a(x,y,t)$と表す.ただし,$t$は超音波の
送信時刻を0とする経過時間を表す.$x,y$および$t$はいずれも離散変数だが,
簡単のため連続変数として扱い,$a(x,y,t)$に関する微分や積分は,離散化して評価することを前提としたものと解釈する.
\section{計測結果}
Fig.3に計測結果の一部を示す.この図は,時刻t=20~23μ秒で観測された波動場を1μ秒間隔のスナップショットとして示したもので,左から右へ表面波が伝播する様子が示されている.初動位置を正確に特定することは難しいが,大きな振幅を持つ波動が通過した背後の領域にも,多重散乱により振動が継続する様子が見られる.
\begin{figure}[t]
\begin{center}
\includegraphics[clip,scale=0.5]{Figs/cod.eps}
\caption{
	超音波の送信位置$\cal S$と受信領域$\cal R$の配置.
}
\label{fig:fig4}
\end{center}
\end{figure}

3.位相構造と波数ベクトル分布の評価
Fig.4に周波数0.4,0.6,0.9および1.2MHzの波形成分に対する位相の空間分布を示す.相対的に低周波の0.4と0.6MHzでは,等位相線(波面)がy軸方向に伸びる1次元的な構造を示している.ただし,波面は屈曲して直線的ではない.一方,0.9および1.2MHzでは,y方向への位相の揺らぎが大きく,平面波的な構造が見られない.Fig.5に,位相分布の勾配を中央差分で近似して求めた,波数ベクトルkの確率密度分布を示す.Fig.5-(a)は,波数ベクトルの大きさ|k |に関する,(b)はx軸方向から測ったkの方向に関する確率密度を示している.(a)の図にあるように,|k |の確率密度は非対称かつ有限な幅をもち,ガウス分布的でもない.また,周波数が大きくなるにつれ分散が増加している.これは,波数と周波数の関係が確定的に定められないこと,高周波になる程波動場の分散性が強まるが,波数の大きさは一定範囲に留まることを示している.また,波数ベクトルは入射方向に配向するが,相対的に高周波の1.0MHzと1.2MHzでは配向性が低下し,伝搬経路の屈曲が強まることを示している.これら波数ベクトルの確率密度と周波数の関係を記述する法則を見出すことは今後の課題だが,花崗岩のランダム媒体としてのモデル化においては,このような波数ベクトルの特徴を反映する必要がある.


\section{数値解析モデル}
	%%%%%%%%%%%%%%%%%%%%%%%%%%%%%%%%%%%%%%%%%%%%%%%%%%%%%%%%%%%%%%%%%%%%%%%
\subsection{多孔質体モデル}
図-\ref{fig:fig3}に,ランダムウォーク・シミュレーションに用いる4種類の
多孔質体構造(モデルタイプ1$\sim$4)を示す.これらは固相粒子配置の規則性,粒径分布,
粒子形状がマクロ拡散挙動に与える影響を調べることを意図したものである.
図-\ref{fig:fig3}でグレーの部分は固相を,黄色の部分は間隙を表し,
間隙水が存在しない$(S_r=0)$状態での多孔質体モデルを示している. 
モデルタイプ1は,直径$\frac{W}{10}$の円形粒子を,ユニットセル内で10$\times$10の正方格子状に配置したもので,
飽和度が拡散係数に与える影響や,拡散係数の大小を議論する際の参照モデルの役割を果たす.
モデルタイプ2は,直径$\frac{W}{10}$の円形粒子100個をユニットセル内にランダムに配置したものである.
粒子直径と粒子数$N_p$はモデルタイプ1と同じであることから,間隙率$n$も互いに等しく,
その値は
\begin{equation}
	n=1-\frac{\pi}{4}\simeq 0.215
	\label{eqn:n_val}
\end{equation}
である.一方,モデルタイプ3と4は,粒径分布の影響をみるためのもので,
タイプ3は円形粒子を,タイプ4はアスペクト比$a$が$0.4\sim 1.0$の楕円粒子を
充填したものである.なお,アスペクト比$a$は一様分布で,粒子直径はガンマ分布に従って
ランダムに与える.ただし,いずれのモデルも間隙率$n$は式(\ref{eqn:n_val})程度と
なるよう粒子数を設定した.また,粒子直径は最小値を$\frac{W}{100}$最大値を$\frac{W}{5}$とし,
平均粒子直径が約$0.085W$,粒子直径の標準偏差が約$0.05$,粒子数$N_p$が100前後と
なるようにした.さらに,モデルタイプ2$\sim$4は不規則な充填構造であるため,
固相粒子の粒径や数,配置に応じてマクロ拡散係数値も変動すると予想される.
そこで,これら3種類のモデルタイプについては,粒子位置や粒径が異なる20個のモデルを作成し,
各々のモデルについて複数の飽和度でランダムウォークによる拡散シミュレーションを行った.
以上,タイプ1から4のモデルに関する
特徴を表\ref{tbl:types}にまとめて示す.
\begin{table}[htb]
  \caption{多孔質構造(タイプ1$\sim$4)の特徴}
%{\small
  \begin{tabular}{c|c|c|c|c}
\hline
    タイプ & 1 & 2 & 3 & 4 \\
	\hline\hline
	  充填構造& 単純立方& \multicolumn{3}{c}{ランダム} 
%	  & ランダム &ランダム 
	  \\
    \hline
粒子形状 & 円 &円 & 円 & 楕円\\
	\hline
   平均粒径 & 0.1 & 0.1 & 0.089 & 0.082\\ 
 (標準偏差) & (0) & (0) & (0.047) & (0.050) \\
\hline
粒子数 & 100 & 100 & 94$\sim$119 & 81$\sim$145 \\
(平均) & (100) & (100) & (107) & (111) \\
\hline
モデル数& 1 &20 &20 & 20 \\
\hline
  \end{tabular}
\label{tbl:types}
%}
\end{table}
\begin{figure}[t]
\begin{center}
%\includegraphics[clip,scale=0.4]{Figs/fig2.eps}
\caption{
	数値シミュレーションに用いた4種類の多孔質構造モデル. 
}
\label{fig:fig3}
\end{center}
\end{figure}
%%%%%%%%%%%%%%%%%%%%%%%%%%%%%%%%%%%%%%%%%%%%%%%%%%%
\subsection{間隙水の配置}
モデルタイプ1$\sim$4の多孔質構造モデルそれぞれについて,飽和度$Sr$を0.2から1.0まで0.1刻みで
変化させた不飽和多孔質体モデルを作成する.なお,タイプ1については,固相粒子の配置は一通りに
決まっているため,間隙水の分布だけが異なる20通りのモデルを各飽和度で作成した.
すなわち,4種類のモデルタイプ,9段階の飽和度で各20個の多孔質体モデルを用い,
合計720ケースのランダムウォーク・シミュレーションを行う.
各多孔質体モデルは,水分配置を十分な空間解像度で表現するため,ユニットセルの一辺を1,024分割し,
セルサイズを$h=\frac{W}{1,204}$とした.
液相配置決定のためのモンテカルロ・シミュレーションで必要となる界面自由エネルギー$\gamma_{\alpha\beta}$の値は,
$\alpha=1$で気相を,$\alpha=2,3$でそれぞれ液相と固相を表すとき,
\begin{equation}
	\gamma_{12}=70.0,\,  \gamma_{23}=370.0, \, \gamma_{31}=420.0, \ \ 
	[{\rm mJ/m}]
	\label{eqn:gamma_val} 
\end{equation}
とした\cite{Berkowitz}.これは,気相として空気を,液相として水を,固相として石英を想定したもので,
固相表面の濡れ角\cite{Gennes}が$45$度となる親水的な表面に相当する.
なお,界面自由エネルギーの最小化では,$\gamma_{\alpha\beta}$の値は互いの比だけが影響し,
近傍セルの接触面積$\Delta s$や$\gamma_{\alpha\beta}$の絶対値には依存しない.
以上の計算条件に対して得られた不飽和多孔質体モデルの一例を図\ref{fig:fig4}に示す.
これは,モデルタイプ2で,飽和度を$Sr=0.2,0.4,0.6$および$0.8$としたときの結果を示したもので,
飽和度が低い場合には固相粒子間の狭隘部に間隙水が集中し,飽和度が比較的高い場合には
固相粒子直径程度の気泡を残しつつ,次第に広い間隙部分を水分が埋めていく様子が現れている.
\begin{figure}[t]
\begin{center}
%\includegraphics[clip,scale=0.4]{Figs/fig3.eps}
\caption{
	4つの異なる飽和度における間隙水分布の様子.
	モデルタイプ2,飽和度$Sr$が(a) 0.2,(b) 0.4,(c) 0.6,(d) 0.8の場合を示す. 
}
\label{fig:fig4}
\end{center}
\end{figure}
%%%%%%%%%%%%%%%%%%%%%%%%%%%%%%%%%%%%%%%%%%%%%%%%%%%
\subsection{ランダムウォークとマクロ拡散係数の評価}
%本研究では,ランダムウォークを液相セルの中心をノードとする格子上で行うため,
%空間ステップ長は液相セルのサイズ$h=\frac{W}{1,024}$に等しい. 
ユニットセル内各点からの影響が反映された拡散特性を調べるためには,
液相内にある全てのランダムウォークノードを,いずれかの拡散粒子が
繰り返し訪れることができるように拡散粒子数$N_{wk}$や時間ステップ$N_t$を設定する
必要がある.本研究で用いる多孔質体モデルは,ユニットセル中には,およそ100個の固相粒子がある.
そこで,各固相粒子の周辺におよそ100個の拡散粒子が常時存在する状況が生じるように, 
ここでは1万個のランダムウォーカーをユニットセル中に一様に分布させた状態から
ランダムウォーク・シミュレーションを行った.
また,適切な時間ステップ数$N_t$を設定するために,
式(\ref{eqn:abnormal})を無次元化し,以下に示す無次元化時間を元に$N_t$を与えた.\\
式(\ref{eqn:abnormal})には時間と長さの次元を持つ量が含まれる.
そこで,ユニットセルサイズ$W$を長さの基準にとり,変位成分$u_i,v_i$を
\begin{equation}
	U_i=\frac{u_i}{W},\ \  V_i=\frac{v_i}{W}
\end{equation}
と無次元化する.時間に関しては,
\begin{equation}
	\tau=\frac{D_0}{W^2}t
	\label{eqn:tau}
\end{equation}
により,液相の拡散係数$D_0$のスケールにあった無次元化時間を導入する.
このとき式(\ref{eqn:abnormal})を
\begin{equation}
	\left<U_i\right>=\left<V_i\right>=2\bar K \tau ^{\alpha}
	\label{eqn:Kb}
\end{equation}
と書けば,一般化拡散係数$\bar{K}$は
\begin{equation}
	\bar{K}=\frac{K}{D_0W^{2(1-\alpha)}}
	\label{eqn:Kb_def}
\end{equation}
で与えられる無次元量となる,
ここで,単位時間$\tau=1$は$t=W^2/D_0$に相当し,この時刻を式(\ref{eqn:Einstein})に代入すれば
\begin{equation}
	\left<u_i^2\right> =\frac{2D}{D_0}W^2
\end{equation}
となる.従って,$D$と$D_0$のオーダーが大きく異ならない限り,$\tau=1$の時点では,
ユニットセル内で拡散が十分進行した状態にあると言える.
またこのことは,ユニットセルにおける拡散挙動は,$\tau<1$程度の時間スケールで調べれば
十分であることを意味する.そこで,以下のシミュレーションでは$\tau=1/4$となる
時間ステップまでランダムウォークを行うこととする.
$\tau=1/4$に達するための時間ステップ数は,式(\ref{eqn:dt})と式(\ref{eqn:tau})より,
\begin{equation}
	N_t=\left. \frac{t}{\Delta t} \right|_{\tau=1/4}=\frac{W^2}{h^2}=1,048,576
\end{equation}
と与えられる.\\
\hspace{\parindent}
以上の設定で,無限に広い液相領域における拡散解析をランダムウォークで行った
結果を図-\ref{fig:rwk0}に示す.
この図において,(a)は無次元化した平均2乗変位$\left<U^2\right>$と時間$\tau$の関係を,
(b)は$\tau=0.23$における拡散粒子の変位分布を表している.
なお,図\ref{fig:rwk0}-(b)に示されるように,拡散はほぼ等方的であることから,
$X$方向変位$U_i$,$Y$方向変位$V_i$を区別せずに平均した,
\begin{equation}
	\left< U^2 \right>=\frac{
			\left<U_i^2 \right>
		+
		\left<V_i^2 \right>
	}{2}
	=\sum_{i=1}^{N_p}
	\frac{U_i^2+V_i^2}{2N_p}
	\label{eqn:Ubar}
\end{equation}
を,図\ref{fig:rwk0}-(a)では示している.この結果から一般化拡散係数$\bar K$と
べき指数$\alpha$を求めると,それぞれ
\begin{equation}
	\bar{K}=0.9996, \ \ \alpha=1.0000
\end{equation}
となる.この問題では拡散の妨げとなる固相や気相領域が存在しないことから,
マクロ拡散係数$D$とミクロ拡散係数$D_0$は同じである.
従って,$\bar K=1, \alpha=1$が正解であり,ランダムウォークでの拡散解析によって
非常に近い値が得られていることが分かる.
また,図\ref{fig:rwk0}-(b)の
変位分布について,共分散行列を
\begin{equation}
	\fat{S}( \{U_i\},\{V_i\})
	=\left(
	\begin{array}{cc}
		S_{UU} & S_{UV}	 \\
		S_{VU} & S_{VV} 	
	\end{array}
	\right)
\end{equation}
として,各分散値を計算すれば,
\begin{equation}
	\bar{S}= \frac{S_{UU}+S_{VV}}{2}=0.4604
\end{equation}
\begin{equation}
	\fat{S}( \{U_i\},\{V_i\})
	=
	\bar{S}
	\left(
	\begin{array}{cc}
		 0.9897 & -0.0008 \\
		-0.0008 &  1.0103
	\end{array}
	\right)
\end{equation}
であった.このように,対角項は1$\%$程度の差で一致し,非対角項はそれよりも3桁程度は小さく
ほぼ等方的な拡散が再現されていることが分かる.以上のように,拡散場は等方的で平均2乗変位の
グラフが滑らかに推移し,マクロ拡散係数も正しく得られていることから,拡散粒子数も十分で
あったと考えられる.
\begin{figure}[t]
\begin{center}
%\includegraphics[clip,scale=0.32]{Figs/fig10.eps}
\caption{
	ランダムウォークによる無限液相領域における拡散解析の結果.
	(a) 平均2乗変位の時間変化. (b)$\tau=$0.23における拡散粒子の変位分布. 
}
\label{fig:rwk0}
\end{center}
\end{figure}


\section{数値解析結果と考察}
	本節では,参照モデルとしての役割を果たすモデルタイプ1に対するシミュレーション
結果をはじめに,次に,不規則な充填構造をもつモデルタイプ2,3および4に対する
結果を,モデルタイプ1と比較する形で示す.
それらの結果を踏まえ,固相粒子形状や充填構造,粒径分布が
マクロ拡散係数や異常拡散の程度に与える影響を調べる.
%
\subsection{モデルタイプ1に対する結果}
図-\ref{fig:fig5}に,ランダムウォーク・シミュレーションで得られた
平均2乗変位の時間変化を示す.このグラフは,横軸を無次元化時間$\tau$とし,
縦軸を$W$で無次元化した拡散粒子の平均2乗変位としたもので,
$S_r$=0.2から1.0まで,異なる9段階の飽和度に対する結果を示している.
先に述べたように,各飽和度での計算は20通りの異なる水分配置に対して行っており,
図-\ref{fig:fig5}はそれら全ての結果を合わせて算出した平均2乗変位を表している.
%
%なお,平均2乗変位の算出に先立ち,拡散粒子変位の$XY$平面内における分布を調べたところ,
%目立った配向性はなく等方的であった.そこで,$X$方向変位$U_i$,
%$Y$方向変位$V_i$を区別せずに平均した,
%\begin{equation}
%	\left< U^2 \right>=\frac{
%			\left<U_i^2 \right>
%		+
%		\left<V_i^2 \right>
%	}{2}
%	=\sum_{i=1}^{N_p}
%	\frac{U_i^2+V_i^2}{2N_p}
%	\label{eqn:Ubar}
%\end{equation}
%を平均2乗変位として図\ref{fig:fig5}に示した.
このグラフから明らかなように,いずれの飽和度でも$\left<U^2\right>$は時間に対して増加するが,
飽和度が$S_r=0.2$から0.5程度の場合,$\left<U^2\right>$の値は途中で頭打ちとなる.
これは,互いに分離した液相領域が多数存在するために,有限な液相領域のサイズを超えて変位が
増加できないためである.一方,$S_r>0.5$のときには,$\left<U^2\right>$は$\tau$に対して
単調に増加を続ける.このことは,飽和度$S_r$の増加に伴い孤立していた液相領域が連結して,
ユニットセルをパーコレートする拡散パスが次第に形成されることを表している.\\
\hspace{\parindent}
図-\ref{fig:fig5}の結果に式(\ref{eqn:Kb})を最小2乗法でフィッティングし,
拡散係数$\bar{K}$とべき指数$\alpha$を求めた結果を,それぞれ図-\ref{fig:fig6}と
図-\ref{fig:fig7}に示す.$\bar{K}$は$S_r<0.4$ではほぼゼロで,
概ね$S_r=0.4$をしきい値として増加をはじめ,最終的に間隙が完全に飽和した$S_r=1.0$では
マクロ拡散係数の値が0.4程度となっている.
つまり,飽和状態でマクロ拡散係数は,液相の拡散係数$D_0$の4割程度となり,
間隙率が約21.5\%であることを考慮すれば,ユニットセルに占める拡散相の比率よりも
大きな値であると言える.このことは,多孔質体のマクロ拡散係数を評価する際,
拡散相と非拡散相(ここでは気相と固相)の割合だけでなく,
間隙形状の効果を考慮することが必要であることを意味している.
一方,図-\ref{fig:fig6}に示したべき指数$\alpha$の挙動を見ると,
$\alpha$は$S_r=0.3$から顕著に増加し,$S_r=1.0$でのみ$\alpha=1$となっている.
すなわち,飽和度に応じた程度の差はあるものの,不飽和状態では常に遅い異常拡散が起こること,
通常拡散となるのは飽和状態の場合に限られることが示されている.
\begin{figure}[t]
\begin{center}
%\includegraphics[clip,scale=0.50]{Figs/fig4.eps}
\caption{
	平均2乗変位$\left<U^2\right>$の時間変化.
	モデルタイプ1,飽和度$S_r=0.2\sim 1.0$に対する結果.
}
\label{fig:fig5}
\end{center}
\end{figure}
%---------------------------------------------------
\begin{figure}
\begin{center}
%\includegraphics[clip,scale=0.50]{Figs/fig5.eps}
\caption{
	マクロ拡散係数$\bar K$と飽和度の関係(モデルタイプ1対する結果).
	}
\label{fig:fig6}
\end{center}
\end{figure}
%---------------------------------------------------
\begin{figure}
\begin{center}
%\includegraphics[clip,scale=0.5]{Figs/fig6.eps}
\caption{
	べき指数$\alpha$と飽和度の関係(モデルタイプ1に対する結果).
	}
\label{fig:fig7}
\end{center}
\end{figure}
\vspace{-5mm}
%%%%%%%%%%%%%%%%%%%%%%%%%%%
%
\begin{figure}[t]
\begin{center}
%\includegraphics[clip,scale=0.50]{Figs/fig11.eps}
\caption{
	平均2乗変位$\left<U^2\right>$の時間変化.
	モデルタイプ2,飽和度$S_r=0.2\sim 1.0$に対する結果.
}
\label{fig:fig11}
\end{center}
\end{figure}
\begin{figure}[t]
\begin{center}
%\includegraphics[clip,scale=0.50]{Figs/fig12.eps}
\caption{
	平均2乗変位$\left<U^2\right>$の時間変化.
	モデルタイプ3,飽和度$S_r=0.2\sim 1.0$に対する結果.
}
\label{fig:fig12}
\end{center}
\end{figure}
\begin{figure}[t]
\begin{center}
%\includegraphics[clip,scale=0.50]{Figs/fig13.eps}
\caption{
	平均2乗変位$\left<U^2\right>$の時間変化.
	モデルタイプ4,飽和度$S_r=0.2\sim 1.0$に対する結果.
}
\label{fig:fig13}
\end{center}
\end{figure}
\begin{figure}[h]
\begin{center}
%\includegraphics[clip,scale=0.50]{Figs/fig7.eps}
\caption{
	マクロ拡散係数$\bar K$と飽和度の関係(モデルタイプ1$\sim$4に対する結果).
	}
\label{fig:fig8}
\end{center}
\end{figure}
%
\begin{figure}[h]
\begin{center}
%\includegraphics[clip,scale=0.5]{Figs/fig8.eps}
\caption{
	べき指数$\alpha$と飽和度の関係(モデルタイプ1$\sim$4に対する結果).
	}
\label{fig:fig9}
\end{center}
\end{figure}
\subsection{モデルタイプによる拡散挙動の違い}
図-\ref{fig:fig11}$\sim$図-\ref{fig:fig13}に,タイプ2$\sim$4のモデルに対する
平均2乗変位の時間変化を示す.それぞれのグラフは,図-\ref{fig:fig5}と同様,
横軸が無次元化時間$\tau$,縦軸は平均2乗変位$\left<U^2\right>$としたもので,
計算を行った全ての飽和度$S_r$に対する結果が示されている.
これらタイプ2$\sim$4に対する結果は互いによく似た形状の曲線群となっている.
一方,タイプ1との比較でみると,時間に対して直線的に変化するケースがタイプ2$\sim$4
では見当たらず,程度の差はあれ,いずれも異常拡散となっていることが示唆されている.\\
\hspace{\parindent}
平均2乗変位の時間推移から決定した,タイプ1$\sim$4のモデルに対する
マクロ拡散係数$\bar{K}$とべき指数$\alpha$を,それぞれ,図-\ref{fig:fig8}と
図-\ref{fig:fig9}に示す.
図-\ref{fig:fig8}にあるように,拡散係数$\bar{K}$はモデルタイプによらず
飽和度$S_r$に対して類似した関数形で単調増加する.
ただし,不規則な充填構造を持つモデルタイプ2と3および4は,モデルタイプ1に比べて
より拡散係数の値が小さい.これは,モデルタイプ1では固相粒子が同一直線上
に並んでいるため,間隙に十分な水分がある場合,拡散粒子が直線的な経路でユニットセル
を横断あるいは縦断できるためと考えられる.
これに対し,不規則な充填構造を持つモデルでは,拡散粒子物が不規則に並んだ
固相粒子を迂回しながら移動し,拡散経路が常に屈曲することからマクロ拡散係
数値は小さくなり,べき指数も1に達することなく常に遅い拡散となっている.
% --- 追加
ただし,タイプ1のべき指数$\alpha$は,$Sr<0.4$の低飽和度の側では
他のモデルと比べて若干小さな値となっている.
これは,タイプ2$\sim$4では大小様々な粒径の固相粒子を迂回して拡散粒子が
移動するのに対し,タイプ1で水分が少ない場合は,拡散粒子が常に同一粒径
の固相粒子表面近傍を這うように迂回しながら移動する必要があることに起因する.
%
次に,モデルタイプ3と4の結果を比較すると,$\bar{K},\alpha$とも大きな
違いはなく,平均粒径や間隙率,飽和度が同程度であれば,粒子形状がマクロな
拡散に与える影響は小さいことが分かる.
一方,モデルタイプ2では,べき指数$\alpha$の挙動はその他不規則充填構造モデル
(タイプ3,4)と大差ないものの,
マクロ拡散係数は$S_r>0.8$程度の飽和度において他と比べ明らかに小さい.
これは以下の理由によると考えられる.
%
粒子配置がランダムな場合,水分量が多いときでも間隙のネットワークが屈曲し,
拡散経路が長くなる. さらに,均一粒径の固相粒子を充填した場合,
広い粒径分布を持つ粒子を充填したときに比べ,互いに接触した粒子が長いネットワークを作り易い.
つまり,拡散のボトルネックとなるような狭隘部を迂回する経路が見つかりにくくなる.
タイプ2のモデルでは,これら2つの効果が相まって,拡散係数が他のモデルよりも
小さくなると考えられる.
%
%均一粒径の粒子を容器内に密に充填することが,広い粒径分布をもつ粒子を
%同じ容器に充填することに比べてより困難なことは,日常経験からも明らかである.
%このことは,互いに接触して密に配列した粒子のネットワークが,容器を横断あるいは縦断するように発達し易いことを意味する.
%従って,均一粒径の粒子が不規則に充填されているとき,拡散物質は
%固相粒子が密に並んだ箇所を簡単には迂回できず,変位の増加が抑制される.
%この結果,マクロ拡散係数の値がモデル2では小さくなった原因と考えられる.
%%単一粒径の粒子を不規則な配置で充填した場合,粒子の一部は平均より密に,
%%残る部分は疎に充填されざるを得ない.固相粒子の疎に配置された箇所では,
%%拡散粒子は移動しやすく,密に固相粒子が配置された領域では移動が抑制される.
%%従って,固相粒子が密に配置された領域を迂回する拡散経路が無い限り,
%%局所的な拡散係数の増加は平均2乗変位の増加,すなわちマクロ拡散係数の
%%増加に貢献しない.
%%以上の挙動を整理すると,モデルタイプ1では,拡散経路が液相領域に制限される
%%ことでマクロ拡散係数は,液相の拡散係の1/2程度まで低減される.
%%モデルタイプ2から4では,
%%拡散経路の屈曲による効果が加わることでより拡散係数が低下する.
%%さらに,単分散粒子系を不規則に充填したタイプ2では,局所的な間隙率の変動に起因した
%%拡散物質の移動抑制効果が現れる.さらに,べき指数がモデルタイプ2から4では1
%%に達しないことから判断して,拡散経路の屈曲が遅い異常拡散の原因となることがわかる.
\subsection{ランダムウォーカー変位の確率密度分布}
最後に,ランダムウォーカー変位の確率密度分布について調べた結果を
図-\ref{fig:fig10}に示す.これは,時刻$\tau=0.25$における
%変位成分$U_i$と$V_i$を,ヒストグラムを正規化して示したものであり,
水平変位$U_i$と鉛直変位$V_i$を併せてヒストグラム化したもので,
縦軸は正規化されており確率密度とみなすことができる.
これら4つのプロットのうち(a)と(b)は,モデルタイプ1に対する結果を
(c)と(d)はモデルタイプ4に対する結果を表す.
それぞれ飽和度は$S_r=$0.7と1.0で,$S_r=0.7$のときの拡散係数$\bar K$は,
$S_r=1.0$の場合に比べていずれのモデルも概ね1/4程度となっている.
なお,青の実線は,変位成分$\left\{U_i,V_i\right\}$と同じ分散をもつ正規分布を示している.
タイプ1に対する結果を表示するにあたり,ヒストグラムの階級幅を,モデルタイプ4のプロット
に比べて意図的に粗く設定している.タイプ1のモデルでは,固相粒子が規則的に配置されているために,
階級幅を固相粒子間距離である$0.1W$よりも小さくすると,確率密度分布に
周期$0.1W$の凹凸が現れ,全体の分布形状がわかりにくくなる.
これを避けるために,図-\ref{fig:fig10}の(a)と(b)では階級幅を
0.67$W$とし,(c)と(d)ではその半分の値としている.\\
\hspace{\parindent}
ここで,図-\ref{fig:fig10}-(b)をみると,
モデルタイプ1では,飽和度$S_r=1.0$のとき確率密度分布がほぼ完全な正規分布となって
いることが分かる.実際,この場合のべき指数は$\alpha=1$であり,
通常の拡散方程式に従う結果であることが確認される.
一方,モデルタイプ4では,(d)のプロットにあるように$S_r=1.0$で概ね正規分布に近い形
となるものの,(b)との比較でみると若干正規分布よりも裾の広い形になっていることに気づく.
このケースでは,べき指数も$\alpha=0.7$と1より小さく,やや遅いタイプの拡散で,
その影響が確率密度分布にも現れていることが分かる.
次に,飽和度が$0.7$の場合について見れば,正規分布からのずれがより
明確となっている.特に(c)のケースでは,確率密度分布が中央で尖った形となっており,
裾野も正規分布に比べて明らかに広い.
このとき$\alpha$は0.5程度の遅い異常拡散となっており,
変位の確率密度も正規分布で表現できないことは明らかである.
以上の結果は,不飽和多孔質媒体のマクロ拡散問題を解析する際,
モデルタイプ1で$S_r=1.0$のような特別な状況を除き,
通常の拡散方程式を用いることことはできないことを意味する.
従って,アップスケーリングされた拡散問題を考える際には,
例えば非整数階微分の拡散方程式を用いることや,
図\ref{fig:fig10}に示されるような変位の確率密度に従うランダムウォーク
を設計する等の対応が必要と考えられる.
%間隙スケールの情報を取り込んだ上で,マクロな拡散解析をランダムウォークで行うことができる.
%その場合,平均2乗変位の時間変化に対する関数形を仮定することなくアップスケーリングを行うことができ,
%同様な手順を繰り返すことで,多段階のアップスケーリングにつなげることが可能となる.
%いずれのアプローチが有効であるかは今後検討すべき課題と.
%
\begin{figure}[h]
\begin{center}
%\includegraphics[clip,scale=0.35]{Figs/fig9.eps}
\caption{
	ランダムウォーカー変位の確率密度分布.青の実線は,分散を一致させたときの正規分布を表す.
	}
\label{fig:fig10}
\end{center}
\end{figure}


\section{まとめ}
本研究では,2次元不飽和多孔質体モデルを用いたランダムウォークによる拡散解析を行い,
間隙構造や水分量が分子拡散挙動に与える影響について調べた.この方法では,
任意の形状と粒径分布をもつ固相粒子で構成された多孔質体を扱うことができ,
間隙構造や間隙率,飽和度や粒子表面の濡れ性を自由に設定したモデルを用いて
拡散解析を行うことができる.また,ランダムウォークによるシミュレーション結果から,
一般化された拡散係数と,平均2乗変位のべき指数を求めることで,異常拡散の程度も
定量的に評価することができる.本稿では,上記の方法により4種類の多孔質構造モデルに
ついて拡散シミュレーションを行った結果,以下に示す知見を得ることができた.
\begin{itemize}
\item
	多孔質体では,飽和度と間隙率に応じて拡散経路が制限され,
	マクロ拡散係数は液相の拡散係数よりも小さくなる.ただしその低減率は,
	間隙形状の影響を受ける.
\item
	拡散経路が屈曲することにより,マクロスケールでの拡散は遅い異常拡散となり,
	通常の拡散方程式には従わなくなる.
\item
	固相粒子のアスペクト比が0.4$\sim$1.0程度であれば,
	粒子形状が拡散挙動に与える影響は小さい.
\item
	均一な粒径の固相粒子が不規則に充填された多孔質構造では,
	多分散系の固相粒子で構成された多孔質構造に比べて小さなマクロ拡散係数をもつ.
\item
	%平均2乗変位のべき指数で表される異常拡散の程度は,
	%飽和度に対して単調に減少する.	
	平均2乗変位のべき指数$\alpha$は飽和度に対して単調に増加するが,
	通常拡散を意味する1を超えることはない.すなわち,異常拡散の程度は飽和度に対して単調に減少する.
\item
	異常拡散の程度は,固相粒子配置が不規則で飽和度が低い場合に大きい.
	一方,粒径分布や粒子形状が異常拡散の程度に当てる影響は顕著でない.
	このことは,拡散経路の屈曲が異常拡散の主たる原因であることを示唆する.
\end{itemize}
以上は,間隙スケールの形状を考慮した拡散解析を行うことで,不飽和多孔質体に
おける拡散挙動の理解を深めるために有用な情報が得られることを示している.\\
\hspace{\parindent}
今後は,同様な方法を3次元問題に適用し,実験データの解釈に利用することが課題となる.
そのためには,各計算過程(固相粒子のパッキング,間隙水配置の決定,ランダムウォーク)
の効率化を行うことも必要となる.また,本手法を間隙水の移流を考慮した拡散解析に
拡張し,機械的分散のメカニズムを調べること,拡散物質の吸着や拡散物質の
反応を考慮すること,ランダムウォークベースのアップスケーリング手法を考案することも
今後の重要な課題であると考えられる.
%%%%%%%%%%%%%%%%%%%%%%%%%%%%%%%%%%%%%%%%%%%%%%%%%%%%%%%%%%%%%%%%%%%%%%%%%%%%%
%%%%%%%%%%%%%%%%%%%%%%%%%%%%%%%%%%%%%%%%%%%%%%%%%%%%%%%%%%%%%%%%%%%%%%%%%%%%%
%\newpage
%\lastpagecontrol[2cm]{13.7cm}
\vspace{0mm}
\begin{thebibliography}{99}
%\vspace{5mm}
\begin{spacing}{1.175}
\bibitem{JAEA}
	日本原子力研究開発機構 福島研究開発部門 福島研究開発拠点 福島環境安全センター:  
	福島における放射性セシウムの環境動態研究の現状(平成30年度版),JAEA-Research,
	2019-002,DOI:10.11484/jaea-research-2019-002,2019.
\bibitem{NUMO}
	原子力発電環境整備機構:地層処分事業の安全確保(2010年度版)-確かな技術による安全な地層処分の実現のために-, NUMO-TR-11-01, 2011.
\bibitem{Bear1}
	Bear,J.:Dynamics of fluids in porous media,Dover,1972.
\bibitem{Bear2}
	Bear,J.and Bachmat,Y.:Introduction to modeling of transport phenomena in porous media, Kluwer Academic Publishers,1990.
\bibitem{Nakano}
	中野政詩:土の物質移動学,東京大学出版会, 1991.
\bibitem{Fujinawa}
	藤縄克之:環境地下水学,共立出版,2010.
\bibitem{Rubin}
	Rubin,S., Dror,I. and Berkowitz, B.: Experimental and 
		modeling analysis of coupled non-Fickian transport and 
		sorption in natural soils,Journal of Contaminant Hydrology, vol.132,pp.28-36,2012.
\newpage
\bibitem{Masuda}
	Masuda,A.,Ushida,K. and Okamoto, T. :Direct observation of spatiotemporal dependence of anomalous diffusion in 
		inhomogeneous 
		fluid by sampling-volume-controlled fluorescence correlation spectroscopy, Physical Review E, vol.72,pp.060101-1-4,2005.
\bibitem{Non_Fick_review}
	Berkowitz,B., Cortis,A. and Scher,A.:Modeling non-Fickian transport in geological formations as a continuous time random walk, Reviews of Geophysics,vol.44,pp.1-49,2006.
\bibitem{Upscaling_review}
	Frippiat,C.C. and Holeyman,A.E.:A comparative review of upscaling methods for solute transport in heterogeneous porous media, Journal of Hydrology, vol.262,pp.150-176,2008.
%\bibitem{Cuadros}
%	Cuadros, J. : Clay as sealing material in nuclear waste repositories, %Geology Today, Vol.24, No.3, pp.99-103, 2008.
\bibitem{Liang}
	Liang,Z.,Ioannidis,M.and Chatzis,I.:Permeability and electrical conductivity of porous media from 3D stochastic replicas of the microstructure, Chemical Engineering Science,vol.55,pp.5247-5262,2000. 
\bibitem{chalk_model}
	Talkudar,M.S., Torsaeter, O. and Howard, J.J.:
	Stochastic reconstruction, 3D characterization and network modeling of chalk, Journal of Petroleum Science and Engineering, 
	vol.35, pp.1-21,2002.
\bibitem{Digital_porous}
	Kainourgiakis,M.E.,Kikkinides,E.S.,Galani,Charalambop-oulou,G.C. and Stubos,A.K.:
	Digitally reconstructed porous media: Transport and sorption properties,
	Transport in Porous Media, Vol.58, pp.43-62, 2005.
\bibitem{Kim}
	Kim,D.H., Young,J.K., Lee, J.-S. and Yun, T.S..:Thermal and electrical response of unsaturated hydropholic and hydrophobic granular materials,
	 Geotechnical Testing Journal,Vol.34,No.5,pp.1-9,2011.
\bibitem{Narsilio}
	Narsilio,G.A., Kress,J. and Yun, T. S.:
	Characterisation of conduction phenomena in soils at particle-scale: 
	Finite element analyses in conjunction with syhtetic 3D imaging, Computers and Geotechnics, vol.38, pp.828-836, 2010.
\bibitem{Berkowitz}   
        Berkowitz,B. and Hansen,P.D.:A Numerical Study of the Distribution of 	Water in Partially Saturated Porous Rock, 
	Transport in Porous Media, Vol.45, pp.303-319, 2001.
\bibitem{MC}  
        Lu, N.,Zeidman, B. D., Lusk, M. T., Willson, C. S. and Wu, D. T.: 
	A Monte Carlo paradigm for capillarity in porous media, Geophysical Research Letters, Vol.37, L23402, 2010.
\bibitem{comp_phys}  
	Metropolis,N.,Rosenbluth,A.W.,Rosenbluth,M. N.,Teller,A. H. and	Teller,E.:
		Equation of state calculations by fast computing machines, 
	J.Chem.Phys., 21(6), pp.1087-1092, 1953.
\bibitem{SurfChem}
	Butt, H.-J., Graf, K. and Kappl, M., 
	"Physics and chemistry of interfaces", WILEY-VCH, 2006.
\bibitem{Kimoto}
	中島 唯一,木本 和志,河村 雄行:間隙水分布を考慮した不飽和多孔質体の熱伝導解析, 
	土木学会論文集A2, Vol.74, No.2, pp.I105-I114,2018.
\bibitem{Toda}
	戸田 盛和,物理学30講シリーズ 分子運動30講,朝倉書店,1996.
\bibitem{Rwk_textbook}
	Klaffter, J., and Sololov,I.M.,秋元 琢磨(訳): 
	ランダムウォークはじめの一歩,共立出版, 2018.
\bibitem{Gennes}
ドゥジェンヌ,ブロシェール ヴィアール,ケレ:表面張力の物理学, 吉岡書店, 2003.
%\bibitem{Terada}
%	寺田 賢二郎, 菊池 昇: 均質化法入門, 丸善株式会社, 2003.
%\bibitem{Zohdi}
%        Zohdi, T. I. and Wriggers, P. : An Intruduction to Computational Micromechanics,Springer, 2008.
%\bibitem{Iwasaki}
%	岩崎 佳介, 木本 和志, 市川 康明: 含水した砂質土の熱応答特性に関する実験および数値解析,  土木学会論文集A2, Vol.70, No.2,  ppI115-I124, 2014.
%\bibitem{Yun}
%	Yun, T. S., and Evans, T. M. : Three-dimensional random walk network model %for thermal conductivity in pariculate materials, 
%	Computers and Geotechnics, vol.37, pp.991-998,2010.
\end{spacing}
\end{thebibliography}
\vspace{-5mm}
\begin{flushright}
	\small
	\bf{ (Received July 19, 2019)\\
	(Accepted December 24, 2019)}
\end{flushright}
%\lastpagesettings
\newpage
%\lastpagecontrol[1cm]{13.7cm}
\lastpagecontrol[1cm]{25.7cm}
\end{document}

%\begin{minipage}[c]{13.7cm}
%\end{minipage}
%\lastpagecontrol[0cm]{13.7cm}
%\newpage
%\begin{multicols}{1}
%-------------------------------------------------
%-------------------------------------------------
%\end{multicols}

