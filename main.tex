%%#!platex
%
% Example of Japanese Paper of JSCE
% for LaTeX2e users
%
% revised on 4/25/2014
%
%%%%%%%%%%%%%%%%%%%%%%%%%%%%%%%%%%%%%%%%%%%
%
% もし jis フォントメトリックを使う場合は,以下をアンコメントしてください.
% \DeclareFontShape{JY1}{mc}{m}{n}{<-> s * jis}{}
% \DeclareFontShape{JY1}{gt}{m}{n}{<-> s * jisg}{}
%
\documentclass{jsce}
%
\usepackage{epic,eepic,eepicsup}
%\usepackage{graphicx,multicol}
\usepackage{graphicx}
\usepackage{multicol}
%\usepackage{showkeys}
\usepackage{setspace}
%  amsを使う方は以下をアンコメントしてください.
%\usepackage{amssymb,amsmath}
% 英語はサポートしているかどうか不明
% \inenglish
% 学会サンプルに times とあるので指定しておきます
\usepackage{times}
%
\finalversion
\pagestyle{empty}
%
\title{
	超音波計測に基づく花崗岩中の表面波伝播特性に関する研究
}%
\endtitle{
A STUDY ON THE PROPAGATION CHARACTERISTICS OF SURFACE WAVES IN A GRANITE BASED ON ULTRASONIC MEASUREMENT
}
%
% emailアドレスのフォントをタイプライター体にしたい方は次行をアンコメント
% \emailstyle{\ttfamily}
% emailアドレスを公開される方は,
%% \thanks{○○○○○○\email{your_name@foo.ac.jp}}のようにしてください.
%
\author{木本 和志\thanks{正会員 博士(工学) 岡山大学 環境生命科学研究科(〒700-8530 岡山県岡山市北区3丁目1番地1号)\email{kimoto@cc.okayama-u.ac.jp}}・
岡野 蒼\thanks{学生会員 岡山大学環境生命科学研究科 (〒700-8530 岡山県岡山市北区3丁目1番地1号)}・
斎藤 隆泰\thanks{正会員 博士(工学)群馬大学大学院理工学府 環境創生部門(〒376-8515 群馬県桐生市天神町 1-5-1)}・
佐藤 忠信\thanks{正会員 博士(工学)神戸学院大・現代社会学部 (〒000-0000 *******************************)}・
松井 裕哉\thanks{正会員 ●●(☓☓)日本原子力研究開発機構・幌延深地層研究センター(〒098-3224 北海道天塩郡幌延町北進432番地2)}
}
\endauthor{Kazushi KIMOTO, Aoi OKANO, Takahiro SAITOH, Hiroya MATSUI and Tadanobu SATO}
%
\abstract{
\small
本研究は,不飽和多孔質体における物質拡散挙動を調べることを目的に,ランダムウォークをベースとした
数値拡散解析手法の開発を行ったものである.ここで提案する方法では,
固相粒子を充填して作成した周期多孔質構造のユニットセルにおいて,
モンテカルロ法を用いて界面エネルギーが極小値をとるように間隙水を配置する.
作成した数値不飽和多孔質媒体は,液相を拡散媒体としたランダムウォークによる
物質拡散解析に用いる.ランダムウォーク・シミュレーションで得られた
拡散粒子変位の時刻歴から,拡散係数と平均2乗変位に関する時間のべき指数を評価し,
多孔質媒体が示すマクロな拡散特性を定量化する.本稿では,以上の方法で2次元拡散解析を行い,
飽和度や,固相粒子の形状と粒径分布が不飽和多孔質体のマクロ拡散に与える影響を調べた.
その結果,主として粒径分布が拡散係数の大きさに,固相粒子の充填構造が異常拡散の程度に
寄与することが明らかとなった.
}
%
\keywords{unsaturated porous media, Monte Carlo method, 
pore water distribution, random walk, anomalous diffusion}
%
\endabstract{% Yes blank line
\normalsize
This study investigates the propagation characteristics of high frequency surface wave in a granite block by ultrasonic measurement. In the experiment, a surface wave field excited by a line-focus transducer is scanned finely over a rectangular aperture with a laser Doppler vibrometer. From the measured waveforms, the spatial distribution of Fourier phase is reconstructed, and the structure of the wave front configuration is investigated. Due to the heterogeneity of the granite, we find the fluctuation of the wave front measured by the distribution of spatial gradient of phase. The intensity of the fluctuation is quantified as a local wavenumber vector. As the result it is found that the granite sample can be characterized by a stochastic wave number whose probability density is frequency dependent, asymmetric and non-Gaussian with a finite support. 
}
%
% \titlepagecontrol{1}
%
%\receivedate{2019.7.19}
% \receivedate{January 15, 1991}
%
% \def\theenumi{\alph{enumi}}  % もし enumerate 最初の箇条を (a) と
% \def\labelenumi{(\theenumi)} % したい場合・・・
%
\begin{document}
\maketitle
%%%%%%%%%%%%%%%%%%%%%%%%%%%%%%%%%%%%%%%%%%%%%%%%%%%%%%%%%%%%%%%%%%%%%%%%%%%%%
\section{はじめに}
	%\section{はじめに}
弾性波試験は材料物性や形状の非破壊評価にしばしば用いられている.建設材料の多くは,物性や形状が不均一なランダムかつ多孔質な媒体である.例えば,岩石は,複数の鉱物種で構成された,き裂やボイドを含む典型的なランダム多孔質媒体である.このような不均質媒体では,弾性波は屈折や散乱により非常に複雑な伝播挙動を示す.そのため,不均質材の弾性波試験の高精度化や信頼性の担保には,材料の不均質性を考慮した波動伝播モデルによるデータ解釈が必要となる.またそのようなモデルは,媒体のランダム性も適切に表現できるものでなければならない.波動媒体の不均質性を表現する際,物性値を平均と偏差成分に分け,偏差成分が既知の確率分布に従うとして解析が行われる.例えば,時間調和な波動場を考える場合,支配波動方程式に含まれる波数の項を,平均値と確率変数としての偏差成分に分離して記述する.このとき,波数の偏差成分がどのような確率分布に従うかは自明でなく,材料に応じて異なる可能性がある.また,確率分布が材料のどのよう性質を反映して決まるかということも,弾性波試験の観点からは重要である.これらのことは,理論解析だけで明らかにすることはできず,現実の材料をランダム不均質媒体としてモデル化するための確率分布は,弾性波の実測データを基に構築する必要がある.
 以上を踏まえ本研究では,花崗岩試料用いた超音波計測を行い,実測データから波数ベクトルが従う確率分布を推定する.花崗岩は数mm~数cm程度の結晶粒から成る多結晶質体で,超音波はき裂や粒界で散乱を起こす.そのため,材料の不均質部と強く相互作用する弾性波を室内試験で観測するための実験供試体として利用し易い.ここでは,圧電超音波探触子で花崗岩試料に表面波を励起し,その伝播挙動をレーザー振動計で可視化するとともに,波動場の位相と波数の構造を調べる.以下,超音波計測方法と計測結果,推定した位相の空間分布と,波数ベクトルの確率密度分布を順に示す.
\cite{Fujinawa})
\hspace{\parindent}
以上のことを踏まえ,本研究では,間隙スケールでの幾何形状や物性値をもとに,
不飽和多孔質体における巨視的な物質拡散挙動を評価するための数値解析
手法の開発を行う.具体的には,マルコフ連鎖モンテカルロ法を用い,任意の
形状と粒径分布をもつ固相粒子で構成される不飽和多孔質体の数値モデル
を作成する.作成した不飽和多孔質体モデルはランダムウォークによる分子拡散
シミュレーションに用い,マクロな拡散係数を求めるとともにマクロスケールでみた場合の
拡散が通常の拡散方程式に従うものであるか否かを調べる.
分子拡散による物質輸送は非常に遅いプロセスだが,そのメカニズムは拡散物質に
常に作用する.そこで本研究では,多孔質体における分子拡散による輸送特性を調べることを
第一に考え,間隙水の移動は考慮していない.また,多孔質構造の影響を見るためには,
複数の多孔質体モデル間での比較が必要となることから,計算負荷の低い2次元問題での
シミュレーションを行う.
%ただし,本論文で述べる方法は,原理的には3次元問題にもそのまま適用することができ,
%間隙水の移動による効果をランダムウォークに取り込むことは可能であることを指摘しておく.\\
ただし,本論文で述べる方法は,原理的には3次元問題にもそのまま適用することができる.
また,間隙水の移動による機械的分散効果も,拡散粒子の移動確率を異方的に
することで,ランダムウォーク・シミュレーションに取り込むことも可能である.
これらの拡張は,本稿で提案するシミュレーション手法をより現実的な問題へ
適用し,今後,実測データとの比較を行う上での次に取り組むべき課題である.\\
\hspace{\parindent}
以下,本論文では,第2節において不飽和多孔質体における2次元拡散問題の設定を示し,
続く第3節において,本研究で用いる一連の数値解析手法について述べる.
第4節では,ランダムウォークによる数値拡散解析に用いたシミュレーションモデルとその
諸元,各種解析条件を示す.第5節では,シミュレーション結果を示し,不飽和多孔質体
におけるマクロ拡散挙動に与える,固相粒子の充填構造,粒子形状,粒径分布と飽和度の
影響を調べる.最後に,本研究のまとめを今後の課題とともに述べる.\\
\hspace{\parindent}
本研究と同様なアプローチは,多孔質体における熱や物質の輸送特性を調べることを
目的とした研究にこれまでにも用いられている.例えば,X線CT試験や統計的な方法で作成した
多孔質体モデルは,間隙中の流速や物質輸送特性を調べるためにしばしば用いられてきた
\cite{Liang}$^-$\cite{Narsilio}.また,間隙水の配置を決定するためにモンテカルロ法を
用いた例も,これまでの研究でいくつか報告が行われている\cite{Berkowitz,MC}.
本研究は,これらの既往の研究で提案された手法とアイデアに負うところが大きく,個々の
数値解析手法を新規に開発するものでない.しかしながら,本研究独自の取り組みと
貢献は以下の3点にあると考えられる.1点目として,本研究では任意の形状と粒径分布を
持つ固相粒子が充填された多孔質構造モデルを作成して,マクロ拡散挙動を調べていること
が挙げられる.
これにより,水分量だけでなく,固相構造が間隙水中の物質拡散挙動に与える影響を詳細に調べることが
可能となっている.2点目は,マクロ拡散係数と水分量の関係だけでなく,異常拡散の
程度が飽和度に対してどのように変化するかを調べた点にある.また3点目として,固相粒子の形状
と粒径分布,充填構造が拡散挙動に与える影響を調べた結果,異常拡散の起源
が拡散経路の屈曲にあることを明らかにしたことが挙げられる.これら第2,第3点目に述べた知見は,
本研究で行った数値シミュレーションの結果として因果関係が明らかになったという点で
独自の貢献といえる.

	\vspace{-2mm}
\section{超音波計測実験}
	%\vspace{-4mm}
	%\section{2次元拡散問題の設定}
固相と液相,および気相からなる2次元不飽和多孔質媒体における溶質の拡散問題を考える.
多孔質媒体は十分に広く無限領域とみなすことができ,間隙水は静止していると仮定する.
また,拡散物質(溶質)の液相中での自己拡散係数$D_0$は場所に依らず一定かつ既知として,
分子拡散によって液相内をランダムに移動する拡散物質を,粗視化した多数の
粒子として表現する.それら粒子群の変位とその時間発展挙動を調べることで,不飽和多孔質媒体
全体としての平均的な拡散係数(マクロ拡散係数)$K$を求める.
なお,無限領域を数値解析上で模擬するために,多孔質媒体は周期性をもつと仮定し,そのユニット
セル$\Omega$を対象として一連の解析を行う.\\
\hspace{\parindent}
図-\ref{fig:fig0}に示すように,ユニットセル$\omega$は一辺の長さが$W$の正方形領域とし,
多孔質体の間隙率を$n$,飽和度を$S_r$で表す.
ここで,液相領域を$\Omega_f$,気相領域を$\Omega_a$とすれば,$n$と$S_r$は
\begin{equation}
	n=\frac{\left|\Omega_f\cup\Omega_a\right|}{\left| \Omega \right|}, \ \ 
	S_r=\frac{\left|\Omega_f\right|}{\left|\Omega_f\cup\Omega_a\right|}
	\label{eqn:def_n_Sr}
\end{equation}
で与えられる.ユニットセル内での位置を表す$xy$直交座標は,ユニットセルの各辺と平行に,
図\ref{fig:fig0}に示すようにとる.ただし,数値解析条件を設定する際には,これらをセルサイズ
$W$で無次元化した座標
\begin{equation}
	X=\frac{x}{W}, \ \
	Y=\frac{y}{W}
	\label{eqn:XYcod}
\end{equation}
を用いる.
%なお,マクロ拡散係数$K$の定義は次節で述べる.
なお,マクロ拡散係数$K$の精確な定義は,
次節,第4項において,ランダムウォーク結果を用いたマクロ拡散係数の数値的な
評価方法とともに述べる.
\begin{figure}[t]
\begin{center}
	%\includegraphics[clip,scale=0.45]{Figs/fig0.eps}
\caption{
	周期不飽和多孔質体のユニットセル$\Omega$.
}
\label{fig:fig0}
\end{center}
\end{figure}
\begin{figure}[t]
\begin{center}
%\includegraphics[clip,scale=0.45]{Figs/fig1.eps}
\caption{
数値シミュレーションのプロセス.
	(a),(a')固相粒子のランダムパッキング,
	(b),(b')液相配置の決定,
	(c),(c')正方格子上のランダムウォーク.
}
\label{fig:fig1}
\end{center}
\end{figure}
\section{数値解析法}
ランダムウォークによる拡散解析を行うにあたり,不飽和多孔質媒体の間隙構造を表現した形状モデルが必要となる.
形状モデルの作成には,著者ら\cite{Kimoto}が提案した,モンテカルロ・シミュレーションによる方法を用いる.
この方法では,固相粒子をユニットセル内にランダム充填して多孔質構造を作成し,
その後,間隙領域に所定の飽和度となるように液相を配置する.
固相粒子のパッキングと液相領域の配置決定は,それぞれの目的にあわせて定義したエネルギー
$E$の最小化問題として定式化し,モンテカルロ・シミュレーションで最適解を探索する.
いずれも,比較的少数の入力データを与えるだけで計算を実行することができ,
多数の不飽和多孔質体モデルを簡単かつ自動的に生成できることがこのアプローチの利点である.
以下ではランダムパッキングと間隙水配置決定のためのモンテカルロ・シミュレーションについて概説し,
続いて,ランダムウォークによる拡散解析法とマクロ拡散係数の決定方法を説明する.
%%%%
\subsection{固相粒子のランダムパッキング}
任意の形状をもつ剛体固相粒子をユニットセル内に充填する問題を考える.
個々の粒子の形状と粒子群の粒径分布は既知とし,ユニットセル内に
含まれる固相粒子数$N_p$は所定の間隙率となるように与えておく.
固相粒子の初期配置は,図\ref{fig:fig1}-(a)に示すようにランダムに設定する.
その結果,多数の粒子が初期状態では互いに重なって配置されるため,
粒子位置と向きを繰返し変更し,粒子間に重なりが生じない配置(図-\ref{fig:fig1}の(a'))をメトロポリス法に基づくモンテカルロ・シミュレーション(MC)\cite{comp_phys}と
焼きなまし法(SA:simulated annealing)で探索する.
ここで,ユニットセルに充填する$N_p$個の固相粒子のうち,第$i$番目の粒子が占める
領域を$A_i$,その面積を$\left| A_i \right|$と表す.
このとき,異なる2つの粒子が重なった部分の面積を合計した値は,
\begin{equation}
	E=\sum_{i,j=1, i\neq j}^{N_p} \left| A_i\cap A_j \right|
	\label{eqn:overlap}
\end{equation}
で与えられる. そこで,式(\ref{eqn:overlap})で与えられる$E$を,
モンテカルロ・シミュレーションにおけるエネルギーとして用い,
$E$が0となるような固相粒子の配置を探索する.
モンテカルロ・シミュレーションでは,任意に選択された固相粒子の
位置と向きを,仮想的に変更した場合に生じるエネルギーの変化$\Delta E$を計算する.
その結果,$\Delta E<0$であれば無条件に,$\Delta E \geq 0$
であれば,
\begin{equation}
	P=\exp\left( -\frac{\Delta E}{k_BT}\right)
	\label{eqn:Boltzmann}
\end{equation}
によって与えられる確率$P$で粒子配置実際に変更して系の状態を更新する.ここに,$k_B$はボルツマン定数を,
$T$は最適化のための仮想温度を表す.温度$T$は,全ての粒子について位置と向きの
変更を検討して状態更新あるいは維持を行った後,所定の量$\Delta T(<0)$だけ低下させる.
温度$T+\Delta T$において同じ手順で系の状態更新を行い,この作業を$E$が十分に小さくなるまで
繰り返す.なお,間隙率$n$があまり大きく無い場合,$E$が完全にゼロとなる粒子配置を
許容し得る計算時間で見出すことは難しい.そこで本研究では,エネルギー$E$と,達成すべき
固相領域の面積$(1-n)W^2$比が0.1$\%$以下となった時点で計算を終了することとした.
また,固相粒子間で共有される領域面積$\left| A_i \cap A_j \right|$の評価は,
固相粒子をピクセルデータで表現し,2つの粒子$i$と$j$に共有されるピクセル数を
カウントすることによって行った.
%%%%%%%%%
\subsection{間隙水配置の決定\cite{MC,Kimoto}}
間隙水(液相)の配置は,気相−液相,固相−液相,固相−気相界面の,全界面エネルギー$E$が
停留値を取るように決定する.いま,$\alpha$相と$\beta$相が接するときの界面自由エネルギー
を$\gamma_{\alpha \beta}$とすれば,全界面エネルギーは,
\begin{equation}
	E=\sum_{\alpha,\beta, \alpha\neq\beta}
	\int_{\partial A_{\alpha\beta}}\gamma_{\alpha \beta}dS
\end{equation}
で与えられる.ただし,$\partial A_{\alpha\beta}$はユニットセルに含まれる
$\alpha$相と$\beta$相の界面を表す.液相の配置を数値解析モデルにおいて表現するためには,
図-\ref{fig:fig1}の(b)に示すように,間隙領域を一辺が$h$の小さな正方形セルに分割し,
各セルが液相あるいは気相いずれの状態にあるかを定める.液相の初期配置は,
この図にあるようランダムに設定する.ただし,液相状態にあるセルの総数は,
指定された飽和度となるようにとる.
ここで,セル総数を$M$,第$i$セルの状態と界面エネルギーをそれぞれ$\alpha_i, E_i$で表し,
第$i$セルを取り囲む8つの近傍セルの番号を集めた集合を$I(i)$とすれば,$E_i$は
\begin{equation}
	E_i=\sum_{j\in I(i)}\gamma_{\alpha_i \alpha_j}\Delta s
	\label{eqn:E_surf}
\end{equation}
で計算できる.また,全界面エネルギ$E$は,その和として
\begin{equation}
	E=\sum_{i=1}^{M}E_i
	\label{eqn:Etot}
\end{equation}
で与えられる.なお,式(\ref{eqn:E_surf})の$\Delta s$は,隣接するセル界面の面積を意味し,
同相のセル境界では$\gamma_{\alpha\alpha}=0$と解釈する.モンテカルロ・シミュレーションでは,
ランダムに選択した液相および気相セルのペアで,互いの状態を仮想的に交換したときの
界面エネルギーの変化$\Delta E$を求める.$\Delta E$が負となる場合は,
液相と気相の状態をピクセル間で実際に交換し,$\Delta E \geq 0$の場合は,
ボルツマン分布(\ref{eqn:Boltzmann})で与えられる確率$P$で状態交換を行うか否かを判定する.
このような状態更新を全ての液相セルについて温度$T$一定のもとで行った後,
所定量だけ温度を下げ,同じ手順で状態更新を繰り返す.この作業を界面自由エネルギー$E$に
変化が見られなくなるまで行うことで最終的な液相の配置を決定する.
以上の計算過程で気相と液相セルそれぞれの数は変化しないため,
初期状態で設定した所望の飽和度もつ不飽和多孔質体モデルを作成することができる.
なお,Luら\cite{MC}は,このような方法で間隙水分布を決定した結果,固相粒子の接触部に
形成されるメニスカスの曲率半径がヤング-ラプラスの式\cite{SurfChem}で与えられる
曲率半径とよく一致することや,CTスキャンで取得した間隙水分布とシミュレーション結果が
良好に一致することを示している.中島ら\cite{Kimoto}は,同様な手法を用い,固相表面が
親水的な場合と撥水的な場合のシミュレーションを行い,設定した通りの濡れ角で気泡や液滴
が形成されることを示している.このように,本研究で用いる水分配置決定方法は,
既往の研究によってその妥当性が示されているものである.
\subsection{ランダムウォーク}
液相中の拡散物質が分子拡散によってランダムに移動する様をランダムウォークでシミュレートする.
このとき,ランダムウォーカーの物理的意味は,拡散物質を粗視化して表現した粒子であることから,
以下ではランダムウォーカーを拡散粒子と呼ぶことにする.なお,拡散粒子の数密度は,
拡散物質の濃度$C$に比例すると解釈でき,$C$が通常の拡散方程式に従う場合は
通常拡散,そうで無い場合は異常拡散と呼ばれる.\\
\hspace{\parindent}
本研究で行うランダムウォークシミュレーションでは,液相セルの中心点をノードとする正方格子上で
拡散粒子を移動させる.拡散粒子の移動は,時間ステップ$\Delta t$毎に,隣接する上下,
左右いずれかのノードへ指定された確率で行う.
図-\ref{fig:fig1}の(c)に示したように,4つの隣接ノードへの移動確率を$p_1\sim p_4$で,
黒の点で示した現在のノードにとどまる確率を$p_0$と表せば,
隣接セル全てが液相の場合は,分子拡散が等方的であることを踏まえ
\begin{equation}
	p_1=p_2=p_3=p_4=\frac{1}{4}, \ \ p_0=0
	\label{eqn:iso_p}
\end{equation}
とする.一方,いずれかの隣接セルが固相あるいは気相の場合は,それらのセル中心ノードへ
の移動確率を0とし,現在位置に滞留する確率$p_0$を$\sum_{i=0}^4p_i=1$で与える.
例えば,右と上の隣接セルが液相でない場合,
\begin{equation}
	p_2=p_3=0, \ \ p_1=p_4=\frac{1}{4}
\end{equation}
とし,現在位置にとどまる確率$p_0$を,
\begin{equation}
	p_0=1-\sum_{i=1}^4 p_i=\frac{1}{2}
\end{equation}
とする.図-\ref{fig:fig1}の(c')は,ノード間の移動確率を上記のように定めて行った
ランダムウォークの結果(一部)を示したものである.赤の実線が拡散粒子の移動経路を,黒の点が
その過程で経由したノードを表している.\\
ユニットセル全体が液相で占められる場合,液相の自己拡散係数$D_0$は,ランダムウォークの
時間ステップ$\Delta t$とノード間隔$h$,隣接ノード間の移動確率$p_i=p=\frac{1}{4},\,
(i=1,\dots 4)$により
\begin{equation}
	D_0=\frac{ph^2}{\Delta t}=\frac{h^2}{4\Delta t}
	\label{eqn:D0}
\end{equation}
で与えられることが,ランダムウォーク解析に関する初等的な結果として知られている\cite{Toda}.
従って,$D_0$と$h$を与えれば,時間ステップ$\Delta t$は,次の式で決めることができる.
\begin{equation}
	\Delta t=\frac{h^2}{4D_0}
	\label{eqn:dt}
\end{equation}
後に示す数値シミュレーションでは,多数のランダムウォーカーをユニットセル内に配置し,
予め指定した時間ステップ$N_t$までランダムウォークを行う.
その際,各時間ステップにおける拡散粒子の平均2乗変位を記録し,次項に述べる方法
でマクロ拡散係数を求める.
\subsection{マクロ拡散係数の評価}
ランダムウォークに用いる拡散粒子の総数を$N_{wk}$,
そのうち第$i$番目の粒子に関する,時刻$t$における初期位置からの変位を
\begin{equation}
	\mathbf{u}_i(t)=\left(u_i(t),v_i(t)\right)
	\label{eqn:u_i}
\end{equation}
と表す.拡散粒子の平均2乗変位を
\begin{equation}
	\left< u_i^2 \right>=
	\frac{1}{N_{wk}} \sum_{i=1}^{N_{wk}}u_i^2
	, \ \ 
	\left< v_i^2 \right>=
	\frac{1}{N_{wk}} \sum_{i=1}^{N_{wk}}v_i^2
	\label{eqn:u2b}
\end{equation}
と書くとき,拡散係数と平均2乗変位の間には
\begin{equation}
	\left< u_i^2 \right>=2 D_{x} t, \ \ 
	\left< v_i^2 \right>=2 D_{y} t
	\label{eqn:Einstein}
\end{equation}
の関係がある.ここで$D_x$と$D_y$は,それぞれ$x$方向,$y$方向への拡散係数を表す.
よって,ランダムウオークシミュレーションの結果から$\left<u_i^2\right>$
や$\left<v_i^2\right>$の時刻歴を求めて式(\ref{eqn:Einstein})
でフィッティングすれば,拡散係数$D_x,D_y$を得ることができる.
ユニットセル全体が液相で占められる場合,その結果は$D_x=D_y=D_0$となるが,
不飽和多孔質体の場合,固相配置と間隙水分布に応じて拡散経路が制限されるため
$D_x,D_y<D_0$となる.$D_x$や$D_y$は,多孔質体全体としての拡散係数を表すため,
間隙スケールをミクロ,ユニットセルスケールをマクロスケールと考え,
$D_x,D_y$をマクロ拡散係数と呼ぶ.また,拡散が等方的である場合,両者をまとめて
\begin{equation}
	D=D_x=D_y
	\label{eqn:D_iso}
\end{equation}
と書くことにする.この場合,$u_i$と$v_i$の平均2乗変位も一致することから,
両者を区別することなく
\begin{equation}
	\left<u^2\right>
	=
	\left<u_i^2\right>
	=
	\left<v_i^2\right>
	\label{eqn:}
\end{equation}
と書く. 平均2乗変位の時間変化が式(\ref{eqn:Einstein})に従う場合,拡散物質の濃度$C$は拡散方程式
\begin{equation}
	\frac{\partial C}{\partial t}=D\nabla ^2 C
	\label{eqn:diff_eq}
\end{equation}
に従う.一方,マクロな拡散が式(\ref{eqn:diff_eq})に従わない事例が少なくないことは
既往の研究\cite{Rubin}$^-$\cite{Upscaling_review}に報告されており,その場合,平均2乗変位の時刻歴はべき関数
\begin{equation}
	\left< u_i^2 \right>=
	\left< v_i^2 \right>=2 K t^\alpha
	\label{eqn:abnormal}
\end{equation}
によってより精度良く近似される.式(\ref{eqn:abnormal})の係数$K$は,
一般化された拡散係数を表し,べき指数$\alpha$が1のときには通常の拡散
係数に一致する\cite{Rwk_textbook}.一方,$\alpha \neq 1$の場合は,異常拡散と呼ばれ
$\alpha < 1$は"遅い拡散"を,$\alpha>1$は"速い拡散"と呼ばれる.
つまり,$K$は拡散のスケールを,$\alpha$は拡散のタイプを表す指標となっている.
ただし,本研究のモデルで拡散が促進される要因は無く,$\alpha$は常に1以下である.
以下では,平均2乗変位の時刻歴から$K$と$\alpha$を求め,間隙構造と水分量が
これらの係数に与える影響を調べる.
%%

\section{位相分布の再構成}
	%%%%%%%%%%%%%%%%%%%%%%%%%%%%%%%%%%%%%%%%%%%%%%%%%%%%%%%%%%%%%%%%%%%%%%%
\subsection{多孔質体モデル}
図-\ref{fig:fig3}に,ランダムウォーク・シミュレーションに用いる4種類の
多孔質体構造(モデルタイプ1$\sim$4)を示す.これらは固相粒子配置の規則性,粒径分布,
粒子形状がマクロ拡散挙動に与える影響を調べることを意図したものである.
図-\ref{fig:fig3}でグレーの部分は固相を,黄色の部分は間隙を表し,
間隙水が存在しない$(S_r=0)$状態での多孔質体モデルを示している. 
モデルタイプ1は,直径$\frac{W}{10}$の円形粒子を,ユニットセル内で10$\times$10の正方格子状に配置したもので,
飽和度が拡散係数に与える影響や,拡散係数の大小を議論する際の参照モデルの役割を果たす.
モデルタイプ2は,直径$\frac{W}{10}$の円形粒子100個をユニットセル内にランダムに配置したものである.
粒子直径と粒子数$N_p$はモデルタイプ1と同じであることから,間隙率$n$も互いに等しく,
その値は
\begin{equation}
	n=1-\frac{\pi}{4}\simeq 0.215
	\label{eqn:n_val}
\end{equation}
である.一方,モデルタイプ3と4は,粒径分布の影響をみるためのもので,
タイプ3は円形粒子を,タイプ4はアスペクト比$a$が$0.4\sim 1.0$の楕円粒子を
充填したものである.なお,アスペクト比$a$は一様分布で,粒子直径はガンマ分布に従って
ランダムに与える.ただし,いずれのモデルも間隙率$n$は式(\ref{eqn:n_val})程度と
なるよう粒子数を設定した.また,粒子直径は最小値を$\frac{W}{100}$最大値を$\frac{W}{5}$とし,
平均粒子直径が約$0.085W$,粒子直径の標準偏差が約$0.05$,粒子数$N_p$が100前後と
なるようにした.さらに,モデルタイプ2$\sim$4は不規則な充填構造であるため,
固相粒子の粒径や数,配置に応じてマクロ拡散係数値も変動すると予想される.
そこで,これら3種類のモデルタイプについては,粒子位置や粒径が異なる20個のモデルを作成し,
各々のモデルについて複数の飽和度でランダムウォークによる拡散シミュレーションを行った.
以上,タイプ1から4のモデルに関する
特徴を表\ref{tbl:types}にまとめて示す.
\begin{table}[htb]
  \caption{多孔質構造(タイプ1$\sim$4)の特徴}
%{\small
  \begin{tabular}{c|c|c|c|c}
\hline
    タイプ & 1 & 2 & 3 & 4 \\
	\hline\hline
	  充填構造& 単純立方& \multicolumn{3}{c}{ランダム} 
%	  & ランダム &ランダム 
	  \\
    \hline
粒子形状 & 円 &円 & 円 & 楕円\\
	\hline
   平均粒径 & 0.1 & 0.1 & 0.089 & 0.082\\ 
 (標準偏差) & (0) & (0) & (0.047) & (0.050) \\
\hline
粒子数 & 100 & 100 & 94$\sim$119 & 81$\sim$145 \\
(平均) & (100) & (100) & (107) & (111) \\
\hline
モデル数& 1 &20 &20 & 20 \\
\hline
  \end{tabular}
\label{tbl:types}
%}
\end{table}
\begin{figure}[t]
\begin{center}
%\includegraphics[clip,scale=0.4]{Figs/fig2.eps}
\caption{
	数値シミュレーションに用いた4種類の多孔質構造モデル. 
}
\label{fig:fig3}
\end{center}
\end{figure}
%%%%%%%%%%%%%%%%%%%%%%%%%%%%%%%%%%%%%%%%%%%%%%%%%%%
\subsection{間隙水の配置}
モデルタイプ1$\sim$4の多孔質構造モデルそれぞれについて,飽和度$Sr$を0.2から1.0まで0.1刻みで
変化させた不飽和多孔質体モデルを作成する.なお,タイプ1については,固相粒子の配置は一通りに
決まっているため,間隙水の分布だけが異なる20通りのモデルを各飽和度で作成した.
すなわち,4種類のモデルタイプ,9段階の飽和度で各20個の多孔質体モデルを用い,
合計720ケースのランダムウォーク・シミュレーションを行う.
各多孔質体モデルは,水分配置を十分な空間解像度で表現するため,ユニットセルの一辺を1,024分割し,
セルサイズを$h=\frac{W}{1,204}$とした.
液相配置決定のためのモンテカルロ・シミュレーションで必要となる界面自由エネルギー$\gamma_{\alpha\beta}$の値は,
$\alpha=1$で気相を,$\alpha=2,3$でそれぞれ液相と固相を表すとき,
\begin{equation}
	\gamma_{12}=70.0,\,  \gamma_{23}=370.0, \, \gamma_{31}=420.0, \ \ 
	[{\rm mJ/m}]
	\label{eqn:gamma_val} 
\end{equation}
とした\cite{Berkowitz}.これは,気相として空気を,液相として水を,固相として石英を想定したもので,
固相表面の濡れ角\cite{Gennes}が$45$度となる親水的な表面に相当する.
なお,界面自由エネルギーの最小化では,$\gamma_{\alpha\beta}$の値は互いの比だけが影響し,
近傍セルの接触面積$\Delta s$や$\gamma_{\alpha\beta}$の絶対値には依存しない.
以上の計算条件に対して得られた不飽和多孔質体モデルの一例を図\ref{fig:fig4}に示す.
これは,モデルタイプ2で,飽和度を$Sr=0.2,0.4,0.6$および$0.8$としたときの結果を示したもので,
飽和度が低い場合には固相粒子間の狭隘部に間隙水が集中し,飽和度が比較的高い場合には
固相粒子直径程度の気泡を残しつつ,次第に広い間隙部分を水分が埋めていく様子が現れている.
\begin{figure}[t]
\begin{center}
%\includegraphics[clip,scale=0.4]{Figs/fig3.eps}
\caption{
	4つの異なる飽和度における間隙水分布の様子.
	モデルタイプ2,飽和度$Sr$が(a) 0.2,(b) 0.4,(c) 0.6,(d) 0.8の場合を示す. 
}
\label{fig:fig4}
\end{center}
\end{figure}
%%%%%%%%%%%%%%%%%%%%%%%%%%%%%%%%%%%%%%%%%%%%%%%%%%%
\subsection{ランダムウォークとマクロ拡散係数の評価}
%本研究では,ランダムウォークを液相セルの中心をノードとする格子上で行うため,
%空間ステップ長は液相セルのサイズ$h=\frac{W}{1,024}$に等しい. 
ユニットセル内各点からの影響が反映された拡散特性を調べるためには,
液相内にある全てのランダムウォークノードを,いずれかの拡散粒子が
繰り返し訪れることができるように拡散粒子数$N_{wk}$や時間ステップ$N_t$を設定する
必要がある.本研究で用いる多孔質体モデルは,ユニットセル中には,およそ100個の固相粒子がある.
そこで,各固相粒子の周辺におよそ100個の拡散粒子が常時存在する状況が生じるように, 
ここでは1万個のランダムウォーカーをユニットセル中に一様に分布させた状態から
ランダムウォーク・シミュレーションを行った.
また,適切な時間ステップ数$N_t$を設定するために,
式(\ref{eqn:abnormal})を無次元化し,以下に示す無次元化時間を元に$N_t$を与えた.\\
式(\ref{eqn:abnormal})には時間と長さの次元を持つ量が含まれる.
そこで,ユニットセルサイズ$W$を長さの基準にとり,変位成分$u_i,v_i$を
\begin{equation}
	U_i=\frac{u_i}{W},\ \  V_i=\frac{v_i}{W}
\end{equation}
と無次元化する.時間に関しては,
\begin{equation}
	\tau=\frac{D_0}{W^2}t
	\label{eqn:tau}
\end{equation}
により,液相の拡散係数$D_0$のスケールにあった無次元化時間を導入する.
このとき式(\ref{eqn:abnormal})を
\begin{equation}
	\left<U_i\right>=\left<V_i\right>=2\bar K \tau ^{\alpha}
	\label{eqn:Kb}
\end{equation}
と書けば,一般化拡散係数$\bar{K}$は
\begin{equation}
	\bar{K}=\frac{K}{D_0W^{2(1-\alpha)}}
	\label{eqn:Kb_def}
\end{equation}
で与えられる無次元量となる,
ここで,単位時間$\tau=1$は$t=W^2/D_0$に相当し,この時刻を式(\ref{eqn:Einstein})に代入すれば
\begin{equation}
	\left<u_i^2\right> =\frac{2D}{D_0}W^2
\end{equation}
となる.従って,$D$と$D_0$のオーダーが大きく異ならない限り,$\tau=1$の時点では,
ユニットセル内で拡散が十分進行した状態にあると言える.
またこのことは,ユニットセルにおける拡散挙動は,$\tau<1$程度の時間スケールで調べれば
十分であることを意味する.そこで,以下のシミュレーションでは$\tau=1/4$となる
時間ステップまでランダムウォークを行うこととする.
$\tau=1/4$に達するための時間ステップ数は,式(\ref{eqn:dt})と式(\ref{eqn:tau})より,
\begin{equation}
	N_t=\left. \frac{t}{\Delta t} \right|_{\tau=1/4}=\frac{W^2}{h^2}=1,048,576
\end{equation}
と与えられる.\\
\hspace{\parindent}
以上の設定で,無限に広い液相領域における拡散解析をランダムウォークで行った
結果を図-\ref{fig:rwk0}に示す.
この図において,(a)は無次元化した平均2乗変位$\left<U^2\right>$と時間$\tau$の関係を,
(b)は$\tau=0.23$における拡散粒子の変位分布を表している.
なお,図\ref{fig:rwk0}-(b)に示されるように,拡散はほぼ等方的であることから,
$X$方向変位$U_i$,$Y$方向変位$V_i$を区別せずに平均した,
\begin{equation}
	\left< U^2 \right>=\frac{
			\left<U_i^2 \right>
		+
		\left<V_i^2 \right>
	}{2}
	=\sum_{i=1}^{N_p}
	\frac{U_i^2+V_i^2}{2N_p}
	\label{eqn:Ubar}
\end{equation}
を,図\ref{fig:rwk0}-(a)では示している.この結果から一般化拡散係数$\bar K$と
べき指数$\alpha$を求めると,それぞれ
\begin{equation}
	\bar{K}=0.9996, \ \ \alpha=1.0000
\end{equation}
となる.この問題では拡散の妨げとなる固相や気相領域が存在しないことから,
マクロ拡散係数$D$とミクロ拡散係数$D_0$は同じである.
従って,$\bar K=1, \alpha=1$が正解であり,ランダムウォークでの拡散解析によって
非常に近い値が得られていることが分かる.
また,図\ref{fig:rwk0}-(b)の
変位分布について,共分散行列を
\begin{equation}
	\fat{S}( \{U_i\},\{V_i\})
	=\left(
	\begin{array}{cc}
		S_{UU} & S_{UV}	 \\
		S_{VU} & S_{VV} 	
	\end{array}
	\right)
\end{equation}
として,各分散値を計算すれば,
\begin{equation}
	\bar{S}= \frac{S_{UU}+S_{VV}}{2}=0.4604
\end{equation}
\begin{equation}
	\fat{S}( \{U_i\},\{V_i\})
	=
	\bar{S}
	\left(
	\begin{array}{cc}
		 0.9897 & -0.0008 \\
		-0.0008 &  1.0103
	\end{array}
	\right)
\end{equation}
であった.このように,対角項は1$\%$程度の差で一致し,非対角項はそれよりも3桁程度は小さく
ほぼ等方的な拡散が再現されていることが分かる.以上のように,拡散場は等方的で平均2乗変位の
グラフが滑らかに推移し,マクロ拡散係数も正しく得られていることから,拡散粒子数も十分で
あったと考えられる.
\begin{figure}[t]
\begin{center}
%\includegraphics[clip,scale=0.32]{Figs/fig10.eps}
\caption{
	ランダムウォークによる無限液相領域における拡散解析の結果.
	(a) 平均2乗変位の時間変化. (b)$\tau=$0.23における拡散粒子の変位分布. 
}
\label{fig:rwk0}
\end{center}
\end{figure}


\section{データ解析結果と考察}
%	本節では,参照モデルとしての役割を果たすモデルタイプ1に対するシミュレーション
結果をはじめに,次に,不規則な充填構造をもつモデルタイプ2,3および4に対する
結果を,モデルタイプ1と比較する形で示す.
それらの結果を踏まえ,固相粒子形状や充填構造,粒径分布が
マクロ拡散係数や異常拡散の程度に与える影響を調べる.
%
\subsection{モデルタイプ1に対する結果}
図-\ref{fig:fig5}に,ランダムウォーク・シミュレーションで得られた
平均2乗変位の時間変化を示す.このグラフは,横軸を無次元化時間$\tau$とし,
縦軸を$W$で無次元化した拡散粒子の平均2乗変位としたもので,
$S_r$=0.2から1.0まで,異なる9段階の飽和度に対する結果を示している.
先に述べたように,各飽和度での計算は20通りの異なる水分配置に対して行っており,
図-\ref{fig:fig5}はそれら全ての結果を合わせて算出した平均2乗変位を表している.
%
%なお,平均2乗変位の算出に先立ち,拡散粒子変位の$XY$平面内における分布を調べたところ,
%目立った配向性はなく等方的であった.そこで,$X$方向変位$U_i$,
%$Y$方向変位$V_i$を区別せずに平均した,
%\begin{equation}
%	\left< U^2 \right>=\frac{
%			\left<U_i^2 \right>
%		+
%		\left<V_i^2 \right>
%	}{2}
%	=\sum_{i=1}^{N_p}
%	\frac{U_i^2+V_i^2}{2N_p}
%	\label{eqn:Ubar}
%\end{equation}
%を平均2乗変位として図\ref{fig:fig5}に示した.
このグラフから明らかなように,いずれの飽和度でも$\left<U^2\right>$は時間に対して増加するが,
飽和度が$S_r=0.2$から0.5程度の場合,$\left<U^2\right>$の値は途中で頭打ちとなる.
これは,互いに分離した液相領域が多数存在するために,有限な液相領域のサイズを超えて変位が
増加できないためである.一方,$S_r>0.5$のときには,$\left<U^2\right>$は$\tau$に対して
単調に増加を続ける.このことは,飽和度$S_r$の増加に伴い孤立していた液相領域が連結して,
ユニットセルをパーコレートする拡散パスが次第に形成されることを表している.\\
\hspace{\parindent}
図-\ref{fig:fig5}の結果に式(\ref{eqn:Kb})を最小2乗法でフィッティングし,
拡散係数$\bar{K}$とべき指数$\alpha$を求めた結果を,それぞれ図-\ref{fig:fig6}と
図-\ref{fig:fig7}に示す.$\bar{K}$は$S_r<0.4$ではほぼゼロで,
概ね$S_r=0.4$をしきい値として増加をはじめ,最終的に間隙が完全に飽和した$S_r=1.0$では
マクロ拡散係数の値が0.4程度となっている.
つまり,飽和状態でマクロ拡散係数は,液相の拡散係数$D_0$の4割程度となり,
間隙率が約21.5\%であることを考慮すれば,ユニットセルに占める拡散相の比率よりも
大きな値であると言える.このことは,多孔質体のマクロ拡散係数を評価する際,
拡散相と非拡散相(ここでは気相と固相)の割合だけでなく,
間隙形状の効果を考慮することが必要であることを意味している.
一方,図-\ref{fig:fig6}に示したべき指数$\alpha$の挙動を見ると,
$\alpha$は$S_r=0.3$から顕著に増加し,$S_r=1.0$でのみ$\alpha=1$となっている.
すなわち,飽和度に応じた程度の差はあるものの,不飽和状態では常に遅い異常拡散が起こること,
通常拡散となるのは飽和状態の場合に限られることが示されている.
\begin{figure}[t]
\begin{center}
%\includegraphics[clip,scale=0.50]{Figs/fig4.eps}
\caption{
	平均2乗変位$\left<U^2\right>$の時間変化.
	モデルタイプ1,飽和度$S_r=0.2\sim 1.0$に対する結果.
}
\label{fig:fig5}
\end{center}
\end{figure}
%---------------------------------------------------
\begin{figure}
\begin{center}
%\includegraphics[clip,scale=0.50]{Figs/fig5.eps}
\caption{
	マクロ拡散係数$\bar K$と飽和度の関係(モデルタイプ1対する結果).
	}
\label{fig:fig6}
\end{center}
\end{figure}
%---------------------------------------------------
\begin{figure}
\begin{center}
%\includegraphics[clip,scale=0.5]{Figs/fig6.eps}
\caption{
	べき指数$\alpha$と飽和度の関係(モデルタイプ1に対する結果).
	}
\label{fig:fig7}
\end{center}
\end{figure}
\vspace{-5mm}
%%%%%%%%%%%%%%%%%%%%%%%%%%%
%
\begin{figure}[t]
\begin{center}
%\includegraphics[clip,scale=0.50]{Figs/fig11.eps}
\caption{
	平均2乗変位$\left<U^2\right>$の時間変化.
	モデルタイプ2,飽和度$S_r=0.2\sim 1.0$に対する結果.
}
\label{fig:fig11}
\end{center}
\end{figure}
\begin{figure}[t]
\begin{center}
%\includegraphics[clip,scale=0.50]{Figs/fig12.eps}
\caption{
	平均2乗変位$\left<U^2\right>$の時間変化.
	モデルタイプ3,飽和度$S_r=0.2\sim 1.0$に対する結果.
}
\label{fig:fig12}
\end{center}
\end{figure}
\begin{figure}[t]
\begin{center}
%\includegraphics[clip,scale=0.50]{Figs/fig13.eps}
\caption{
	平均2乗変位$\left<U^2\right>$の時間変化.
	モデルタイプ4,飽和度$S_r=0.2\sim 1.0$に対する結果.
}
\label{fig:fig13}
\end{center}
\end{figure}
\begin{figure}[h]
\begin{center}
%\includegraphics[clip,scale=0.50]{Figs/fig7.eps}
\caption{
	マクロ拡散係数$\bar K$と飽和度の関係(モデルタイプ1$\sim$4に対する結果).
	}
\label{fig:fig8}
\end{center}
\end{figure}
%
\begin{figure}[h]
\begin{center}
%\includegraphics[clip,scale=0.5]{Figs/fig8.eps}
\caption{
	べき指数$\alpha$と飽和度の関係(モデルタイプ1$\sim$4に対する結果).
	}
\label{fig:fig9}
\end{center}
\end{figure}
\subsection{モデルタイプによる拡散挙動の違い}
図-\ref{fig:fig11}$\sim$図-\ref{fig:fig13}に,タイプ2$\sim$4のモデルに対する
平均2乗変位の時間変化を示す.それぞれのグラフは,図-\ref{fig:fig5}と同様,
横軸が無次元化時間$\tau$,縦軸は平均2乗変位$\left<U^2\right>$としたもので,
計算を行った全ての飽和度$S_r$に対する結果が示されている.
これらタイプ2$\sim$4に対する結果は互いによく似た形状の曲線群となっている.
一方,タイプ1との比較でみると,時間に対して直線的に変化するケースがタイプ2$\sim$4
では見当たらず,程度の差はあれ,いずれも異常拡散となっていることが示唆されている.\\
\hspace{\parindent}
平均2乗変位の時間推移から決定した,タイプ1$\sim$4のモデルに対する
マクロ拡散係数$\bar{K}$とべき指数$\alpha$を,それぞれ,図-\ref{fig:fig8}と
図-\ref{fig:fig9}に示す.
図-\ref{fig:fig8}にあるように,拡散係数$\bar{K}$はモデルタイプによらず
飽和度$S_r$に対して類似した関数形で単調増加する.
ただし,不規則な充填構造を持つモデルタイプ2と3および4は,モデルタイプ1に比べて
より拡散係数の値が小さい.これは,モデルタイプ1では固相粒子が同一直線上
に並んでいるため,間隙に十分な水分がある場合,拡散粒子が直線的な経路でユニットセル
を横断あるいは縦断できるためと考えられる.
これに対し,不規則な充填構造を持つモデルでは,拡散粒子物が不規則に並んだ
固相粒子を迂回しながら移動し,拡散経路が常に屈曲することからマクロ拡散係
数値は小さくなり,べき指数も1に達することなく常に遅い拡散となっている.
% --- 追加
ただし,タイプ1のべき指数$\alpha$は,$Sr<0.4$の低飽和度の側では
他のモデルと比べて若干小さな値となっている.
これは,タイプ2$\sim$4では大小様々な粒径の固相粒子を迂回して拡散粒子が
移動するのに対し,タイプ1で水分が少ない場合は,拡散粒子が常に同一粒径
の固相粒子表面近傍を這うように迂回しながら移動する必要があることに起因する.
%
次に,モデルタイプ3と4の結果を比較すると,$\bar{K},\alpha$とも大きな
違いはなく,平均粒径や間隙率,飽和度が同程度であれば,粒子形状がマクロな
拡散に与える影響は小さいことが分かる.
一方,モデルタイプ2では,べき指数$\alpha$の挙動はその他不規則充填構造モデル
(タイプ3,4)と大差ないものの,
マクロ拡散係数は$S_r>0.8$程度の飽和度において他と比べ明らかに小さい.
これは以下の理由によると考えられる.
%
粒子配置がランダムな場合,水分量が多いときでも間隙のネットワークが屈曲し,
拡散経路が長くなる. さらに,均一粒径の固相粒子を充填した場合,
広い粒径分布を持つ粒子を充填したときに比べ,互いに接触した粒子が長いネットワークを作り易い.
つまり,拡散のボトルネックとなるような狭隘部を迂回する経路が見つかりにくくなる.
タイプ2のモデルでは,これら2つの効果が相まって,拡散係数が他のモデルよりも
小さくなると考えられる.
%
%均一粒径の粒子を容器内に密に充填することが,広い粒径分布をもつ粒子を
%同じ容器に充填することに比べてより困難なことは,日常経験からも明らかである.
%このことは,互いに接触して密に配列した粒子のネットワークが,容器を横断あるいは縦断するように発達し易いことを意味する.
%従って,均一粒径の粒子が不規則に充填されているとき,拡散物質は
%固相粒子が密に並んだ箇所を簡単には迂回できず,変位の増加が抑制される.
%この結果,マクロ拡散係数の値がモデル2では小さくなった原因と考えられる.
%%単一粒径の粒子を不規則な配置で充填した場合,粒子の一部は平均より密に,
%%残る部分は疎に充填されざるを得ない.固相粒子の疎に配置された箇所では,
%%拡散粒子は移動しやすく,密に固相粒子が配置された領域では移動が抑制される.
%%従って,固相粒子が密に配置された領域を迂回する拡散経路が無い限り,
%%局所的な拡散係数の増加は平均2乗変位の増加,すなわちマクロ拡散係数の
%%増加に貢献しない.
%%以上の挙動を整理すると,モデルタイプ1では,拡散経路が液相領域に制限される
%%ことでマクロ拡散係数は,液相の拡散係の1/2程度まで低減される.
%%モデルタイプ2から4では,
%%拡散経路の屈曲による効果が加わることでより拡散係数が低下する.
%%さらに,単分散粒子系を不規則に充填したタイプ2では,局所的な間隙率の変動に起因した
%%拡散物質の移動抑制効果が現れる.さらに,べき指数がモデルタイプ2から4では1
%%に達しないことから判断して,拡散経路の屈曲が遅い異常拡散の原因となることがわかる.
\subsection{ランダムウォーカー変位の確率密度分布}
最後に,ランダムウォーカー変位の確率密度分布について調べた結果を
図-\ref{fig:fig10}に示す.これは,時刻$\tau=0.25$における
%変位成分$U_i$と$V_i$を,ヒストグラムを正規化して示したものであり,
水平変位$U_i$と鉛直変位$V_i$を併せてヒストグラム化したもので,
縦軸は正規化されており確率密度とみなすことができる.
これら4つのプロットのうち(a)と(b)は,モデルタイプ1に対する結果を
(c)と(d)はモデルタイプ4に対する結果を表す.
それぞれ飽和度は$S_r=$0.7と1.0で,$S_r=0.7$のときの拡散係数$\bar K$は,
$S_r=1.0$の場合に比べていずれのモデルも概ね1/4程度となっている.
なお,青の実線は,変位成分$\left\{U_i,V_i\right\}$と同じ分散をもつ正規分布を示している.
タイプ1に対する結果を表示するにあたり,ヒストグラムの階級幅を,モデルタイプ4のプロット
に比べて意図的に粗く設定している.タイプ1のモデルでは,固相粒子が規則的に配置されているために,
階級幅を固相粒子間距離である$0.1W$よりも小さくすると,確率密度分布に
周期$0.1W$の凹凸が現れ,全体の分布形状がわかりにくくなる.
これを避けるために,図-\ref{fig:fig10}の(a)と(b)では階級幅を
0.67$W$とし,(c)と(d)ではその半分の値としている.\\
\hspace{\parindent}
ここで,図-\ref{fig:fig10}-(b)をみると,
モデルタイプ1では,飽和度$S_r=1.0$のとき確率密度分布がほぼ完全な正規分布となって
いることが分かる.実際,この場合のべき指数は$\alpha=1$であり,
通常の拡散方程式に従う結果であることが確認される.
一方,モデルタイプ4では,(d)のプロットにあるように$S_r=1.0$で概ね正規分布に近い形
となるものの,(b)との比較でみると若干正規分布よりも裾の広い形になっていることに気づく.
このケースでは,べき指数も$\alpha=0.7$と1より小さく,やや遅いタイプの拡散で,
その影響が確率密度分布にも現れていることが分かる.
次に,飽和度が$0.7$の場合について見れば,正規分布からのずれがより
明確となっている.特に(c)のケースでは,確率密度分布が中央で尖った形となっており,
裾野も正規分布に比べて明らかに広い.
このとき$\alpha$は0.5程度の遅い異常拡散となっており,
変位の確率密度も正規分布で表現できないことは明らかである.
以上の結果は,不飽和多孔質媒体のマクロ拡散問題を解析する際,
モデルタイプ1で$S_r=1.0$のような特別な状況を除き,
通常の拡散方程式を用いることことはできないことを意味する.
従って,アップスケーリングされた拡散問題を考える際には,
例えば非整数階微分の拡散方程式を用いることや,
図\ref{fig:fig10}に示されるような変位の確率密度に従うランダムウォーク
を設計する等の対応が必要と考えられる.
%間隙スケールの情報を取り込んだ上で,マクロな拡散解析をランダムウォークで行うことができる.
%その場合,平均2乗変位の時間変化に対する関数形を仮定することなくアップスケーリングを行うことができ,
%同様な手順を繰り返すことで,多段階のアップスケーリングにつなげることが可能となる.
%いずれのアプローチが有効であるかは今後検討すべき課題と.
%
\begin{figure}[h]
\begin{center}
%\includegraphics[clip,scale=0.35]{Figs/fig9.eps}
\caption{
	ランダムウォーカー変位の確率密度分布.青の実線は,分散を一致させたときの正規分布を表す.
	}
\label{fig:fig10}
\end{center}
\end{figure}


\section{まとめ}
本研究では,花崗岩中を伝播する表面波を計測して位相の空間分布を求め,波数ベクトルの確率密度分布を推定した.その結果,媒体の不均質性によって波面が屈曲すること,波数ベクトルの確率密度分布は周波数に強く依存すること,高周波になるにつれ確率密度がブロードになること,波数は非対称かつ有限な幅の非ガウス的な確率分布に従うことが明らかとなった.今後は,波数ベクトルの空間分布構造とその周波数依存性,岩石物性との関係について調べることが,工学的な応用を展開する上での課題となる.

本研究では,2次元不飽和多孔質体モデルを用いたランダムウォークによる拡散解析を行い,
間隙構造や水分量が分子拡散挙動に与える影響について調べた.この方法では,
任意の形状と粒径分布をもつ固相粒子で構成された多孔質体を扱うことができ,
間隙構造や間隙率,飽和度や粒子表面の濡れ性を自由に設定したモデルを用いて
拡散解析を行うことができる.また,ランダムウォークによるシミュレーション結果から,
一般化された拡散係数と,平均2乗変位のべき指数を求めることで,異常拡散の程度も
定量的に評価することができる.本稿では,上記の方法により4種類の多孔質構造モデルに
ついて拡散シミュレーションを行った結果,以下に示す知見を得ることができた.
\begin{itemize}
\item
	多孔質体では,飽和度と間隙率に応じて拡散経路が制限され,
	マクロ拡散係数は液相の拡散係数よりも小さくなる.ただしその低減率は,
	間隙形状の影響を受ける.
\item
	拡散経路が屈曲することにより,マクロスケールでの拡散は遅い異常拡散となり,
	通常の拡散方程式には従わなくなる.
\item
	固相粒子のアスペクト比が0.4$\sim$1.0程度であれば,
	粒子形状が拡散挙動に与える影響は小さい.
\item
	均一な粒径の固相粒子が不規則に充填された多孔質構造では,
	多分散系の固相粒子で構成された多孔質構造に比べて小さなマクロ拡散係数をもつ.
\item
	%平均2乗変位のべき指数で表される異常拡散の程度は,
	%飽和度に対して単調に減少する.	
	平均2乗変位のべき指数$\alpha$は飽和度に対して単調に増加するが,
	通常拡散を意味する1を超えることはない.すなわち,異常拡散の程度は飽和度に対して単調に減少する.
\item
	異常拡散の程度は,固相粒子配置が不規則で飽和度が低い場合に大きい.
	一方,粒径分布や粒子形状が異常拡散の程度に当てる影響は顕著でない.
	このことは,拡散経路の屈曲が異常拡散の主たる原因であることを示唆する.
\end{itemize}
以上は,間隙スケールの形状を考慮した拡散解析を行うことで,不飽和多孔質体に
おける拡散挙動の理解を深めるために有用な情報が得られることを示している.\\
\hspace{\parindent}
今後は,同様な方法を3次元問題に適用し,実験データの解釈に利用することが課題となる.
そのためには,各計算過程(固相粒子のパッキング,間隙水配置の決定,ランダムウォーク)
の効率化を行うことも必要となる.また,本手法を間隙水の移流を考慮した拡散解析に
拡張し,機械的分散のメカニズムを調べること,拡散物質の吸着や拡散物質の
反応を考慮すること,ランダムウォークベースのアップスケーリング手法を考案することも
今後の重要な課題であると考えられる.
%%%%%%%%%%%%%%%%%%%%%%%%%%%%%%%%%%%%%%%%%%%%%%%%%%%%%%%%%%%%%%%%%%%%%%%%%%%%%
%%%%%%%%%%%%%%%%%%%%%%%%%%%%%%%%%%%%%%%%%%%%%%%%%%%%%%%%%%%%%%%%%%%%%%%%%%%%%
%\newpage
%\lastpagecontrol[2cm]{13.7cm}
\vspace{0mm}
\begin{thebibliography}{99}
%\vspace{5mm}
\begin{spacing}{1.175}
\bibitem{JAEA}
	日本原子力研究開発機構 福島研究開発部門 福島研究開発拠点 福島環境安全センター:  
	福島における放射性セシウムの環境動態研究の現状(平成30年度版),JAEA-Research,
	2019-002,DOI:10.11484/jaea-research-2019-002,2019.
\bibitem{NUMO}
	原子力発電環境整備機構:地層処分事業の安全確保(2010年度版)-確かな技術による安全な地層処分の実現のために-, NUMO-TR-11-01, 2011.
\bibitem{Bear1}
	Bear,J.:Dynamics of fluids in porous media,Dover,1972.
\bibitem{Bear2}
	Bear,J.and Bachmat,Y.:Introduction to modeling of transport phenomena in porous media, Kluwer Academic Publishers,1990.
\bibitem{Nakano}
	中野政詩:土の物質移動学,東京大学出版会, 1991.
\bibitem{Fujinawa}
	藤縄克之:環境地下水学,共立出版,2010.
\bibitem{Rubin}
	Rubin,S., Dror,I. and Berkowitz, B.: Experimental and 
		modeling analysis of coupled non-Fickian transport and 
		sorption in natural soils,Journal of Contaminant Hydrology, vol.132,pp.28-36,2012.
\newpage
\bibitem{Masuda}
	Masuda,A.,Ushida,K. and Okamoto, T. :Direct observation of spatiotemporal dependence of anomalous diffusion in 
		inhomogeneous 
		fluid by sampling-volume-controlled fluorescence correlation spectroscopy, Physical Review E, vol.72,pp.060101-1-4,2005.
\bibitem{Non_Fick_review}
	Berkowitz,B., Cortis,A. and Scher,A.:Modeling non-Fickian transport in geological formations as a continuous time random walk, Reviews of Geophysics,vol.44,pp.1-49,2006.
\bibitem{Upscaling_review}
	Frippiat,C.C. and Holeyman,A.E.:A comparative review of upscaling methods for solute transport in heterogeneous porous media, Journal of Hydrology, vol.262,pp.150-176,2008.
%\bibitem{Cuadros}
%	Cuadros, J. : Clay as sealing material in nuclear waste repositories, %Geology Today, Vol.24, No.3, pp.99-103, 2008.
\bibitem{Liang}
	Liang,Z.,Ioannidis,M.and Chatzis,I.:Permeability and electrical conductivity of porous media from 3D stochastic replicas of the microstructure, Chemical Engineering Science,vol.55,pp.5247-5262,2000. 
\bibitem{chalk_model}
	Talkudar,M.S., Torsaeter, O. and Howard, J.J.:
	Stochastic reconstruction, 3D characterization and network modeling of chalk, Journal of Petroleum Science and Engineering, 
	vol.35, pp.1-21,2002.
\bibitem{Digital_porous}
	Kainourgiakis,M.E.,Kikkinides,E.S.,Galani,Charalambop-oulou,G.C. and Stubos,A.K.:
	Digitally reconstructed porous media: Transport and sorption properties,
	Transport in Porous Media, Vol.58, pp.43-62, 2005.
\bibitem{Kim}
	Kim,D.H., Young,J.K., Lee, J.-S. and Yun, T.S..:Thermal and electrical response of unsaturated hydropholic and hydrophobic granular materials,
	 Geotechnical Testing Journal,Vol.34,No.5,pp.1-9,2011.
\bibitem{Narsilio}
	Narsilio,G.A., Kress,J. and Yun, T. S.:
	Characterisation of conduction phenomena in soils at particle-scale: 
	Finite element analyses in conjunction with syhtetic 3D imaging, Computers and Geotechnics, vol.38, pp.828-836, 2010.
\bibitem{Berkowitz}   
        Berkowitz,B. and Hansen,P.D.:A Numerical Study of the Distribution of 	Water in Partially Saturated Porous Rock, 
	Transport in Porous Media, Vol.45, pp.303-319, 2001.
\bibitem{MC}  
        Lu, N.,Zeidman, B. D., Lusk, M. T., Willson, C. S. and Wu, D. T.: 
	A Monte Carlo paradigm for capillarity in porous media, Geophysical Research Letters, Vol.37, L23402, 2010.
\bibitem{comp_phys}  
	Metropolis,N.,Rosenbluth,A.W.,Rosenbluth,M. N.,Teller,A. H. and	Teller,E.:
		Equation of state calculations by fast computing machines, 
	J.Chem.Phys., 21(6), pp.1087-1092, 1953.
\bibitem{SurfChem}
	Butt, H.-J., Graf, K. and Kappl, M., 
	"Physics and chemistry of interfaces", WILEY-VCH, 2006.
\bibitem{Kimoto}
	中島 唯一,木本 和志,河村 雄行:間隙水分布を考慮した不飽和多孔質体の熱伝導解析, 
	土木学会論文集A2, Vol.74, No.2, pp.I105-I114,2018.
\bibitem{Toda}
	戸田 盛和,物理学30講シリーズ 分子運動30講,朝倉書店,1996.
\bibitem{Rwk_textbook}
	Klaffter, J., and Sololov,I.M.,秋元 琢磨(訳): 
	ランダムウォークはじめの一歩,共立出版, 2018.
\bibitem{Gennes}
ドゥジェンヌ,ブロシェール ヴィアール,ケレ:表面張力の物理学, 吉岡書店, 2003.
%\bibitem{Terada}
%	寺田 賢二郎, 菊池 昇: 均質化法入門, 丸善株式会社, 2003.
%\bibitem{Zohdi}
%        Zohdi, T. I. and Wriggers, P. : An Intruduction to Computational Micromechanics,Springer, 2008.
%\bibitem{Iwasaki}
%	岩崎 佳介, 木本 和志, 市川 康明: 含水した砂質土の熱応答特性に関する実験および数値解析,  土木学会論文集A2, Vol.70, No.2,  ppI115-I124, 2014.
%\bibitem{Yun}
%	Yun, T. S., and Evans, T. M. : Three-dimensional random walk network model %for thermal conductivity in pariculate materials, 
%	Computers and Geotechnics, vol.37, pp.991-998,2010.
\end{spacing}
\end{thebibliography}
\vspace{-5mm}
\begin{flushright}
	\small
	\bf{ (Received July 19, 2019)\\
	(Accepted December 24, 2019)}
\end{flushright}
%\lastpagesettings
\newpage
%\lastpagecontrol[1cm]{13.7cm}
\lastpagecontrol[1cm]{25.7cm}
\end{document}

%\begin{minipage}[c]{13.7cm}
%\end{minipage}
%\lastpagecontrol[0cm]{13.7cm}
%\newpage
%\begin{multicols}{1}
%-------------------------------------------------
%-------------------------------------------------
%\end{multicols}

