%\section{2次元拡散問題の設定}
Fig.1に実験に用いた花崗岩供試体を示す.試験片は岡山県万成地域の採石場で採取した万成花崗岩をブロック状に切断加工したものである.表面波の励起は,供試体表面に接触させた圧電超音波探触子で行う.探触子は,曲率半径26.1mm,投影面積25mm×40mmの圧電素子をくさび状のシュー材に取り付けたもので,シュー先端部の幅と長さは2×50mm,共振周波数は1MHzである.これは,試験片中に円筒波を発生させ,幾何減衰の影響を小さくすることを意図したものである.試験片は位置を精確に調整するために,水平2軸,回転1軸の3軸ステージ上に固定する.送信探触子は400Vの矩形電圧パルスを印加して駆動する.受信にはレーザードップラー振動計(LDV)を用い,受信波形をオシロスコープ上で4,096回平均化した後,デジタル波形として収録する.サンプリング周波数は15MHz,計測
%Fig. 2 The layout of the source and measurement apertures
%Fig.3 Snapshots of the measured ultrasonic wave field.
時間範囲は200μ秒とした.Fig.2に送信および受信領域の配置を示す.図中のSで示した線は,送信センサーの接触位置を表し,この部分で試験片に鉛直動が加えられる. RはLDVでスキャンする波形観測領域を示す.スキャンピッチはx,y方向とも0.5mmとし,R上で計41×61=2,501の波形を取得した.Fig.3に計測結果の一部を示す.この図は,時刻t=20~23μ秒で観測された波動場を1μ秒間隔のスナップショットとして示したもので,左から右へ表面波が伝播する様子が示されている.初動位置を正確に特定することは難しいが,大きな振幅を持つ波動が通過した背後の領域にも,多重散乱により振動が継続する様子が見られる.

3.位相構造と波数ベクトル分布の評価
Fig.4に周波数0.4,0.6,0.9および1.2MHzの波形成分に対する位相の空間分布を示す.相対的に低周波の0.4と0.6MHzでは,等位相線(波面)がy軸方向に伸びる1次元的な構造を示している.ただし,波面は屈曲して直線的ではない.一方,0.9および1.2MHzでは,y方向への位相の揺らぎが大きく,平面波的な構造が見られない.Fig.5に,位相分布の勾配を中央差分で近似して求めた,波数ベクトルkの確率密度分布を示す.Fig.5-(a)は,波数ベクトルの大きさ|k |に関する,(b)はx軸方向から測ったkの方向に関する確率密度を示している.(a)の図にあるように,|k |の確率密度は非対称かつ有限な幅をもち,ガウス分布的でもない.また,周波数が大きくなるにつれ分散が増加している.これは,波数と周波数の関係が確定的に定められないこと,高周波になる程波動場の分散性が強まるが,波数の大きさは一定範囲に留まることを示している.また,波数ベクトルは入射方向に配向するが,相対的に高周波の1.0MHzと1.2MHzでは配向性が低下し,伝搬経路の屈曲が強まることを示している.これら波数ベクトルの確率密度と周波数の関係を記述する法則を見出すことは今後の課題だが,花崗岩のランダム媒体としてのモデル化においては,このような波数ベクトルの特徴を反映する必要がある.

