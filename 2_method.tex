%\section{2次元拡散問題の設定}
固相と液相,および気相からなる2次元不飽和多孔質媒体における溶質の拡散問題を考える.
多孔質媒体は十分に広く無限領域とみなすことができ,間隙水は静止していると仮定する.
また,拡散物質(溶質)の液相中での自己拡散係数$D_0$は場所に依らず一定かつ既知として,
分子拡散によって液相内をランダムに移動する拡散物質を,粗視化した多数の
粒子として表現する.それら粒子群の変位とその時間発展挙動を調べることで,不飽和多孔質媒体
全体としての平均的な拡散係数(マクロ拡散係数)$K$を求める.
なお,無限領域を数値解析上で模擬するために,多孔質媒体は周期性をもつと仮定し,そのユニット
セル$\Omega$を対象として一連の解析を行う.\\
\hspace{\parindent}
図-\ref{fig:fig0}に示すように,ユニットセル$\omega$は一辺の長さが$W$の正方形領域とし,
多孔質体の間隙率を$n$,飽和度を$S_r$で表す.
ここで,液相領域を$\Omega_f$,気相領域を$\Omega_a$とすれば,$n$と$S_r$は
\begin{equation}
	n=\frac{\left|\Omega_f\cup\Omega_a\right|}{\left| \Omega \right|}, \ \ 
	S_r=\frac{\left|\Omega_f\right|}{\left|\Omega_f\cup\Omega_a\right|}
	\label{eqn:def_n_Sr}
\end{equation}
で与えられる.ユニットセル内での位置を表す$xy$直交座標は,ユニットセルの各辺と平行に,
図\ref{fig:fig0}に示すようにとる.ただし,数値解析条件を設定する際には,これらをセルサイズ
$W$で無次元化した座標
\begin{equation}
	X=\frac{x}{W}, \ \
	Y=\frac{y}{W}
	\label{eqn:XYcod}
\end{equation}
を用いる.
%なお,マクロ拡散係数$K$の定義は次節で述べる.
なお,マクロ拡散係数$K$の精確な定義は,
次節,第4項において,ランダムウォーク結果を用いたマクロ拡散係数の数値的な
評価方法とともに述べる.
\begin{figure}[t]
\begin{center}
	%\includegraphics[clip,scale=0.45]{Figs/fig0.eps}
\caption{
	周期不飽和多孔質体のユニットセル$\Omega$.
}
\label{fig:fig0}
\end{center}
\end{figure}
\begin{figure}[t]
\begin{center}
%\includegraphics[clip,scale=0.45]{Figs/fig1.eps}
\caption{
数値シミュレーションのプロセス.
	(a),(a')固相粒子のランダムパッキング,
	(b),(b')液相配置の決定,
	(c),(c')正方格子上のランダムウォーク.
}
\label{fig:fig1}
\end{center}
\end{figure}
\section{数値解析法}
ランダムウォークによる拡散解析を行うにあたり,不飽和多孔質媒体の間隙構造を表現した形状モデルが必要となる.
形状モデルの作成には,著者ら\cite{Kimoto}が提案した,モンテカルロ・シミュレーションによる方法を用いる.
この方法では,固相粒子をユニットセル内にランダム充填して多孔質構造を作成し,
その後,間隙領域に所定の飽和度となるように液相を配置する.
固相粒子のパッキングと液相領域の配置決定は,それぞれの目的にあわせて定義したエネルギー
$E$の最小化問題として定式化し,モンテカルロ・シミュレーションで最適解を探索する.
いずれも,比較的少数の入力データを与えるだけで計算を実行することができ,
多数の不飽和多孔質体モデルを簡単かつ自動的に生成できることがこのアプローチの利点である.
以下ではランダムパッキングと間隙水配置決定のためのモンテカルロ・シミュレーションについて概説し,
続いて,ランダムウォークによる拡散解析法とマクロ拡散係数の決定方法を説明する.
%%%%
\subsection{固相粒子のランダムパッキング}
任意の形状をもつ剛体固相粒子をユニットセル内に充填する問題を考える.
個々の粒子の形状と粒子群の粒径分布は既知とし,ユニットセル内に
含まれる固相粒子数$N_p$は所定の間隙率となるように与えておく.
固相粒子の初期配置は,図\ref{fig:fig1}-(a)に示すようにランダムに設定する.
その結果,多数の粒子が初期状態では互いに重なって配置されるため,
粒子位置と向きを繰返し変更し,粒子間に重なりが生じない配置(図-\ref{fig:fig1}の(a'))をメトロポリス法に基づくモンテカルロ・シミュレーション(MC)\cite{comp_phys}と
焼きなまし法(SA:simulated annealing)で探索する.
ここで,ユニットセルに充填する$N_p$個の固相粒子のうち,第$i$番目の粒子が占める
領域を$A_i$,その面積を$\left| A_i \right|$と表す.
このとき,異なる2つの粒子が重なった部分の面積を合計した値は,
\begin{equation}
	E=\sum_{i,j=1, i\neq j}^{N_p} \left| A_i\cap A_j \right|
	\label{eqn:overlap}
\end{equation}
で与えられる. そこで,式(\ref{eqn:overlap})で与えられる$E$を,
モンテカルロ・シミュレーションにおけるエネルギーとして用い,
$E$が0となるような固相粒子の配置を探索する.
モンテカルロ・シミュレーションでは,任意に選択された固相粒子の
位置と向きを,仮想的に変更した場合に生じるエネルギーの変化$\Delta E$を計算する.
その結果,$\Delta E<0$であれば無条件に,$\Delta E \geq 0$
であれば,
\begin{equation}
	P=\exp\left( -\frac{\Delta E}{k_BT}\right)
	\label{eqn:Boltzmann}
\end{equation}
によって与えられる確率$P$で粒子配置実際に変更して系の状態を更新する.ここに,$k_B$はボルツマン定数を,
$T$は最適化のための仮想温度を表す.温度$T$は,全ての粒子について位置と向きの
変更を検討して状態更新あるいは維持を行った後,所定の量$\Delta T(<0)$だけ低下させる.
温度$T+\Delta T$において同じ手順で系の状態更新を行い,この作業を$E$が十分に小さくなるまで
繰り返す.なお,間隙率$n$があまり大きく無い場合,$E$が完全にゼロとなる粒子配置を
許容し得る計算時間で見出すことは難しい.そこで本研究では,エネルギー$E$と,達成すべき
固相領域の面積$(1-n)W^2$比が0.1$\%$以下となった時点で計算を終了することとした.
また,固相粒子間で共有される領域面積$\left| A_i \cap A_j \right|$の評価は,
固相粒子をピクセルデータで表現し,2つの粒子$i$と$j$に共有されるピクセル数を
カウントすることによって行った.
%%%%%%%%%
\subsection{間隙水配置の決定\cite{MC,Kimoto}}
間隙水(液相)の配置は,気相−液相,固相−液相,固相−気相界面の,全界面エネルギー$E$が
停留値を取るように決定する.いま,$\alpha$相と$\beta$相が接するときの界面自由エネルギー
を$\gamma_{\alpha \beta}$とすれば,全界面エネルギーは,
\begin{equation}
	E=\sum_{\alpha,\beta, \alpha\neq\beta}
	\int_{\partial A_{\alpha\beta}}\gamma_{\alpha \beta}dS
\end{equation}
で与えられる.ただし,$\partial A_{\alpha\beta}$はユニットセルに含まれる
$\alpha$相と$\beta$相の界面を表す.液相の配置を数値解析モデルにおいて表現するためには,
図-\ref{fig:fig1}の(b)に示すように,間隙領域を一辺が$h$の小さな正方形セルに分割し,
各セルが液相あるいは気相いずれの状態にあるかを定める.液相の初期配置は,
この図にあるようランダムに設定する.ただし,液相状態にあるセルの総数は,
指定された飽和度となるようにとる.
ここで,セル総数を$M$,第$i$セルの状態と界面エネルギーをそれぞれ$\alpha_i, E_i$で表し,
第$i$セルを取り囲む8つの近傍セルの番号を集めた集合を$I(i)$とすれば,$E_i$は
\begin{equation}
	E_i=\sum_{j\in I(i)}\gamma_{\alpha_i \alpha_j}\Delta s
	\label{eqn:E_surf}
\end{equation}
で計算できる.また,全界面エネルギ$E$は,その和として
\begin{equation}
	E=\sum_{i=1}^{M}E_i
	\label{eqn:Etot}
\end{equation}
で与えられる.なお,式(\ref{eqn:E_surf})の$\Delta s$は,隣接するセル界面の面積を意味し,
同相のセル境界では$\gamma_{\alpha\alpha}=0$と解釈する.モンテカルロ・シミュレーションでは,
ランダムに選択した液相および気相セルのペアで,互いの状態を仮想的に交換したときの
界面エネルギーの変化$\Delta E$を求める.$\Delta E$が負となる場合は,
液相と気相の状態をピクセル間で実際に交換し,$\Delta E \geq 0$の場合は,
ボルツマン分布(\ref{eqn:Boltzmann})で与えられる確率$P$で状態交換を行うか否かを判定する.
このような状態更新を全ての液相セルについて温度$T$一定のもとで行った後,
所定量だけ温度を下げ,同じ手順で状態更新を繰り返す.この作業を界面自由エネルギー$E$に
変化が見られなくなるまで行うことで最終的な液相の配置を決定する.
以上の計算過程で気相と液相セルそれぞれの数は変化しないため,
初期状態で設定した所望の飽和度もつ不飽和多孔質体モデルを作成することができる.
なお,Luら\cite{MC}は,このような方法で間隙水分布を決定した結果,固相粒子の接触部に
形成されるメニスカスの曲率半径がヤング-ラプラスの式\cite{SurfChem}で与えられる
曲率半径とよく一致することや,CTスキャンで取得した間隙水分布とシミュレーション結果が
良好に一致することを示している.中島ら\cite{Kimoto}は,同様な手法を用い,固相表面が
親水的な場合と撥水的な場合のシミュレーションを行い,設定した通りの濡れ角で気泡や液滴
が形成されることを示している.このように,本研究で用いる水分配置決定方法は,
既往の研究によってその妥当性が示されているものである.
\subsection{ランダムウォーク}
液相中の拡散物質が分子拡散によってランダムに移動する様をランダムウォークでシミュレートする.
このとき,ランダムウォーカーの物理的意味は,拡散物質を粗視化して表現した粒子であることから,
以下ではランダムウォーカーを拡散粒子と呼ぶことにする.なお,拡散粒子の数密度は,
拡散物質の濃度$C$に比例すると解釈でき,$C$が通常の拡散方程式に従う場合は
通常拡散,そうで無い場合は異常拡散と呼ばれる.\\
\hspace{\parindent}
本研究で行うランダムウォークシミュレーションでは,液相セルの中心点をノードとする正方格子上で
拡散粒子を移動させる.拡散粒子の移動は,時間ステップ$\Delta t$毎に,隣接する上下,
左右いずれかのノードへ指定された確率で行う.
図-\ref{fig:fig1}の(c)に示したように,4つの隣接ノードへの移動確率を$p_1\sim p_4$で,
黒の点で示した現在のノードにとどまる確率を$p_0$と表せば,
隣接セル全てが液相の場合は,分子拡散が等方的であることを踏まえ
\begin{equation}
	p_1=p_2=p_3=p_4=\frac{1}{4}, \ \ p_0=0
	\label{eqn:iso_p}
\end{equation}
とする.一方,いずれかの隣接セルが固相あるいは気相の場合は,それらのセル中心ノードへ
の移動確率を0とし,現在位置に滞留する確率$p_0$を$\sum_{i=0}^4p_i=1$で与える.
例えば,右と上の隣接セルが液相でない場合,
\begin{equation}
	p_2=p_3=0, \ \ p_1=p_4=\frac{1}{4}
\end{equation}
とし,現在位置にとどまる確率$p_0$を,
\begin{equation}
	p_0=1-\sum_{i=1}^4 p_i=\frac{1}{2}
\end{equation}
とする.図-\ref{fig:fig1}の(c')は,ノード間の移動確率を上記のように定めて行った
ランダムウォークの結果(一部)を示したものである.赤の実線が拡散粒子の移動経路を,黒の点が
その過程で経由したノードを表している.\\
ユニットセル全体が液相で占められる場合,液相の自己拡散係数$D_0$は,ランダムウォークの
時間ステップ$\Delta t$とノード間隔$h$,隣接ノード間の移動確率$p_i=p=\frac{1}{4},\,
(i=1,\dots 4)$により
\begin{equation}
	D_0=\frac{ph^2}{\Delta t}=\frac{h^2}{4\Delta t}
	\label{eqn:D0}
\end{equation}
で与えられることが,ランダムウォーク解析に関する初等的な結果として知られている\cite{Toda}.
従って,$D_0$と$h$を与えれば,時間ステップ$\Delta t$は,次の式で決めることができる.
\begin{equation}
	\Delta t=\frac{h^2}{4D_0}
	\label{eqn:dt}
\end{equation}
後に示す数値シミュレーションでは,多数のランダムウォーカーをユニットセル内に配置し,
予め指定した時間ステップ$N_t$までランダムウォークを行う.
その際,各時間ステップにおける拡散粒子の平均2乗変位を記録し,次項に述べる方法
でマクロ拡散係数を求める.
\subsection{マクロ拡散係数の評価}
ランダムウォークに用いる拡散粒子の総数を$N_{wk}$,
そのうち第$i$番目の粒子に関する,時刻$t$における初期位置からの変位を
\begin{equation}
	\mathbf{u}_i(t)=\left(u_i(t),v_i(t)\right)
	\label{eqn:u_i}
\end{equation}
と表す.拡散粒子の平均2乗変位を
\begin{equation}
	\left< u_i^2 \right>=
	\frac{1}{N_{wk}} \sum_{i=1}^{N_{wk}}u_i^2
	, \ \ 
	\left< v_i^2 \right>=
	\frac{1}{N_{wk}} \sum_{i=1}^{N_{wk}}v_i^2
	\label{eqn:u2b}
\end{equation}
と書くとき,拡散係数と平均2乗変位の間には
\begin{equation}
	\left< u_i^2 \right>=2 D_{x} t, \ \ 
	\left< v_i^2 \right>=2 D_{y} t
	\label{eqn:Einstein}
\end{equation}
の関係がある.ここで$D_x$と$D_y$は,それぞれ$x$方向,$y$方向への拡散係数を表す.
よって,ランダムウオークシミュレーションの結果から$\left<u_i^2\right>$
や$\left<v_i^2\right>$の時刻歴を求めて式(\ref{eqn:Einstein})
でフィッティングすれば,拡散係数$D_x,D_y$を得ることができる.
ユニットセル全体が液相で占められる場合,その結果は$D_x=D_y=D_0$となるが,
不飽和多孔質体の場合,固相配置と間隙水分布に応じて拡散経路が制限されるため
$D_x,D_y<D_0$となる.$D_x$や$D_y$は,多孔質体全体としての拡散係数を表すため,
間隙スケールをミクロ,ユニットセルスケールをマクロスケールと考え,
$D_x,D_y$をマクロ拡散係数と呼ぶ.また,拡散が等方的である場合,両者をまとめて
\begin{equation}
	D=D_x=D_y
	\label{eqn:D_iso}
\end{equation}
と書くことにする.この場合,$u_i$と$v_i$の平均2乗変位も一致することから,
両者を区別することなく
\begin{equation}
	\left<u^2\right>
	=
	\left<u_i^2\right>
	=
	\left<v_i^2\right>
	\label{eqn:}
\end{equation}
と書く. 平均2乗変位の時間変化が式(\ref{eqn:Einstein})に従う場合,拡散物質の濃度$C$は拡散方程式
\begin{equation}
	\frac{\partial C}{\partial t}=D\nabla ^2 C
	\label{eqn:diff_eq}
\end{equation}
に従う.一方,マクロな拡散が式(\ref{eqn:diff_eq})に従わない事例が少なくないことは
既往の研究\cite{Rubin}$^-$\cite{Upscaling_review}に報告されており,その場合,平均2乗変位の時刻歴はべき関数
\begin{equation}
	\left< u_i^2 \right>=
	\left< v_i^2 \right>=2 K t^\alpha
	\label{eqn:abnormal}
\end{equation}
によってより精度良く近似される.式(\ref{eqn:abnormal})の係数$K$は,
一般化された拡散係数を表し,べき指数$\alpha$が1のときには通常の拡散
係数に一致する\cite{Rwk_textbook}.一方,$\alpha \neq 1$の場合は,異常拡散と呼ばれ
$\alpha < 1$は"遅い拡散"を,$\alpha>1$は"速い拡散"と呼ばれる.
つまり,$K$は拡散のスケールを,$\alpha$は拡散のタイプを表す指標となっている.
ただし,本研究のモデルで拡散が促進される要因は無く,$\alpha$は常に1以下である.
以下では,平均2乗変位の時刻歴から$K$と$\alpha$を求め,間隙構造と水分量が
これらの係数に与える影響を調べる.
%%
