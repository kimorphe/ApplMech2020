%\section{超音波計測実験}
\subsection{実験供試体}
図-\ref{fig:fig1}に実験に用いた花崗岩供試体を示す.
この試験片は岡山県万成地域の採石場で採取した万成花崗岩をブロック状に切断加工したものである.
試験片表面には,目視で認められるような欠けや割れ,風化はない.
主要鉱物は,カリ長石,ナトリウム長石,石英および雲母の四種類で,
特徴的な桃色の色合いをした箇所がカリ長石である。
試験片のサイズはは長さ$L=178$mm, 幅$W=56$mm,厚さ$H=30$mmで,
図-\ref{fig:fig1}のような$xyz$座標系において$x$方向へ伝播する
超音波を計測する.全ての計測は,試験片の上面$(z=0)$mmにおいて行う.
また,均質材における波動伝播挙動との比較を行うため,同様な
超音波計測は,アルミニウムブロックの供試体でも行う.
アルミニウム供試体のサイズは,長さ200mm,幅150mm,厚さ50mmの
直方体で,音響異方性がほとんど無いことを予め確認している。
\begin{figure}
\begin{center}
\includegraphics[clip,scale=0.5]{Figs/samples.eps}
\caption{
	超音波計測に用いた花崗岩供試体.
}
\label{fig:fig1}
\end{center}
\end{figure}
\subsection{超音波探触子(送信子)}
表面波の励起は,供試体表面に接触させた圧電超音波探触子で行う.
送信に用いた超音波探触子の外観は図-\ref{fig:fig3}のようである。
圧電素子を収納した筐体部分と,圧電素子がマウントされた
ウェッジ(シュー)部分が示されている。
圧電素子は曲率半径26.1mm,投影面積が25mm×40mmの瓦状で、
共振周波数は2MHzのものを用いている。
圧電素子は同じ曲率半径を持つウェッジ上縁部に接着されており、
ウェッジ内部を伝播した縦波が先端部に集束するように設計されている。
ウェッジ先端部を試験片に接触させて用いれば、
試験片内部を円筒状に広がる弾性波が,ウェッジ先端部から励起される。
ウェッジの先端部は幅と長さが1×50mmである。
このような線集束型の探触子を用いることで,入射位置と伝播方向を
定義して制御することができる
また,試験片表面から強い半円筒波状の超音波を励起することで,
点波源から半球状の球面波を励起した場合に比べ,幾何減衰の影響を
小さくすることができる.
\begin{figure}[h]
\begin{center}
\includegraphics[clip,scale=1.0]{Figs/fig3.eps}
\caption{
	超音波の送信に用いた線集束探触子の外観.(a)正面,(b)側面から見た様子.
}
\label{fig:fig2}
\end{center}
\end{figure}
\subsection{超音波計測装置の構成}
図-\ref{fig:fig3}に,超音波計測装置の構成を示す.
計測装置は,3軸ステージ,レーザードップラー振動計(LDV),オシロスコープ,および
高周波スクウェア−ウェーブパルサーで構成されている.
試験片は,位置を精確に調整するために,水平2軸,回転1軸の3軸ステージ上に固定する.
送信に用いる探触子は,試験片表面に接触させて固定し,
400Vの矩形電圧パルスを,スクウェア−ウェーブパルサで印加して駆動する.
受信にはLDVを用い,受信波形はオシロスコープへ転送して4,096回平均化した後,
デジタル波形としてPCに収録する.なお,サンプリング周波数は15MHz,
計測時間範囲は200$\mu$秒とした.送信波形は2MHz程度のパルス状の波形だが,
岩石供試体では低い周波数成分が主として透過する.さらに,多重散乱により
振動の継続時間が送信パルス幅より長くなることから,サンプリングレートを
若干低めにし,振動が十分に小さくなる時間まで計測時間を長めにとるように条件を
設定した。
\begin{figure}[t]
\begin{center}
\includegraphics[clip,scale=0.5]{Figs/ut_setup.eps}
\caption{
	超音波計測装置の構成.
}
\label{fig:fig3}
\end{center}
\end{figure}
\subsection{送受信位置}
図-\ref{fig:fig4}に,送信および受信領域の配置を示す.
図中の${\cal S}$は送信位置,すなわち,線集束探触子のウェッジ先端部が接触する位置を表しており,
この部分で試験片に鉛直動が加えられる.${\cal R}$はLDVでスキャンする波形観測領域を示す.
$\cal R$は20mm$\times$30mmの矩形領域にとり,計測ピッチは$x$,$y$方向とも0.5mmとすることで,
$\cal R$上の正方格子状に配置された観測点で計41×61=2,501の供試体表面における鉛直動の時刻歴波形を取得した.
ここで,観測点の$x$および$y$軸方向の間隔を,それぞれ$\Delta x,\Delta y$とすれば,
$x$方向に$i$番目,$y$方向に$j$番目の観測点の座標$(x_i,\, y_j)$は
\begin{equation}
	(x_i,\, y_j)=(x_0+i\Delta x,\, y_0+j\Delta y)
	\label{eqn:x_ij}
\end{equation}
と書くことができる.また,観測点全体がなす格子を${\cal G}$とすれば,
\begin{equation}
	{\cal G} = \left\{ 
	(x_i,\, y_j)\left| i=0,\dots N_x, \, j=0,\dots N_y  \right.
	\right\}
	\label{eqn:Grid}
\end{equation}
と表される.ただし,$N_x$と$N_y$は$x$および$y$軸方向の観測点数を表し,
格子点数や格子間隔は実際には
\begin{equation}
	\Delta x=\Delta y=0.5 {\rm mm}, \ \ N_X=41, \, N_y=61
	\label{eqn:grid_prms}
\end{equation}
\begin{equation}
	(x_0,y_0)=(0,-15){\rm mm}
	\label{eqn:grid_corner}
\end{equation}
である.ここで,$t$を時間変数として,位置$(x,y)$において観測した時刻歴波形を
$a(x,y,t)$と表す.以下では, 簡単のため$x,y$および$t$はいずれも
連続変数として扱い,$a(x,y,t)$に関する微分や積分は,観測データを使って評価する
際には,適宜離散化して評価する.
\begin{figure}[t]
\begin{center}
\includegraphics[clip,scale=0.5]{Figs/cod.eps}
\caption{
	超音波の送信位置$\cal S$と受信領域$\cal R$の配置.
}
\label{fig:fig4}
\end{center}
\end{figure}
