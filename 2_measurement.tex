%\section{超音波計測実験}
\subsection{実験供試体}
実験に用いた花崗岩供試体を図-\ref{fig:fig1}に示す.
この供試体は,岡山県万成地域の採石場で採取した万成花崗岩を岩石カッターでブロック状に
切断加工したものである.研磨等による切断面の仕上げは行っていないが,供試体表面に凹凸や目視で認められる欠けや割れ,明らかな風化はない.
万成花崗岩の主要造岩鉱物は,カリ長石,ナトリウム長石,石英および雲母の四種類で,
特徴的な桃色の色合いをした箇所がカリ長石である.
これら主要鉱物の割合は,カリ長石34\%,長石17\%,石英44\%,雲母5\%で,平均粒系は順に,
約1.6,0.9,1.1および0.5mmである.これらの鉱物には音響異方性があり,結晶軸からの方向に
応じた弾性波速度の変化は大きい.ただし,供試体のほぼ95\%を占める長石と石英の弾性波速度は,
縦波が5.59$\sim$6.06km/s,横波が3.06$\sim$4.11km/sとのデータが知られている\cite{RockPhys}.
試験片のサイズは長さ$L=178$mm, 幅$W=56$mm,厚さ$H=30$mmで,
計測位置は図-\ref{fig:fig1}のような$xyz$座標系で表す.
超音波の送信と受信は,試験片の上面$(z=0)$mmにおいて行い,$x$軸の正方向へ伝播する表面波を計測する.
また,均質材における波動伝播挙動との比較を行うため,同様な計測を,アルミニウムブロック供試体でも行う.
アルミニウム供試体のサイズは,長さ200mm,幅150mm,厚さ50mmの直方体で,後に述べる送受信位置のとり方は,花崗岩供試体の場合と同様である.
\begin{figure}
\begin{center}
\includegraphics[clip,scale=0.5]{Figs/samples.eps}
\caption{
	超音波計測に用いた花崗岩供試体.
}
\label{fig:fig1}
\end{center}
\end{figure}
\subsection{超音波探触子(送信子)}
超音波の送信は,供試体表面に接触させた圧電超音波探触子で行う.
実験に用いた超音波探触子の外観は図-\ref{fig:fig2}のようであり,この図には
圧電素子を収納した筐体部分と,圧電素子がマウントされたウェッジ(シュー)部分が示されている.
内部に収納された圧電素子は,曲率半径が26.1mm,投影面積が幅25mm×長さ40mmの瓦状のもので,共振周波数は2MHzである.
圧電素子は,曲率半径を合わせて作成されたウェッジ上縁部に接着されており,
圧電素子で励起した縦波がウェッジ内部を伝播して先端部に集束するよう設計されている.
従ってウェッジ先端部を供試体に接触させて用いることで,供試体内部を円筒状に広がる弾性波が,
線状の接触部から励起される.なお,供試体に接触させるウェッジの先端部の幅と長さは1×50mmである.
%
高い超音波の集束効率を得るためには,センサーの開口を大きくとる必要がある.
また,ウェッジ内での減衰を抑えるためには,ウェッジの高さは低い方がよい.
一方で、入射点近傍での受信や,実験装置への取り付け,ウェッジ内部における多重反射波の抑制の
面からはこの逆のことが言える.上に述べた,圧電素子の曲率と幅はこれらの兼ね合いを考慮して
決定したものである.なお,圧電素子の長さは,表面波の波長をおよそ3mm程度と想定し,
その10倍以上となることを目安とした.
%
このような線集束型の探触子を用いることで,入射点と伝播方向が明確に設定される.
また,供試体表面から強い半円筒波状の超音波を励起することで,点波源から半球状の球面波を励起した場合に比べ,
幾何減衰の影響も小さくすることができ,信号/雑音比の点で有利になる.
%
らに重要なことは,伝播方向と伝播距離が共通する複数の地点で波形観測できる点にある.
花崗岩は音響異方性をもつ高減衰な材料である.そのため,伝播速度のゆらぎに関する議論では,方向と距離を指定して統計を取る必要があり,円筒波を励起する今回の探触子は,このことに配慮して選択したものである.
\begin{figure}[h]
\begin{center}
\includegraphics[clip,scale=0.9]{Figs/fig3.eps}
\caption{
	超音波の送信に用いた線集束探触子の外観.(a)正面,(b)側面から見た様子.
}
\label{fig:fig2}
\end{center}
\end{figure}
\subsection{超音波計測装置の構成}
実験に用いた超音波計測装置の構成を図-\ref{fig:fig3}に示す.計測装置は,3軸ステージ,レーザードップラー振動計(LDV),オシロスコープ,および高周波スクウェア−ウェーブパルサーで構成されている.
供試体は水平2軸,回転1軸の3軸ステージ上に固定し ,LDVによるレーザー照射位置を精確に調整する.
その際,送信探触子は,試験片表面に接触させて固定し,供試体とともに移動させる.
探触子の駆動はスクウェア−ウェーブパルサーを用いて行い,400Vの矩形パルス電圧を印加する.
受信にはLDVを用い,受信波形をオシロスコープへ転送し,4,096回の平均化を行った後,デジタル波形としてPCに収録する.サンプリング周波数は15MHz,計測時間範囲は200$\mu$秒とし,全ての計測は同じ条件で行った.
送信探触子の公称周波数は2MHzであるが,サンプリング周波数はやや低めに設定されている.
しかしながら,花崗岩供試体では低い周波数成分が主として透過し,多重散乱により振動の継続時間も送信パルス幅より長くなる傾向にある.このことに配慮し,ここではサンプリングレートを若干低めにし,計測時間範囲を余裕をもって設定することとした.
\begin{figure}[t]
\begin{center}
\includegraphics[clip,scale=0.45]{Figs/ut_setup.eps}
\caption{
	超音波計測装置の構成.
}
\label{fig:fig3}
\end{center}
\end{figure}
\subsection{送受信位置}
図-\ref{fig:fig4}に,送信および受信領域の配置を示す.
ここで,${\cal S}$は送信位置,すなわち,線集束探触子のウェッジ先端が接触する位置を表し,この部分で試験片に鉛直動が加えられる.
${\cal R}$はLDVでスキャンする波形観測領域を表し,その大きさと形状は20mm$\times$30mmの矩形領域になっている.計測ピッチは$x$方向,$y$方向とも0.5mmとし,$\cal R$上の正方格子状に配置された観測点で計41×61=2,501の超音波時刻歴波形を取得する.
なお,送信位置と受信領域の距離は20mmとしている.
これは,送信探触子の筐体に遮蔽され,レーザー光を直接照射することのできない領域が存在するためである.
アルミニウム供試体における観測では,座標原点を供試体表面の中央に取る他は,花崗岩供試体の場合に同じとした.

ここで,観測点格子の$x$および$y$軸方向間隔を,それぞれ,$\Delta x,\Delta y$とすれば,
$x$方向に$i$番目,$y$方向に$j$番目の観測点座標$(x_i,\, y_j)$は
\begin{equation}
	(x_i,\, y_j)=(x_0+i\Delta x,\, y_0+j\Delta y)
	\label{eqn:x_ij}
\end{equation}
と書くことができる.また,観測点が成す格子全体を${\cal G}$とすれば,
\begin{equation}
	{\cal G} = \left\{ 
	(x_i,\, y_j)\left| i=0,\dots N_x, \, j=0,\dots N_y  \right.
	\right\}
	\label{eqn:Grid}
\end{equation}
と表される.ただし,$N_x$と$N_y$は$x$および$y$軸方向の観測点数を表す.
実際の格子(観測)点数や格子間隔は,既に述べた通りであり,それらをまとめて示すと以下の通りとなる.
\begin{equation}
	\Delta x=\Delta y=0.5 {\rm mm}
	\label{eqn:grid_prms}
\end{equation}
\begin{equation}
	N_x=41, \, N_y=61
	\label{eqn:grid_nums}
\end{equation}
\begin{equation}
	(x_0,y_0)=(0,-15){\rm mm}
	\label{eqn:grid_corner}
\end{equation}
以下では,$t$を時間変数とし,位置$(x,y)$において観測した時刻歴波形を$a(x,y,t)$と表す.
簡単のため,$x,y$および$t$はいずれも連続変数として表記するが,$a(x,y,t)$に関する微分や積分などの演算を観測データに施す場合,観測点位置での値を使い,適宜離散化して評価する.
\begin{figure}[t]
\begin{center}
\includegraphics[clip,scale=0.4]{Figs/cod.eps}
\caption{
	超音波の送信位置$\cal S$と受信領域$\cal R$の配置.
}
\label{fig:fig4}
\end{center}
\end{figure}
