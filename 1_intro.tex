%\section{はじめに}
建設分野において弾性波検査の対象となる材料の多くは,形状や物性が不規則かつ不均一な,ランダム媒体である.例えば,コンクリートは,粒径や形状が異なる骨材と気泡がランダムに分布した非均質媒体である.岩盤や岩石も,断層や節理系から,岩石を構成する鉱物粒や介在物,マイクロクラックまで,各種の空間スケールで多様な非均質性を有する.このような非均質ランダム媒体において,弾性波は不均質部との相互作用によって散乱や屈折を起こし,複雑な伝播挙動を示す.そのため,地震探査や岩石コアの弾性波検査,ボーリング時の音波検層結果の解釈は,金属材料のような均質材に対する非破壊検査には無い困難が伴う.特に,弾性波が強い多重散乱を起こしながら媒体を伝播する場合,著しい減衰や波形の変化のために,計測で得られた波形から有用な情報を取り出すことが,一般に難しい.このような困難を克服し,ランダム不均質媒体に対する信頼性や精度の高い弾性波検査技術を開発するには,多重散乱を考慮した波動伝播モデルを構築が必要となる.

物理探査や非破壊検査において反射源位置を特定する際,弾性波速度が既知であることが必要であるが.この理由から,種々の弾性波伝播特性の中でも伝播速度は重要であり,このことはランダム不均質媒体でも同様である.ただし,ランダム媒体の場合,媒体の物性値が場所によって異なるため,計測点毎に弾性波の到達時間と,そこから見積もられる弾性波速度には必然的にばらつきが生じる.弾性波速度のばらつきは反射源位置の同定精度と不確実性に影響するため,ランダム媒体に関しては弾性波速度の平均値だけでなくばらつきも重要な情報となる.また,弾性波速度のばらつきは媒体の不均質性を反映したものであることから,ランダムな不均質性を弾性波計測データから調べる目的においては,速度のばらつきを含め,弾性波速度が従う統計分布自体が興味の対象となる.

ランダム不均質媒体における弾性波速度のばらつきを調べるために,これまで,理論,数値解析および実験的な研究が行われてきた.理論および数値解析的な研究には,波線理論や多重散乱理論を用いて伝播時間解析を行ったものや,差分法モデル等でランダム媒体中を伝播する波動の解析を行ったものなどがある.一方,実験的な研究には,岩石試料を用い,超音波計測によって弾性波速度のばらつきを評価したものが挙げられる.西澤らは,花崗岩試料を透過する縦波と横波の波形を,レーザードップラー振動計を使って計測し,試料表面の振動状況を可視化するとともに,透過波の到達時間とばらつきを調べている.また,超音波計測は,鉱物粒径の異なる3種類の花崗岩に対して行い,不均質性のスケールが伝播時間やそのばらつきに与える影響についても議論をしている.このような研究は,弾性波計測結果から不均質性のスケールを推定するために必要な情報を与えるという意味でも重要なものである.しかしながら,伝播時間や伝播速度のばらつきは,弾性波の伝播距離や方向にも依存する可能性が高く,この点についてはこれまで十分明らかにされていない.

以上のことを踏まえ本研究では,弾性波伝播時間のばらつきが伝播距離に応じてどのように変化するかを明らかにすることを目的に,超音波計測と詳細な波形解析を行う.実験では,典型的なランダム不均質媒体である花崗岩を供試体として用い,圧電トランスデューサで励起した表面波の振動を,レーザードップラー振動計(LDV)で多点計測する.LDVを用いて表面波を対象とした計測を行う理由は,試料表面の振動を高い時空間解像度で広い周波数帯域において観測することで,波動場の進展状況を精確に捉えることを意図したものである.また,超音波の励起には,接触型の線集束トランスデューサを用い,試料内部に円筒波を励起する.これにより,強い超音波送信することに加え,入射方向と伝播距離を明確に定義することができるようになる.計測で得られた一連の波形は周波数領域において解析し,フェルマーの原理に基づいて各観測点における超音波の到達時間を周波数ごとに求める.このようにして得られた到達時間のアンサンブルから,到達時間の確率分布を伝播距離の関数として求める.最後に,到達時間の確率分布から平均と標準偏差を求め,到達時間のばらつきが距離に応じてどのような法則に従い変化していくかを明らかにする.

以下では,超音波計測の方法を,実験供試体,計測点の配置,計測装置の構成の順に述べる.次に,計測で得られた波形データから,表面振動の様子を可視化して,どのような波動場が形成されているかを示す.続いて,各観測点,周波数における到達時間を求める方法を述べ,計測結果から求めた到達時間の空間分布の一例を示す.最後に,到達時間の確率分布とその平均,標準偏差をを伝播距離の関数として求め,到達時間の不確実性が空間的にどのように発展するかを示して結論を導く.
