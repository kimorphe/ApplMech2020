%\section{はじめに}
地盤中の物質輸送メカニズムを精確に理解し,その輸送特性を評価するための信頼できる
モデルと数値解析手法を開発することは,
%例えば,土壌汚染の予防や対策に関して定量的な検討を行う際に重要である.
例えば,放射性物質の土壌や地層中での長期的な移動挙動を理解する上で重要である\cite{JAEA,NUMO}.
地盤は,固相(土粒子),液相(間隙水),気相(間隙空気)
から構成される多孔質媒体で,地表部近傍では不飽和状態にあることが多い.
そのため,地盤中の物質輸送に関する問題は,一般には不飽和多孔質体における
物質輸送問題として扱う必要がある.
多孔質体の間隙水に溶存した物質の移動は,間隙水自体の流れによる移流と
溶質の分子拡散によって生じ,複雑に分岐や合流する間隙のネットワークを
通じて物質が輸送される.間隙水の流れを間隙スケールで見るとき,流速の分布や
方向は空間的に複雑に変化する.そのため,間隙の代表寸法よりも十分大きなスケール
(マクロスケール)で物質輸送をみた場合,溶質は間隙水の平均流速による移流に加え,
局所的な流速変動に起因した分散を起こす.間隙スケール(微視スケール)での
局所的流速変動に起因するこのような見かけの分散は,機械的分散と呼ばれる\cite{Bear1,Bear2}.\\
\hspace{\parindent}
多孔質体における物質輸送をマクロスケール移流-拡散問題として記述する場合,
拡散に関する項には,分子拡散からの寄与と機械的分散の効果を含める必要がある.
実験を行い両者の効果が含まれた拡散係数(流体力学的分散係数\cite{Nakano},
あるいは水力学的分散係数\cite{Fujinawa})
を同定すれば,移流-拡散方程式を解くことで物質の時空間的な変動を調べることができる.
移流-拡散方程式を用いて多孔質体中の物質輸送挙動を調べる方法は実用的ではあるものの,
いくつか潜在的な問題点があると考えられる.
第一に,実験で求める拡散係数の適用範囲に関する問題が挙げられる.
多孔質体のマクロスケールでの拡散係数(マクロ拡散係数)は,多孔質体を構成する材料
や表面の性状,間隙構造や温度など非常に多くの条件に影響を受ける.
これらの条件が拡散係数の計測を行った実験と異なる場合,既知の拡散係数値を内挿
あるいは外挿して用いることが適切かどうかの判断は容易ではない.
第二に,実験による拡散係数の評価では,分子拡散と機械的分散の影響を分離して観測
することや,吸着と遅い拡散の区別は簡単ではない.さらに,粘土中の物質輸送のような
長期に渡る拡散問題や,岩盤や断層系のような空間的に大きなスケールでの不均質性が
問題となる場合は実験自体が難しい.
第三に,対象とする拡散現象をマクロスケールでみたとき,通常の拡散方程式に従う保証はなく,
実際,地盤材料や高分子溶液のように微視的に非均質かつランダムな材料では,
通常の拡散方程式に従わない事例は少なくない
%\cite{Rubin,Masuda,Non_Fick_review,Upscaling_review}.
\cite{Rubin}$^-$\cite{Upscaling_review}.
これらの点を考慮すると,移流-拡散方程式をベースとした拡散現象のモデリングだけでなく,
多孔質体の間隙スケールの物性や幾何形状の情報から,巨視的な拡散挙動を予測し,各種の
輸送係数を予測することができれば非常に有用と考えられる.また,そのようなアプローチが
可能となれば,計測した輸送係数から多孔質体の微視的性状を逆解析する際にも役立つ.\\
\hspace{\parindent}
以上のことを踏まえ,本研究では,間隙スケールでの幾何形状や物性値をもとに,
不飽和多孔質体における巨視的な物質拡散挙動を評価するための数値解析
手法の開発を行う.具体的には,マルコフ連鎖モンテカルロ法を用い,任意の
形状と粒径分布をもつ固相粒子で構成される不飽和多孔質体の数値モデル
を作成する.作成した不飽和多孔質体モデルはランダムウォークによる分子拡散
シミュレーションに用い,マクロな拡散係数を求めるとともにマクロスケールでみた場合の
拡散が通常の拡散方程式に従うものであるか否かを調べる.
分子拡散による物質輸送は非常に遅いプロセスだが,そのメカニズムは拡散物質に
常に作用する.そこで本研究では,多孔質体における分子拡散による輸送特性を調べることを
第一に考え,間隙水の移動は考慮していない.また,多孔質構造の影響を見るためには,
複数の多孔質体モデル間での比較が必要となることから,計算負荷の低い2次元問題での
シミュレーションを行う.
%ただし,本論文で述べる方法は,原理的には3次元問題にもそのまま適用することができ,
%間隙水の移動による効果をランダムウォークに取り込むことは可能であることを指摘しておく.\\
ただし,本論文で述べる方法は,原理的には3次元問題にもそのまま適用することができる.
また,間隙水の移動による機械的分散効果も,拡散粒子の移動確率を異方的に
することで,ランダムウォーク・シミュレーションに取り込むことも可能である.
これらの拡張は,本稿で提案するシミュレーション手法をより現実的な問題へ
適用し,今後,実測データとの比較を行う上での次に取り組むべき課題である.\\
\hspace{\parindent}
以下,本論文では,第2節において不飽和多孔質体における2次元拡散問題の設定を示し,
続く第3節において,本研究で用いる一連の数値解析手法について述べる.
第4節では,ランダムウォークによる数値拡散解析に用いたシミュレーションモデルとその
諸元,各種解析条件を示す.第5節では,シミュレーション結果を示し,不飽和多孔質体
におけるマクロ拡散挙動に与える,固相粒子の充填構造,粒子形状,粒径分布と飽和度の
影響を調べる.最後に,本研究のまとめを今後の課題とともに述べる.\\
\hspace{\parindent}
本研究と同様なアプローチは,多孔質体における熱や物質の輸送特性を調べることを
目的とした研究にこれまでにも用いられている.例えば,X線CT試験や統計的な方法で作成した
多孔質体モデルは,間隙中の流速や物質輸送特性を調べるためにしばしば用いられてきた
\cite{Liang}$^-$\cite{Narsilio}.また,間隙水の配置を決定するためにモンテカルロ法を
用いた例も,これまでの研究でいくつか報告が行われている\cite{Berkowitz,MC}.
本研究は,これらの既往の研究で提案された手法とアイデアに負うところが大きく,個々の
数値解析手法を新規に開発するものでない.しかしながら,本研究独自の取り組みと
貢献は以下の3点にあると考えられる.1点目として,本研究では任意の形状と粒径分布を
持つ固相粒子が充填された多孔質構造モデルを作成して,マクロ拡散挙動を調べていること
が挙げられる.
これにより,水分量だけでなく,固相構造が間隙水中の物質拡散挙動に与える影響を詳細に調べることが
可能となっている.2点目は,マクロ拡散係数と水分量の関係だけでなく,異常拡散の
程度が飽和度に対してどのように変化するかを調べた点にある.また3点目として,固相粒子の形状
と粒径分布,充填構造が拡散挙動に与える影響を調べた結果,異常拡散の起源
が拡散経路の屈曲にあることを明らかにしたことが挙げられる.これら第2,第3点目に述べた知見は,
本研究で行った数値シミュレーションの結果として因果関係が明らかになったという点で
独自の貢献といえる.
