%\section{はじめに}
弾性波試験は材料物性や形状の非破壊評価にしばしば用いられている.建設材料の多くは,物性や形状が不均一なランダムかつ多孔質な媒体である.例えば,岩石は,複数の鉱物種で構成された,き裂やボイドを含む典型的なランダム多孔質媒体である.このような不均質媒体では,弾性波は屈折や散乱により非常に複雑な伝播挙動を示す.そのため,不均質材の弾性波試験の高精度化や信頼性の担保には,材料の不均質性を考慮した波動伝播モデルによるデータ解釈が必要となる.またそのようなモデルは,媒体のランダム性も適切に表現できるものでなければならない.波動媒体の不均質性を表現する際,物性値を平均と偏差成分に分け,偏差成分が既知の確率分布に従うとして解析が行われる.例えば,時間調和な波動場を考える場合,支配波動方程式に含まれる波数の項を,平均値と確率変数としての偏差成分に分離して記述する.このとき,波数の偏差成分がどのような確率分布に従うかは自明でなく,材料に応じて異なる可能性がある.また,確率分布が材料のどのよう性質を反映して決まるかということも,弾性波試験の観点からは重要である.これらのことは,理論解析だけで明らかにすることはできず,現実の材料をランダム不均質媒体としてモデル化するための確率分布は,弾性波の実測データを基に構築する必要がある.
 以上を踏まえ本研究では,花崗岩試料用いた超音波計測を行い,実測データから波数ベクトルが従う確率分布を推定する.花崗岩は数mm~数cm程度の結晶粒から成る多結晶質体で,超音波はき裂や粒界で散乱を起こす.そのため,材料の不均質部と強く相互作用する弾性波を室内試験で観測するための実験供試体として利用し易い.ここでは,圧電超音波探触子で花崗岩試料に表面波を励起し,その伝播挙動をレーザー振動計で可視化するとともに,波動場の位相と波数の構造を調べる.以下,超音波計測方法と計測結果,推定した位相の空間分布と,波数ベクトルの確率密度分布を順に示す.
\cite{Fujinawa})
\hspace{\parindent}
以上のことを踏まえ,本研究では,間隙スケールでの幾何形状や物性値をもとに,
不飽和多孔質体における巨視的な物質拡散挙動を評価するための数値解析
手法の開発を行う.具体的には,マルコフ連鎖モンテカルロ法を用い,任意の
形状と粒径分布をもつ固相粒子で構成される不飽和多孔質体の数値モデル
を作成する.作成した不飽和多孔質体モデルはランダムウォークによる分子拡散
シミュレーションに用い,マクロな拡散係数を求めるとともにマクロスケールでみた場合の
拡散が通常の拡散方程式に従うものであるか否かを調べる.
分子拡散による物質輸送は非常に遅いプロセスだが,そのメカニズムは拡散物質に
常に作用する.そこで本研究では,多孔質体における分子拡散による輸送特性を調べることを
第一に考え,間隙水の移動は考慮していない.また,多孔質構造の影響を見るためには,
複数の多孔質体モデル間での比較が必要となることから,計算負荷の低い2次元問題での
シミュレーションを行う.
%ただし,本論文で述べる方法は,原理的には3次元問題にもそのまま適用することができ,
%間隙水の移動による効果をランダムウォークに取り込むことは可能であることを指摘しておく.\\
ただし,本論文で述べる方法は,原理的には3次元問題にもそのまま適用することができる.
また,間隙水の移動による機械的分散効果も,拡散粒子の移動確率を異方的に
することで,ランダムウォーク・シミュレーションに取り込むことも可能である.
これらの拡張は,本稿で提案するシミュレーション手法をより現実的な問題へ
適用し,今後,実測データとの比較を行う上での次に取り組むべき課題である.\\
\hspace{\parindent}
以下,本論文では,第2節において不飽和多孔質体における2次元拡散問題の設定を示し,
続く第3節において,本研究で用いる一連の数値解析手法について述べる.
第4節では,ランダムウォークによる数値拡散解析に用いたシミュレーションモデルとその
諸元,各種解析条件を示す.第5節では,シミュレーション結果を示し,不飽和多孔質体
におけるマクロ拡散挙動に与える,固相粒子の充填構造,粒子形状,粒径分布と飽和度の
影響を調べる.最後に,本研究のまとめを今後の課題とともに述べる.\\
\hspace{\parindent}
本研究と同様なアプローチは,多孔質体における熱や物質の輸送特性を調べることを
目的とした研究にこれまでにも用いられている.例えば,X線CT試験や統計的な方法で作成した
多孔質体モデルは,間隙中の流速や物質輸送特性を調べるためにしばしば用いられてきた
\cite{Liang}$^-$\cite{Narsilio}.また,間隙水の配置を決定するためにモンテカルロ法を
用いた例も,これまでの研究でいくつか報告が行われている\cite{Berkowitz,MC}.
本研究は,これらの既往の研究で提案された手法とアイデアに負うところが大きく,個々の
数値解析手法を新規に開発するものでない.しかしながら,本研究独自の取り組みと
貢献は以下の3点にあると考えられる.1点目として,本研究では任意の形状と粒径分布を
持つ固相粒子が充填された多孔質構造モデルを作成して,マクロ拡散挙動を調べていること
が挙げられる.
これにより,水分量だけでなく,固相構造が間隙水中の物質拡散挙動に与える影響を詳細に調べることが
可能となっている.2点目は,マクロ拡散係数と水分量の関係だけでなく,異常拡散の
程度が飽和度に対してどのように変化するかを調べた点にある.また3点目として,固相粒子の形状
と粒径分布,充填構造が拡散挙動に与える影響を調べた結果,異常拡散の起源
が拡散経路の屈曲にあることを明らかにしたことが挙げられる.これら第2,第3点目に述べた知見は,
本研究で行った数値シミュレーションの結果として因果関係が明らかになったという点で
独自の貢献といえる.
