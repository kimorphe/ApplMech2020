建設分野において弾性波検査の対象となる材料の多くは,形状や物性が不規則かつ不均一なランダム媒体である.
例えば,コンクリートは,粒径や形状が異なる骨材と気泡がランダムに分布した非均質媒体である.
また,岩盤や岩石も断層や節理系から,岩石を構成する鉱物粒や介在物,マイクロクラックに至るまで,
各種の空間スケールで多様な非均質性を有する\cite{RockPhys}.
このような非均質ランダム媒体において,弾性波は不均質部との相互作用によって散乱や屈折を起こし,
複雑な伝播挙動を示す.そのため,地震探査や岩石コア,多結晶質材の弾性波検査やイメージング
には,緻密な金属材料のような均質材に対する非破壊検査には無い困難が伴う\cite{Sato,Borcea,Thompson}.
特に,弾性波が強い多重散乱を起こしながら媒体を伝播する場合,著しい減衰や波形の変化のために,
計測で得られた波形から有用な情報を取り出すことが一般に難しい.このような困難を克服し,
ランダム不均質媒体に対する信頼性や精度の高い弾性波検査技術を開発するには,多重散乱効果を
考慮した波動伝播モデルの構築が必要となる.

物理探査や非破壊検査において反射源位置を特定する際,弾性波速度が既知である必要がある\cite{Etgen, Schmitz}.
この理由から,種々の弾性波伝播特性の中でも伝播速度は重要と言え,このことは不均質媒体
でも,少なくとも平均的な弾性波速度が必要となる点では同様である\cite{Langenberg, Bleistein}.
ただし,ランダム媒体の場合,媒体の物性値が場所によって異なるため,計測点毎に弾性波の到達時間と,
そこから見積もられる弾性波速度には必然的にばらつきが生じる.弾性波速度のばらつきは反射源位置の同定精度と
不確実性に影響するため,ランダム媒体に関しては弾性波速度の平均値だけでなくばらつきも重要な情報となる.
また,弾性波速度のばらつきは媒体の不均質性を反映したものであることから,ランダムな不均質性を弾性波計測
データから調べる目的においては,速度のばらつきを含め,弾性波速度や到達時間が従う統計分布自体が
興味の対象となる\cite{Yu,Li}.

ランダム不均質媒体における弾性波速度のばらつきを調べるために,これまで,種々の理論,数値解析および実験的
研究が行われてきた\cite{NishizawaI}.例えば,理論および数値解析的な研究には,波線理論や1次散乱理論を用いて
伝播時間解析を行ったもの\cite{Muller, Korn, Spetzler2001}や,差分法モデルでランダム媒体中を伝播する波動の
解析を行いその結果を理論解析と比較したもの\cite{Spetzler}などがある.一方,実験的な研究には,
岩石試料を透過する超音波をレーザードップラー振動計で計測し,弾性波速度のばらつきと鉱物粒径との関係を調べたもの
\cite{Nishizawa1996,Nishizawa2001}や,個々の計測波形と平均波形の乖離やP-S波間でのエネルギー分配の挙動を
不均質性スケールとの関係で調べたものなど\cite{Sivaji,Fukushima}がある.
これら実験的な研究の成果は,弾性波計測結果から不均質性のスケールや強度を推定する上で重要なものと言える.
一方で,伝播時間や伝播速度の揺らぎと,伝播距離や方向の関係は実験的には調べられておらず,
例えば波線理論や散乱理論による予測と一致するかどうかはこれまで明らかにされていない.

以上を踏まえ本研究では,弾性波伝播時間のばらつきが伝播距離に応じてどのように変化するかを明らかにすることを目的に,
超音波計測を実施する.実験では,典型的なランダム不均質媒体である花崗岩を供試体として用い,
圧電トランスデューサで励起した表面波の振動を,レーザードップラー振動計(LDV)で多点計測する.
LDVを用いて表面波を対象とした計測を行う理由は,試料表面の超音波振動を高い時空間解像度と広い周波数帯域で
観測することにより,波動場の伝播状況を精確に捉えることを意図したものである.
超音波の送信には,接触型の線集束トランスデューサを用い試料内部に円筒波を励起する.
これにより,強い超音波を送信できるだけでなく,入射方向と伝播距離を明確に定義することが可能となる.
一連の計測で得られた波形は,周波数領域において解析し,フェルマーの原理に基づき各観測点における到達時間を求める.
このようにして得られた到達時間のアンサンブルから,到達時間の確率分布を,伝播距離の関数として求める.
最後に,到達時間の平均と標準偏差を評価し,到達時間のばらつきが距離に応じてどのような法則に従い変化するかを
明らかにする.

以下では,はじめに超音波計測の方法について述べる.次に,計測で得られた波形データから表面振動の様子を可視化し,
どのような波動場が供試体表面に形成されているかを示す.続いて,各観測点と周波数における到達時間を求める波形解析
方法を示し,計測波形から求めた到達時間の空間分布を示す.最後に,到達時間の確率分布とその平均,標準偏差を
伝播距離の関数として求めた結果を示し,到達時間の不確実性が空間的にどのように発展するかを考察する.
